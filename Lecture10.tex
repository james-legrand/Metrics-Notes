\documentclass[DIV=14,titlepage=false]{scrreprt}

%%%%%%%%%%%%%%%%%%%%%%%%%%%%%%%%%%%%%%%%%%%%%%%%%%%%%%%%%%%%%%%%%%%%%%%%%%%%%%%
%                                Basic Packages                               %
%%%%%%%%%%%%%%%%%%%%%%%%%%%%%%%%%%%%%%%%%%%%%%%%%%%%%%%%%%%%%%%%%%%%%%%%%%%%%%%
% Gives us multiple colors.
\usepackage[usenames,dvipsnames,pdftex]{xcolor}
% Lets us style link colors.
\usepackage{hyperref}
% Lets us import images and graphics.
\usepackage{graphicx}
% Lets us use figures in floating environments.
\usepackage{float}
% Lets us create multiple columns.
\usepackage{multicol}
% Gives us better math syntax.
\usepackage{amsmath,amsfonts,mathtools,amsthm,amssymb}
% Lets us strikethrough text.
\usepackage{cancel}
% Lets us edit the caption of a figure.
\usepackage{caption}
% Lets us import pdf directly in our tex code.
\usepackage{pdfpages}
% Lets us do algorithm stuff.
\usepackage[ruled,vlined,linesnumbered]{algorithm2e}
% Gets rid of some errors.
\usepackage{scrhack}
\def\class{article}
\usepackage{geometry}
\geometry{margin=0.9in}
%%%%%%%%%%%%%%%%%%%%%%%%%%%%%%%%%%%%%%%%%%%%%%%%%%%%%%%%%%%%%%%%%%%%%%%%%%%%%%%
%                                Basic Settings                               %
%%%%%%%%%%%%%%%%%%%%%%%%%%%%%%%%%%%%%%%%%%%%%%%%%%%%%%%%%%%%%%%%%%%%%%%%%%%%%%%

%%%%%%%%%%%%%
%  Symbols  %
%%%%%%%%%%%%%

\let\implies\Rightarrow
\let\impliedby\Leftarrow
\let\iff\Leftrightarrow
\let\epsilon\varepsilon

%%%%%%%%%%%%
%  Tables  %
%%%%%%%%%%%%

\setlength{\tabcolsep}{5pt}
\renewcommand\arraystretch{1.5}

%%%%%%%%%%%%%%%%%%%%%%%
%  Center Title Page  %
%%%%%%%%%%%%%%%%%%%%%%%

\usepackage{titling}
\renewcommand\maketitlehooka{\null\mbox{}\vfill}
\renewcommand\maketitlehookd{\vfill\null}

%%%%%%%%%%%%%%%%%%%%%%%%%%%%%%%%%%%%%%%%%%%%%%%%%%%%%%%
%  Create a grey background in the middle of the PDF  %
%%%%%%%%%%%%%%%%%%%%%%%%%%%%%%%%%%%%%%%%%%%%%%%%%%%%%%%

\usepackage{eso-pic}
\newcommand\definegraybackground{
  \definecolor{reallylightgray}{HTML}{FAFAFA}
  \AddToShipoutPicture{
    \ifthenelse{\isodd{\thepage}}{
      \AtPageLowerLeft{
        \put(\LenToUnit{\dimexpr\paperwidth-222pt},0){
          \color{reallylightgray}\rule{222pt}{297mm}
        }
      }
    }
    {
      \AtPageLowerLeft{
        \color{reallylightgray}\rule{222pt}{297mm}
      }
    }
  }
}

%%%%%%%%%%%%%%%%%%%%%%%%
%  Modify Links Color  %
%%%%%%%%%%%%%%%%%%%%%%%%

\hypersetup{
  % Enable highlighting links.
  colorlinks,
  % Change the color of links to blue.
  linkcolor={black},
  % Change the color of citations to black.
  citecolor={black},
  % Change the color of url's to blue with some black.
  urlcolor=blue
}

%%%%%%%%%%%%%%%%%%
% Fix WrapFigure %
%%%%%%%%%%%%%%%%%%

\newcommand{\wrapfill}{\par\ifnum\value{WF@wrappedlines}>0
    \parskip=0pt
    \addtocounter{WF@wrappedlines}{-1}%
    \null\vspace{\arabic{WF@wrappedlines}\baselineskip}%
    \WFclear
\fi}

%%%%%%%%%%%%%%%%%
% Multi Columns %
%%%%%%%%%%%%%%%%%

\let\multicolmulticols\multicols
\let\endmulticolmulticols\endmulticols

\RenewDocumentEnvironment{multicols}{mO{}}
{%
  \ifnum#1=1
    #2%
  \else % More than 1 column
    \multicolmulticols{#1}[#2]
  \fi
}
{%
  \ifnum#1=1
\else % More than 1 column
  \endmulticolmulticols
\fi
}

\newlength{\thickarrayrulewidth}
\setlength{\thickarrayrulewidth}{5\arrayrulewidth}

%%%%%%%%%%%%%%%%%%%%
%  Import Figures  %
%%%%%%%%%%%%%%%%%%%%

\usepackage{import}
\pdfminorversion=7

% EXAMPLE:
% 1. \incfig{limit-graph}
% 2. \incfig[0.4]{limit-graph}
% Parameters:
% 1. The figure name. It should be located in figures/NAME.tex_pdf.
% 2. (Optional) The width of the figure. Example: 0.5, 0.35.
\newcommand\incfig[2][1]{%
  \def\svgwidth{#1\columnwidth}
  \import{./figures/}{#2.pdf_tex}
}

\begingroup\expandafter\expandafter\expandafter\endgroup
\expandafter\ifx\csname pdfsuppresswarningpagegroup\endcsname\relax
\else
  \pdfsuppresswarningpagegroup=1\relax
\fi

%%%%%%%%%%%%%
%  Correct  %
%%%%%%%%%%%%%

% EXAMPLE:
% 1. \correct{INCORRECT}{CORRECT}
% Parameters:
% 1. The incorrect statement.
% 2. The correct statement.
\definecolor{correct}{HTML}{009900}
\newcommand\correct[2]{{\color{red}{#1 }}\ensuremath{\to}{\color{correct}{ #2}}}



\newcommand{\R}{\mathbb{R}}
\newcommand{\Z}{\mathbb{Z}}
\newcommand{\E}{\mathbb{E}}
\newcommand{\B}{\ensuremath{\mathcal{B}}}
\newcommand{\X}{\ensuremath{\mathcal{X}}}
\newcommand{\Y}{\ensuremath{\mathcal{Y}}}
\newcommand{\mA}{\ensuremath{\mathbf{A}}}
\newcommand{\mB}{\ensuremath{\mathbf{B}}}
\newcommand{\mC}{\ensuremath{\mathbf{C}}}
\newcommand{\mD}{\ensuremath{\mathbf{D}}}
\newcommand{\mX}{\ensuremath{\mathbf{X}}}
\newcommand{\mY}{\ensuremath{\mathbf{Y}}}
\newcommand{\mx}{\ensuremath{\mathbf{x}}}
\newcommand{\my}{\ensuremath{\mathbf{y}}}
\newcommand{\mI}{\ensuremath{\mathbf{I}}}
\newcommand{\mi}{\ensuremath{\mathbf{\iota}}}
\newcommand{\mmu}{\ensuremath{\mathbf{\mu}}}
\newcommand{\mc}{\ensuremath{\mathbf{c}}}
\newcommand{\mSigma}{\ensuremath{\mathbf{\Sigma}}}
\newcommand{\mzero}{\ensuremath{\mathbf{0}}}
\newcommand{\independent}{\perp\!\!\!\!\perp} 
\setlength{\parindent}{0pt}
%%%%%%%%%%%%%%%%%%%%%%%%%%%%%%%%%%%%%%%%%%%%%%%%%%%%%%%%%%%%%%%%%%%%%%%%%%%%%%%
%                                 Environments                                %
%%%%%%%%%%%%%%%%%%%%%%%%%%%%%%%%%%%%%%%%%%%%%%%%%%%%%%%%%%%%%%%%%%%%%%%%%%%%%%%

\usepackage{varwidth}
\usepackage{thmtools}
\usepackage[most,many,breakable]{tcolorbox}

\tcbuselibrary{theorems,skins,hooks}
\usetikzlibrary{arrows,calc,shadows.blur}

%%%%%%%%%%%%%%%%%%%
%  Define Colors  %
%%%%%%%%%%%%%%%%%%%

\definecolor{myblue}{RGB}{45, 111, 177}
\definecolor{mygreen}{RGB}{56, 140, 70}
\definecolor{myred}{RGB}{199, 68, 64}
\definecolor{mypurple}{RGB}{197, 92, 212}

\definecolor{definition}{HTML}{228b22}
\definecolor{theorem}{HTML}{00007B}
\definecolor{example}{HTML}{2A7F7F}
\definecolor{definition}{HTML}{228b22}
\definecolor{prop}{HTML}{191971}
\definecolor{lemma}{HTML}{983b0f}
\definecolor{exercise}{HTML}{88D6D1}

\colorlet{definition}{mygreen!85!black}
\colorlet{claim}{mygreen!85!black}
\colorlet{corollary}{mypurple!85!black}
\colorlet{proof}{theorem}

%%%%%%%%%%%%%%%%%%%%%%
%  Helpful Commands  %
%%%%%%%%%%%%%%%%%%%%%%

% EXAMPLE:
% 1. \createnewtheoremstyle{thmdefinitionbox}{}{}
% 2. \createnewtheoremstyle{thmtheorembox}{}{}
% 3. \createnewtheoremstyle{thmproofbox}{qed=\qedsymbol}{
%       rightline=false, topline=false, bottomline=false
%    }
% Parameters:
% 1. Theorem name.
% 2. Any extra parameters to pass directly to declaretheoremstyle.
% 3. Any extra parameters to pass directly to mdframed.
\newcommand\createnewtheoremstyle[3]{
  \declaretheoremstyle[
  headfont=\bfseries\sffamily, bodyfont=\normalfont, #2,
  mdframed={
    #3,
  },
  ]{#1}
}

% EXAMPLE:
% 1. \createnewcoloredtheoremstyle{thmdefinitionbox}{definition}{}{}
% 2. \createnewcoloredtheoremstyle{thmexamplebox}{example}{}{
%       rightline=true, leftline=true, topline=true, bottomline=true
%     }
% 3. \createnewcoloredtheoremstyle{thmproofbox}{proof}{qed=\qedsymbol}{backgroundcolor=white}
% Parameters:
% 1. Theorem name.
% 2. Color of theorem.
% 3. Any extra parameters to pass directly to declaretheoremstyle.
% 4. Any extra parameters to pass directly to mdframed.
\newcommand\createnewcoloredtheoremstyle[4]{
  \declaretheoremstyle[
  headfont=\bfseries\sffamily\color{#2}, bodyfont=\normalfont, #3,
  mdframed={
    linewidth=2pt,
    rightline=false, leftline=true, topline=false, bottomline=false,
    linecolor=#2, backgroundcolor=#2!5, #4,
  },
  ]{#1}
}

%%%%%%%%%%%%%%%%%%%%%%%%%%%%%%%%%%%
%  Create the Environment Styles  %
%%%%%%%%%%%%%%%%%%%%%%%%%%%%%%%%%%%

\makeatletter
\@ifclasswith\class{nocolor}{
  % Environments without color.

  \createnewtheoremstyle{thmdefinitionbox}{}{}
  \createnewtheoremstyle{thmtheorembox}{}{}
  \createnewtheoremstyle{thmexamplebox}{}{}
  \createnewtheoremstyle{thmclaimbox}{}{}
  \createnewtheoremstyle{thmcorollarybox}{}{}
  \createnewtheoremstyle{thmpropbox}{}{}
  \createnewtheoremstyle{thmlemmabox}{}{}
  \createnewtheoremstyle{thmexercisebox}{}{}
  \createnewtheoremstyle{thmdefinitionbox}{}{}
  \createnewtheoremstyle{thmquestionbox}{}{}
  \createnewtheoremstyle{thmsolutionbox}{}{}

  \createnewtheoremstyle{thmproofbox}{qed=\qedsymbol}{}
  \createnewtheoremstyle{thmexplanationbox}{}{}
}{
  % Environments with color.

  \createnewcoloredtheoremstyle{thmdefinitionbox}{definition}{}{}
  \createnewcoloredtheoremstyle{thmtheorembox}{theorem}{}{}
  \createnewcoloredtheoremstyle{thmexamplebox}{example}{}{
    rightline=true, leftline=true, topline=true, bottomline=true
  }
  \createnewcoloredtheoremstyle{thmclaimbox}{claim}{}{}
  \createnewcoloredtheoremstyle{thmcorollarybox}{corollary}{}{}
  \createnewcoloredtheoremstyle{thmpropbox}{prop}{}{}
  \createnewcoloredtheoremstyle{thmlemmabox}{lemma}{}{}
  \createnewcoloredtheoremstyle{thmexercisebox}{exercise}{}{}

  \createnewcoloredtheoremstyle{thmproofbox}{proof}{qed=\qedsymbol}{backgroundcolor=white}
  \createnewcoloredtheoremstyle{thmexplanationbox}{example}{qed=\qedsymbol}{backgroundcolor=white}
}
\makeatother

%%%%%%%%%%%%%%%%%%%%%%%%%%%%%
%  Create the Environments  %
%%%%%%%%%%%%%%%%%%%%%%%%%%%%%

\declaretheorem[numberwithin=section, style=thmtheorembox,     name=Theorem]{theorem}
\declaretheorem[numbered=no,          style=thmexamplebox,     name=Example]{example}
\declaretheorem[numberwithin=section, style=thmclaimbox,       name=Claim]{claim}
\declaretheorem[numberwithin=section, style=thmcorollarybox,   name=Corollary]{corollary}
\declaretheorem[numberwithin=section, style=thmpropbox,        name=Proposition]{prop}
\declaretheorem[numberwithin=section, style=thmlemmabox,       name=Lemma]{lemma}
\declaretheorem[numberwithin=section, style=thmexercisebox,    name=Exercise]{exercise}
\declaretheorem[numbered=no,          style=thmproofbox,       name=Proof]{replacementproof}
\declaretheorem[numbered=no,          style=thmexplanationbox, name=Proof]{expl}

\makeatletter
\@ifclasswith\class{nocolor}{
  % Environments without color.

  \newtheorem*{note}{Note}

  \declaretheorem[numberwithin=section, style=thmdefinitionbox, name=Definition]{definition}
  \declaretheorem[numberwithin=section, style=thmquestionbox,   name=Question]{question}
  \declaretheorem[numberwithin=section, style=thmsolutionbox,   name=Solution]{solution}
}{
  % Environments with color.

  \newtcbtheorem[number within=section]{Definition}{Definition}{
    enhanced,
    before skip=2mm,
    after skip=2mm,
    colback=red!5,
    colframe=red!80!black,
    colbacktitle=red!75!black,
    boxrule=0.5mm,
    attach boxed title to top left={
      xshift=1cm,
      yshift*=1mm-\tcboxedtitleheight
    },
    varwidth boxed title*=-3cm,
    boxed title style={
      interior engine=empty,
      frame code={
        \path[fill=tcbcolback]
        ([yshift=-1mm,xshift=-1mm]frame.north west)
        arc[start angle=0,end angle=180,radius=1mm]
        ([yshift=-1mm,xshift=1mm]frame.north east)
        arc[start angle=180,end angle=0,radius=1mm];
        \path[left color=tcbcolback!60!black,right color=tcbcolback!60!black,
        middle color=tcbcolback!80!black]
        ([xshift=-2mm]frame.north west) -- ([xshift=2mm]frame.north east)
        [rounded corners=1mm]-- ([xshift=1mm,yshift=-1mm]frame.north east)
        -- (frame.south east) -- (frame.south west)
        -- ([xshift=-1mm,yshift=-1mm]frame.north west)
        [sharp corners]-- cycle;
      },
    },
    fonttitle=\bfseries,
    title={#2},
    #1
  }{def}

  \NewDocumentEnvironment{definition}{O{}O{}}
    {\begin{Definition}{#1}{#2}}{\end{Definition}}

  \newtcolorbox{note}[1][]{%
    enhanced jigsaw,
    colback=gray!20!white,%
    colframe=gray!80!black,
    size=small,
    boxrule=1pt,
    title=\textbf{Note:-},
    halign title=flush center,
    coltitle=black,
    breakable,
    drop shadow=black!50!white,
    attach boxed title to top left={xshift=1cm,yshift=-\tcboxedtitleheight/2,yshifttext=-\tcboxedtitleheight/2},
    minipage boxed title=1.5cm,
    boxed title style={%
      colback=white,
      size=fbox,
      boxrule=1pt,
      boxsep=2pt,
      underlay={%
        \coordinate (dotA) at ($(interior.west) + (-0.5pt,0)$);
        \coordinate (dotB) at ($(interior.east) + (0.5pt,0)$);
        \begin{scope}
          \clip (interior.north west) rectangle ([xshift=3ex]interior.east);
          \filldraw [white, blur shadow={shadow opacity=60, shadow yshift=-.75ex}, rounded corners=2pt] (interior.north west) rectangle (interior.south east);
        \end{scope}
        \begin{scope}[gray!80!black]
          \fill (dotA) circle (2pt);
          \fill (dotB) circle (2pt);
        \end{scope}
      },
    },
    #1,
  }

  \newtcbtheorem{Question}{Question}{enhanced,
    breakable,
    colback=white,
    colframe=myblue!80!black,
    attach boxed title to top left={yshift*=-\tcboxedtitleheight},
    fonttitle=\bfseries,
    title=\textbf{Question:-},
    boxed title size=title,
    boxed title style={%
      sharp corners,
      rounded corners=northwest,
      colback=tcbcolframe,
      boxrule=0pt,
    },
    underlay boxed title={%
      \path[fill=tcbcolframe] (title.south west)--(title.south east)
      to[out=0, in=180] ([xshift=5mm]title.east)--
      (title.center-|frame.east)
      [rounded corners=\kvtcb@arc] |-
      (frame.north) -| cycle;
    },
    #1
  }{def}

  \NewDocumentEnvironment{question}{O{}O{}}
  {\begin{Question}{#1}{#2}}{\end{Question}}

  \newtcolorbox{Solution}{enhanced,
    breakable,
    colback=white,
    colframe=mygreen!80!black,
    attach boxed title to top left={yshift*=-\tcboxedtitleheight},
    title=\textbf{Solution:-},
    boxed title size=title,
    boxed title style={%
      sharp corners,
      rounded corners=northwest,
      colback=tcbcolframe,
      boxrule=0pt,
    },
    underlay boxed title={%
      \path[fill=tcbcolframe] (title.south west)--(title.south east)
      to[out=0, in=180] ([xshift=5mm]title.east)--
      (title.center-|frame.east)
      [rounded corners=\kvtcb@arc] |-
      (frame.north) -| cycle;
    },
  }

  \NewDocumentEnvironment{solution}{O{}O{}}
  {\vspace{-10pt}\begin{Solution}{#1}{#2}}{\end{Solution}}
}
\makeatother

%%%%%%%%%%%%%%%%%%%%%%%%%%%%
%  Edit Proof Environment  %
%%%%%%%%%%%%%%%%%%%%%%%%%%%%

\renewenvironment{proof}[1][\proofname]{\vspace{-10pt}\begin{replacementproof}}{\end{replacementproof}}
\newenvironment{explanation}[1][\proofname]{\vspace{-10pt}\begin{expl}}{\end{expl}}

\theoremstyle{definition}

\newtheorem*{notation}{Notation}
\newtheorem*{previouslyseen}{As previously seen}
\newtheorem*{problem}{Problem}
\newtheorem*{observe}{Observe}
\newtheorem*{property}{Property}
\newtheorem*{intuition}{Intuition}

\setuptoc{toc}{leveldown}

\begin{document}
\vspace{-10pt}
\setcounter{chapter}{9}

\chapter{Probit. Maximum Likelihood.}
\vspace{-10pt}
\section{Binary choice}
Suppose we don't have a continuous dependent variable, rather it is binary: $y_i = \{0,1\}$. We could still use OLS here, let's check out the assumptions:
\begin{itemize}
    \item [(OLS0)]  $(y_i, x_i)\text{ is an i.i.d. sequence}$
    \subitem \textcolor{teal}{$\checkmark$ Binary $y_i$ doesn't break this, we can still have an i.i.d. sequence}
    \item [(OLS1)] $ E(x_i x_i')$  is finite non-singular
    \subitem \textcolor{teal}{$\checkmark$ Binary $y_i$ doesn't affect this}
    \item [(OLS2)] $E(y_i|x_i) = x_i'\beta$
    \subitem \textcolor{magenta}{? $E(y_i|x_i)=1\times P(y_i=1|x_i)+0\times P(y_i=0|x_i)=P(y_i=1|x_i) \overset{?}{=}x_i'\beta$ 
    \subitem Hence, for OLS2 to hold we need use the linear probability model.}
    \item [(OLS3)] Var $(y_i|x_i)= \sigma^2$
    \subitem \textcolor{purple}{$\bigtimes$ $Var(y_i|x_i) = E(y_i^2|x_i)-E(y_i|x_i)^2=E(y_i|x_i)-E(y_i|x_i)^2=x_i'\beta(1-x_i'\beta)$
    \subitem using $y^2=y$. Hence OLS3 cannot hold, we do have heteroskedasticity.}
    \item [(OLS4)] $E\epsilon_i^4 < \infty, \quad E\|x_i\|^4 < \infty$
    \subitem \textcolor{teal}{$\checkmark$ May still hold}
\end{itemize}
We can fix the heteroskedasticity with GLS or White standard errors, but the linear probability model is more of a problem. This model does not restrict predicted probabilities to be between 0 and 1, and the use of any other model will violate OLS2 meaning OLS will not be consistent.\\
The standard alternative is to use a function of the form \[ P(y_i=1|x_i)=F(x_i'\beta)\] where F($\cdot$) is a known CDF, typically assumed to be symmetric about zero, so that $F(u)=1-F(-u)$. The standard choices for F are
\begin{itemize}
    \item Logistic: $F(u) = \frac{e^u}{1+e^u}$, known as the \textbf{logit} model
    \item Normal: $F(u) = \Phi(u)$, known as the \textbf{probit} model
\end{itemize}
This is identical to the latent variable model
\begin{align*}
    y_i*&=x_i'\beta+\epsilon_i\\
    \epsilon_i &\sim F(\cdot)\\
    y_i &= \begin{cases}
        1 \quad \text{ if } y_i*>0\\
        0 \quad \text{otherwise}
    \end{cases}
\end{align*}
Since then
\begin{align*}
    P(y_i=1|x_i)&=P(y_i*>0|x_i)\\
    &= P(x_i'\beta+\epsilon_i>0|x_i)\\
    &= P(\epsilon_i>-x_i\beta |x_i)\\
    &= 1-F(-x_i'\beta)\\
    &= F(x_i'\beta)
\end{align*}
\section{Maximum likelihood estimation}
The probit model is typically estimated by the method of maximum likelihood (ML). Consider the typical setup:
\begin{align*}
    z_1,\dots, z_n \overset{i.i.d.}{\sim} f(\cdot|\theta) \quad & \rightarrow \quad L(\theta) = \prod_{i=1}^{n}f(z_i|\theta)\\
    \log L(\theta) = \ell(\theta)  \quad &=  \quad \sum_{i=1}^{n}\log f(z_i|\theta)\\
    \hat\theta_{ML  \quad} &=  \quad \arg \underset{\theta}{\max}\ell(\theta)
\end{align*}
This is known as a \textit{parametric model}, it requires the specification of the distribution of the data up to an unknown parameter $\theta$.\\
A key property is that the expected log-likelihood is maximised at the true value of the parameter vector $\theta_0$. Set $Z= (z_1,\dots,z_n)$.
\begin{theorem}
    $\theta_0 = \arg \underset{\theta}{\max} \E(\log L (\theta)|Z)$
\end{theorem}
The proof is presented in Lecture 11 using KL divergence. This motivates estimating $\theta$ by finding the value which maximises log-likelihood.
\begin{example}[OLS using MLE]
    \[
      f(Y_1,\dots, Y_n|X,\beta,\sigma^2) \quad : \quad L = \prod_{i=1}^{n}\frac{1}{\sqrt{2 \pi \sigma^2}}e^{-\frac{(Y_i-X_i'\beta)^2}{2\sigma^2}}  
    \]
    \begin{align*}
    \implies \ell = \log L &=\sum_{i=1}^{n}\log\left(\frac{1}{\sqrt{2 \pi \sigma^2}}\right) -\frac{(Y_i-X_i'\beta)^2}{2\sigma^2}\\
    &= n \log\left(\frac{1}{\sqrt{2 \pi \sigma^2}}\right) - \frac{\sum_{i=1}^{n}(Y_i-X_i'\beta)^2}{2\sigma^2}\\
    &= \frac{n}{2}\log(2\pi) - \frac{n}{2}\log(\sigma^2) - \frac{\sum_{i=1}^{n}(Y_i-X_i'\beta)^2}{2\sigma^2}
    \end{align*}
Hence, the FOCs are:
\begin{align}
    \frac{\partial \ell}{\partial \beta} &= - \frac{\sum_{i=1}^{n}(-X_i)(Y_i-X_i'\beta)}{\sigma^2}=0\\
    \frac{\partial ell}{\partial \sigma^2} &= -\frac{n}{2\sigma^2 } + \frac{\sum_{i=1}^{n}(Y_i-X_i'\beta)^2}{2\sigma^4}=0
\end{align}
\begin{minipage}{.5\textwidth}
\begin{align*}
    (10.1) \quad& \implies \sum_{i=1}^{n}X_iY_i-X_iX_i'\hat\beta_{ML}=0\\
    & \implies X'Y -X'X\hat\beta_{ML}=0\\
    & \implies\hat\beta_{ML} = (X'X)^{-1}X'Y = \hat\beta_{OLS}
\end{align*}
\end{minipage}
\begin{minipage}{.5\textwidth}
    \begin{align*}
        (10.2) \quad& \implies  n \sigma^2 = \sum_{i=1}^{n}(Y_i-X_i'\hat\beta_{ML})^2\\
        & \implies \sigma^2 = \frac{1}{n}\sum_{i=1}^{n}(Y_i-X_i'\hat\beta_{ML})^2
    \end{align*}
\end{minipage}
Thus, $\hat\beta_{OLS}$ is actually the MLE for $\beta$, so it has the desirable properties discussed in Lecture 11. However, the ML estimator for the variance is biased due to not correcting for the loss in degrees of freedom from estimating $\hat\beta_{ML}$.
\end{example}

Consider the problem of estimating $\theta$ if you have a vector of data $Z$ with the joint density of its elements given by $f(z|\theta)$.
\begin{definition}[Score]
    The score of the likelihood function is the vector of partial derivatives with respect to the parameters. 
    \[\frac{\partial}{\partial \theta} \log f(Z|\theta) \]
\end{definition}

\begin{theorem}
    If $\log f(Z|\theta)$ is second differentiable and the support of $Z$ doesn't depend on $\theta$ then the score has mean zero:
    \[\E \left[\frac{\partial}{\partial \theta} \log f(Z|\theta)\right]=0\]
\end{theorem}
\begin{proof}
    \begin{align*}
        \E \left[\frac{\partial}{\partial \theta} \log f(Z|\theta)\right] &= \int_{\R} \frac{\frac{\partial}{\partial \theta}f(z|\theta)}{f(z|\theta)}f(z|\theta)dz\\
        &= \frac{\partial}{\partial \theta}\int_{\R}f(z|\theta)dz\\
        &= \frac{\partial}{\partial \theta} 1\\
        &=0
    \end{align*}
\end{proof}
\begin{definition}[Fisher information]
    The covariance matrix of the score  is known as the Fisher information\[ I(\theta)=Var\left(\frac{\partial}{\partial \theta}\log f(Z|\theta)\right)=\E\left[\left(\frac{\partial}{\partial \theta}\log f(Z|\theta)\right)^2|\theta\right]\]
\end{definition}
\begin{note}
Because the likelihood of $\theta$ given $Z$ is always proportional to the probability $f(Z|\theta)$; their logarithms necessarily differ by a constant that is independent of $\theta$, and the derivatives are necessarily equal. Thus one can substitute in $\log L(\theta) = \ell (\theta)$ for $\log f(Z|\theta)$ in the above definitions.
\end{note}

The Fisher information is a way of measuring the amount of information that an observable $Z$ carries about the unknown parameter $\theta$. If $f$ is sharply peaked with respect to changes in $\theta$, it is easy to indicate the "correct" value of $\theta$ from the data, or equivalently, that the data $Z$ provides a lot of information about the parameter $\theta$. If $f$ is flat and spread-out, then it would take many samples of $Z$ to estimate the true value of $\theta$.
Note that $I(\theta)\geq0$. Near the ML estimate, low Fisher information suggests the maximum appears flat, that is, there are many nearby values with similar log-likelihood. Conversely, high Fisher information indicates the maximum is sharp.
\begin{claim}
If we have $n$ i.i.d. distributions (from n samples) then the Fisher information will be $n$ times the Fisher information of a single sample from the common distribution.
    \[I_n(\theta) = nI_1(\theta)\]
\end{claim}

\begin{lemma}[Information equality]
    The variance of the score is equal to the negative expected value of the Hessian matrix of the log-likelihood.
    \[ I(\theta)=Var\left(\frac{\partial}{\partial \theta}\log f(Z|\theta)\right)=-\E\left(\frac{\partial^2}{\partial \theta \partial \theta'}\log f(Z|\theta)\right)\]
\end{lemma}

\begin{proof}
    Let $\boldsymbol{Z}$ be an $m$-component column vector of random variables, not necessarily i.i.d. To ease notation, we denote their joint density as $f(\boldsymbol{Z}|\boldsymbol{\theta})\equiv f$. Also note all expectations are conditional on $\boldsymbol{\theta}$, and integrals are multiple integrals over $z_1, \dots, z_n$.
    \begin{align*}
        \mathbb{E} \frac{\partial^2 \log f}{\partial \boldsymbol{\theta} \partial \boldsymbol{\theta}'} &= \mathbb{E} \left[ \frac{\partial}{\partial \boldsymbol{\theta}}\left( \frac{\partial \log f}{\partial \boldsymbol{\theta}'}\right)\right]\\
        &= \mathbb{E} \left[ \frac{\partial}{\partial \boldsymbol{\theta}}\left( \frac{1}{f}\frac{\partial f}{\partial \boldsymbol{\theta}'}\right)\right]\\
        &= \mathbb{E}\left[-\frac{1}{f^2}\frac{\partial f}{\partial \boldsymbol{\theta}}\frac{\partial f}{\partial \boldsymbol{\theta}'}+\frac{1}{f}\frac{\partial^2f}{\partial \boldsymbol{\theta} \partial \boldsymbol{\theta}'}\right]\\
        &= -\mathbb{E}\left[\left(\frac{1}{f}\frac{\partial f}{\partial \boldsymbol{\theta}}\right)\left(\frac{1}{f}\frac{\partial f}{\partial \boldsymbol{\theta}'}\right)\right] + \mathbb{E}\left[\frac{1}{f}\frac{\partial^2f}{\partial \boldsymbol{\theta} \partial \boldsymbol{\theta}'}\right]
    \end{align*}
To obtain the information equality, we need to show the second term is zero.
\begin{align*}
    \mathbb{E}\left[\frac{1}{f}\frac{\partial^2f}{\partial \boldsymbol{\theta} \partial \boldsymbol{\theta}'}\right] &= \int_{\mathbb{R}} f \frac{1}{f}\frac{\partial^2f}{\partial \boldsymbol{\theta} \partial \boldsymbol{\theta}'} d\boldsymbol{Z}\\
    &= \int_{\mathbb{R}}\frac{\partial^2f}{\partial \boldsymbol{\theta} \partial \boldsymbol{\theta}'} d\boldsymbol{Z}\\
    &= \int_{\mathbb{R}}\frac{\partial}{\partial \boldsymbol{\theta}}\left(\frac{\partial f}{\partial \boldsymbol{\theta}'}\right) d\boldsymbol{Z}\\
    &= \frac{\partial}{\partial \boldsymbol{\theta}}\int_{\mathbb{R}}\frac{\partial f}{\partial \boldsymbol{\theta}'} d\boldsymbol{Z} \quad \text{(we can interchange these because we are economists)}\\
    &= \frac{\partial}{\partial \boldsymbol{\theta}}\frac{\partial}{\partial \boldsymbol{\theta}'}\int_{\mathbb{R}} f d\boldsymbol{Z} \quad \text{(what even is a regularity condition)}\\
    &= \frac{\partial}{\partial \boldsymbol{\theta}}\frac{\partial}{\partial \boldsymbol{\theta}'} 1\\
    &= 0
\end{align*}
\end{proof}

\begin{note}
    All we are assuming here is that we can interchange the order of differentiation and integration; a set of sufficient conditions for this are:
    \begin{enumerate}
        \item The function $\frac{\partial}{\partial \boldsymbol{\theta}}f(\boldsymbol{Z}|\boldsymbol{\theta})$ is continuous in $\boldsymbol{Z}$ and in $\boldsymbol{\theta} \in \Theta$ where $\Theta$ is an open set.
        \item The integral $\int f(\boldsymbol{Z}|\boldsymbol{\theta})d\boldsymbol{Z}$ exists.
        \item $\int\left|\frac{\partial}{\partial \boldsymbol{\theta}}f(\boldsymbol{Z}|\boldsymbol{\theta})\right|d\boldsymbol{Z}<M<\infty$ for all $\boldsymbol{\theta} \in \Theta$
    \end{enumerate}
\end{note}
\textbf{Misspecification and the information equality}\\
Suppose our random variables have joint density $f$ as before, but we specify that they have joint density $g$ instead. As before \[ \mathbb{E}_f \frac{\partial^2 \log g}{\partial \boldsymbol{\theta} \partial \boldsymbol{\theta}'} = -\mathbb{E}_f\left[\left(\frac{1}{g}\frac{\partial g}{\partial \boldsymbol{\theta}}\right)\left(\frac{1}{g}\frac{\partial g}{\partial \boldsymbol{\theta}'}\right)\right] + \mathbb{E}_f\left[\frac{1}{g}\frac{\partial^2g}{\partial \boldsymbol{\theta} \partial \boldsymbol{\theta}'}\right] \] where the $f$ subscript denotes the fact that we are taking the expectation with respect to the true distribution. Previously we made progress because the integrand contained $f\tfrac{1}{f}=1$, however we now have $f\tfrac{1}{g}$ which doesn't simplify. In general, under misspecification \[\mathbb{E}_f\left[\frac{1}{g}\frac{\partial^2g}{\partial \boldsymbol{\theta} \partial \boldsymbol{\theta}'}\right] \not = 0\] and the IE doesn't hold. Note: this does not exclude the possibility that this expected value is after all zero and the IE holds, it just generally isn't.
\begin{theorem}[Cramer-Rao lower bound]
    If $\boldsymbol{\tilde{\theta}}$ is an unbiased estimator of $\boldsymbol{\theta}$, then we have the following bound on its variance \[Var(\boldsymbol{\tilde{\theta}|Z})\geq [I(\boldsymbol{\theta})]^{-1}\]
\end{theorem}
These are both matrices, meaning this inequality tells us the difference between the left and right hand sides is positive semi-definite.\\
This result is similar to the Gauss-Markov theorem which established a lower bound for unbiased estimators in homoskedastic linear regression.
\begin{example}[Information bound for normal regression]
    We will apply the CRLB conditionally on X. Define the expected Hessian 
    \[ \E(H) = 
        \begin{bmatrix}
            \E \left(\frac{\partial^2\ell}{\partial \beta \partial \beta '}|X\right) & \E \left(\frac{\partial^2\ell}{\partial \beta \partial \sigma^2}|X\right)\\
            \E \left(\frac{\partial^2\ell}{\partial \sigma^2 \partial \beta' }|X\right) & \E \left(\frac{\partial^2\ell}{\partial \sigma^2 \partial \sigma^2}|X\right)
        \end{bmatrix}
    \]
    Recall the log likelihood
    \[
        \ell = \frac{n}{2}\log(2\pi) - \frac{n}{2}\log(\sigma^2) - \frac{\sum_{i=1}^{n}(Y_i-X_i'\beta)^2}{2\sigma^2}
    \]
    Thus we have second derivatives
    \begin{alignat*}{2}
        \frac{\partial^2\ell}{\partial \beta \partial \beta '} &= \frac{\partial}{\partial \beta'}\frac{\sum_{i=1}^{n}X_i(Y_i-X_i'\beta)}{\sigma^2} &&= -\frac{1}{\sigma^2} \sum_{i=1}^{n}X_iX_i' = \frac{1}{\sigma^2} X'X\\
        \frac{\partial^2\ell}{\partial \beta \partial \sigma^2} &= \frac{\partial}{\partial \sigma^2}\frac{\sum_{i=1}^{n}X_i(Y_i-X_i'\beta)}{\sigma^2} &&= -\frac{\sum_{i=1}^{n}X_i(Y_i-X_i'\beta)}{\sigma^4} = -\frac{1}{\sigma^4}X'(Y-X\beta)\\
        \frac{\partial^2\ell}{\partial \sigma^2 \partial \sigma^2} &= \frac{n}{2}\frac{1}{\sigma^4} -\frac{\sum_{i=1}^{n}(Y_i-X_i'\beta)^2}{\sigma^6} &&= \frac{n}{2}\frac{1}{\sigma^4} -\frac{1}{\sigma^6}(Y-X\beta)'(Y-X\beta)
    \end{alignat*}
    \begin{align*}
        \implies \E(H) &= \begin{bmatrix}
            \E\left[\frac{1}{\sigma^2} X'X|X\right] & \E\left[-\frac{1}{\sigma^4}X'(Y-X\beta)|X\right]\\
            \E\left[-\frac{1}{\sigma^4}X'(Y-X\beta)|X\right] & \E\left[\frac{n}{2}\frac{1}{\sigma^4} -\frac{1}{\sigma^6}(Y-X\beta)'(Y-X\beta)|X\right]
        \end{bmatrix}\\
        &=\begin{bmatrix}
            \frac{1}{\sigma^2} X'X & 0\\
            0 & \frac{n}{2}\frac{1}{\sigma^4} -\frac{n \sigma^2}{\sigma^6}
        \end{bmatrix}\\
        &=\begin{bmatrix}
            \frac{1}{\sigma^2} X'X & 0\\
            0 & -\frac{n}{2}\frac{1}{\sigma^4} 
        \end{bmatrix}
    \end{align*}
The block diagonal matrix can be inverted to find the lower bound on asymptotic conditional variance 
\[
    [I(\theta)]^{-1} = \begin{bmatrix}
        \sigma^2 (X'X)^{-1} & 0\\
        0 & \frac{2 \sigma^4}{n} 
    \end{bmatrix}
\]
\end{example}
The variance of $\hat \beta_{OLS}= \hat \beta_{ML}$ meets the CRLB. Thus we have the following theorem
\begin{theorem}
    In the normal regression, OLS is the Best Unbiased Estimator (BUE).
\end{theorem}
This result should be distinguished from the Gauss-Markov Theorem that $\hat\beta_{OLS}$ is minimum variance among those estimators that are unbiased and linear in $y$. Theorem 10.2.4 says that $\hat\beta_{OLS}$ is minimum variance in a larger class of estimators that includes non-linear unbiased estimators. This stronger statement is obtained under the normality assumption which is not assumed in the Gauss-Markov Theorem. Put differently, the Gauss-Markov Theorem does not exclude the possibility of some non-linear estimator beating OLS, but this possibility is ruled out by the normality assumption.

As we have already seen, the ML estimator of $\sigma^2$ is biased, so the CRLB does not apply. But the OLS estimator $\hat\sigma^2$ of $\sigma^2$ is unbiased, does it achieve the bound? We know $\frac{(n-k)\hat\sigma^2}{\sigma^2} \sim \chi^2(n-k)$, and $Var(\chi^2(p)=2p)$. Thus
\begin{align*}
    & Var\left(\frac{(n-k)\hat\sigma^2}{\sigma^2}\right)= 2(n-k)\\
    \implies& \frac{(n-k)^2}{\sigma^4}Var(\hat\sigma^2) = 2(n-k)\\
    \implies& Var(\hat\sigma^2) = \frac{2\sigma^4}{n-k}
\end{align*}
Therefore $\hat\sigma^2$ does not attain the CRLB $2\sigma^4/n$. However it can be shown that an unbiased estimator with variance lower than $\hat\sigma^2$ does not exist.
\end{document}