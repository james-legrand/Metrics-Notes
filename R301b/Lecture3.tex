\documentclass[DIV=14,titlepage=false]{scrreprt}
%%%%%%%%%%%%%%%%%%%%%%%%%%%%%%%%%%%%%%%%%%%%%%%%%%%%%%%%%%%%%%%%%%%%%%%%%%%%%%%
%                                Basic Packages                               %
%%%%%%%%%%%%%%%%%%%%%%%%%%%%%%%%%%%%%%%%%%%%%%%%%%%%%%%%%%%%%%%%%%%%%%%%%%%%%%%
% Gives us multiple colors.
\usepackage[usenames,dvipsnames,pdftex]{xcolor}
% Lets us style link colors.
\usepackage{hyperref}
% Lets us import images and graphics.
\usepackage{graphicx}
% Lets us use figures in floating environments.
\usepackage{float}
% Lets us create multiple columns.
\usepackage{multicol}
% Gives us better math syntax.
\usepackage{amsmath,amsfonts,mathtools,amsthm,amssymb}
% Lets us strikethrough text.
\usepackage{cancel}
% Lets us edit the caption of a figure.
\usepackage{caption}
% Lets us import pdf directly in our tex code.
\usepackage{pdfpages}
% Lets us do algorithm stuff.
\usepackage[ruled,vlined,linesnumbered]{algorithm2e}
% Gets rid of some errors.
\usepackage{scrhack}
\def\class{article}
\usepackage{geometry}
\geometry{margin=0.9in}
%%%%%%%%%%%%%%%%%%%%%%%%%%%%%%%%%%%%%%%%%%%%%%%%%%%%%%%%%%%%%%%%%%%%%%%%%%%%%%%
%                                Basic Settings                               %
%%%%%%%%%%%%%%%%%%%%%%%%%%%%%%%%%%%%%%%%%%%%%%%%%%%%%%%%%%%%%%%%%%%%%%%%%%%%%%%

%%%%%%%%%%%%%
%  Symbols  %
%%%%%%%%%%%%%

\let\implies\Rightarrow
\let\impliedby\Leftarrow
\let\iff\Leftrightarrow
\let\epsilon\varepsilon

%%%%%%%%%%%%
%  Tables  %
%%%%%%%%%%%%

\setlength{\tabcolsep}{5pt}
\renewcommand\arraystretch{1.5}

%%%%%%%%%%%%%%%%%%%%%%%
%  Center Title Page  %
%%%%%%%%%%%%%%%%%%%%%%%

\usepackage{titling}
\renewcommand\maketitlehooka{\null\mbox{}\vfill}
\renewcommand\maketitlehookd{\vfill\null}

%%%%%%%%%%%%%%%%%%%%%%%%%%%%%%%%%%%%%%%%%%%%%%%%%%%%%%%
%  Create a grey background in the middle of the PDF  %
%%%%%%%%%%%%%%%%%%%%%%%%%%%%%%%%%%%%%%%%%%%%%%%%%%%%%%%

\usepackage{eso-pic}
\newcommand\definegraybackground{
  \definecolor{reallylightgray}{HTML}{FAFAFA}
  \AddToShipoutPicture{
    \ifthenelse{\isodd{\thepage}}{
      \AtPageLowerLeft{
        \put(\LenToUnit{\dimexpr\paperwidth-222pt},0){
          \color{reallylightgray}\rule{222pt}{297mm}
        }
      }
    }
    {
      \AtPageLowerLeft{
        \color{reallylightgray}\rule{222pt}{297mm}
      }
    }
  }
}

%%%%%%%%%%%%%%%%%%%%%%%%
%  Modify Links Color  %
%%%%%%%%%%%%%%%%%%%%%%%%

\hypersetup{
  % Enable highlighting links.
  colorlinks,
  % Change the color of links to blue.
  linkcolor={black},
  % Change the color of citations to black.
  citecolor={black},
  % Change the color of url's to blue with some black.
  urlcolor=blue
}

%%%%%%%%%%%%%%%%%%
% Fix WrapFigure %
%%%%%%%%%%%%%%%%%%

\newcommand{\wrapfill}{\par\ifnum\value{WF@wrappedlines}>0
    \parskip=0pt
    \addtocounter{WF@wrappedlines}{-1}%
    \null\vspace{\arabic{WF@wrappedlines}\baselineskip}%
    \WFclear
\fi}

%%%%%%%%%%%%%%%%%
% Multi Columns %
%%%%%%%%%%%%%%%%%

\let\multicolmulticols\multicols
\let\endmulticolmulticols\endmulticols

\RenewDocumentEnvironment{multicols}{mO{}}
{%
  \ifnum#1=1
    #2%
  \else % More than 1 column
    \multicolmulticols{#1}[#2]
  \fi
}
{%
  \ifnum#1=1
\else % More than 1 column
  \endmulticolmulticols
\fi
}

\newlength{\thickarrayrulewidth}
\setlength{\thickarrayrulewidth}{5\arrayrulewidth}

%%%%%%%%%%%%%%%%%%%%
%  Import Figures  %
%%%%%%%%%%%%%%%%%%%%

\usepackage{import}
\pdfminorversion=7

% EXAMPLE:
% 1. \incfig{limit-graph}
% 2. \incfig[0.4]{limit-graph}
% Parameters:
% 1. The figure name. It should be located in figures/NAME.tex_pdf.
% 2. (Optional) The width of the figure. Example: 0.5, 0.35.
\newcommand\incfig[2][1]{%
  \def\svgwidth{#1\columnwidth}
  \import{./figures/}{#2.pdf_tex}
}

\begingroup\expandafter\expandafter\expandafter\endgroup
\expandafter\ifx\csname pdfsuppresswarningpagegroup\endcsname\relax
\else
  \pdfsuppresswarningpagegroup=1\relax
\fi

%%%%%%%%%%%%%
%  Correct  %
%%%%%%%%%%%%%

% EXAMPLE:
% 1. \correct{INCORRECT}{CORRECT}
% Parameters:
% 1. The incorrect statement.
% 2. The correct statement.
\definecolor{correct}{HTML}{009900}
\newcommand\correct[2]{{\color{red}{#1 }}\ensuremath{\to}{\color{correct}{ #2}}}



\newcommand{\R}{\mathbb{R}}
\newcommand{\Z}{\mathbb{Z}}
\newcommand{\E}{\mathbb{E}}
\newcommand{\B}{\ensuremath{\mathcal{B}}}
\newcommand{\X}{\ensuremath{\mathcal{X}}}
\newcommand{\Y}{\ensuremath{\mathcal{Y}}}
\newcommand{\mA}{\ensuremath{\mathbf{A}}}
\newcommand{\mB}{\ensuremath{\mathbf{B}}}
\newcommand{\mC}{\ensuremath{\mathbf{C}}}
\newcommand{\mD}{\ensuremath{\mathbf{D}}}
\newcommand{\mX}{\ensuremath{\mathbf{X}}}
\newcommand{\mY}{\ensuremath{\mathbf{Y}}}
\newcommand{\mx}{\ensuremath{\mathbf{x}}}
\newcommand{\my}{\ensuremath{\mathbf{y}}}
\newcommand{\mI}{\ensuremath{\mathbf{I}}}
\newcommand{\mi}{\ensuremath{\mathbf{\iota}}}
\newcommand{\mmu}{\ensuremath{\mathbf{\mu}}}
\newcommand{\mc}{\ensuremath{\mathbf{c}}}
\newcommand{\mSigma}{\ensuremath{\mathbf{\Sigma}}}
\newcommand{\mzero}{\ensuremath{\mathbf{0}}}
\newcommand{\independent}{\perp\!\!\!\!\perp} 
\setlength{\parindent}{0pt}
%%%%%%%%%%%%%%%%%%%%%%%%%%%%%%%%%%%%%%%%%%%%%%%%%%%%%%%%%%%%%%%%%%%%%%%%%%%%%%%
%                                 Environments                                %
%%%%%%%%%%%%%%%%%%%%%%%%%%%%%%%%%%%%%%%%%%%%%%%%%%%%%%%%%%%%%%%%%%%%%%%%%%%%%%%

\usepackage{varwidth}
\usepackage{thmtools}
\usepackage[most,many,breakable]{tcolorbox}

\tcbuselibrary{theorems,skins,hooks}
\usetikzlibrary{arrows,calc,shadows.blur}

%%%%%%%%%%%%%%%%%%%
%  Define Colors  %
%%%%%%%%%%%%%%%%%%%

\definecolor{myblue}{RGB}{45, 111, 177}
\definecolor{mygreen}{RGB}{56, 140, 70}
\definecolor{myred}{RGB}{199, 68, 64}
\definecolor{mypurple}{RGB}{197, 92, 212}

\definecolor{definition}{HTML}{228b22}
\definecolor{theorem}{HTML}{00007B}
\definecolor{example}{HTML}{2A7F7F}
\definecolor{definition}{HTML}{228b22}
\definecolor{prop}{HTML}{191971}
\definecolor{lemma}{HTML}{983b0f}
\definecolor{exercise}{HTML}{88D6D1}

\colorlet{definition}{mygreen!85!black}
\colorlet{claim}{mygreen!85!black}
\colorlet{corollary}{mypurple!85!black}
\colorlet{proof}{theorem}

%%%%%%%%%%%%%%%%%%%%%%
%  Helpful Commands  %
%%%%%%%%%%%%%%%%%%%%%%

% EXAMPLE:
% 1. \createnewtheoremstyle{thmdefinitionbox}{}{}
% 2. \createnewtheoremstyle{thmtheorembox}{}{}
% 3. \createnewtheoremstyle{thmproofbox}{qed=\qedsymbol}{
%       rightline=false, topline=false, bottomline=false
%    }
% Parameters:
% 1. Theorem name.
% 2. Any extra parameters to pass directly to declaretheoremstyle.
% 3. Any extra parameters to pass directly to mdframed.
\newcommand\createnewtheoremstyle[3]{
  \declaretheoremstyle[
  headfont=\bfseries\sffamily, bodyfont=\normalfont, #2,
  mdframed={
    #3,
  },
  ]{#1}
}

% EXAMPLE:
% 1. \createnewcoloredtheoremstyle{thmdefinitionbox}{definition}{}{}
% 2. \createnewcoloredtheoremstyle{thmexamplebox}{example}{}{
%       rightline=true, leftline=true, topline=true, bottomline=true
%     }
% 3. \createnewcoloredtheoremstyle{thmproofbox}{proof}{qed=\qedsymbol}{backgroundcolor=white}
% Parameters:
% 1. Theorem name.
% 2. Color of theorem.
% 3. Any extra parameters to pass directly to declaretheoremstyle.
% 4. Any extra parameters to pass directly to mdframed.
\newcommand\createnewcoloredtheoremstyle[4]{
  \declaretheoremstyle[
  headfont=\bfseries\sffamily\color{#2}, bodyfont=\normalfont, #3,
  mdframed={
    linewidth=2pt,
    rightline=false, leftline=true, topline=false, bottomline=false,
    linecolor=#2, backgroundcolor=#2!5, #4,
  },
  ]{#1}
}

%%%%%%%%%%%%%%%%%%%%%%%%%%%%%%%%%%%
%  Create the Environment Styles  %
%%%%%%%%%%%%%%%%%%%%%%%%%%%%%%%%%%%

\makeatletter
\@ifclasswith\class{nocolor}{
  % Environments without color.

  \createnewtheoremstyle{thmdefinitionbox}{}{}
  \createnewtheoremstyle{thmtheorembox}{}{}
  \createnewtheoremstyle{thmexamplebox}{}{}
  \createnewtheoremstyle{thmclaimbox}{}{}
  \createnewtheoremstyle{thmcorollarybox}{}{}
  \createnewtheoremstyle{thmpropbox}{}{}
  \createnewtheoremstyle{thmlemmabox}{}{}
  \createnewtheoremstyle{thmexercisebox}{}{}
  \createnewtheoremstyle{thmdefinitionbox}{}{}
  \createnewtheoremstyle{thmquestionbox}{}{}
  \createnewtheoremstyle{thmsolutionbox}{}{}

  \createnewtheoremstyle{thmproofbox}{qed=\qedsymbol}{}
  \createnewtheoremstyle{thmexplanationbox}{}{}
}{
  % Environments with color.

  \createnewcoloredtheoremstyle{thmdefinitionbox}{definition}{}{}
  \createnewcoloredtheoremstyle{thmtheorembox}{theorem}{}{}
  \createnewcoloredtheoremstyle{thmexamplebox}{example}{}{
    rightline=true, leftline=true, topline=true, bottomline=true
  }
  \createnewcoloredtheoremstyle{thmclaimbox}{claim}{}{}
  \createnewcoloredtheoremstyle{thmcorollarybox}{corollary}{}{}
  \createnewcoloredtheoremstyle{thmpropbox}{prop}{}{}
  \createnewcoloredtheoremstyle{thmlemmabox}{lemma}{}{}
  \createnewcoloredtheoremstyle{thmexercisebox}{exercise}{}{}

  \createnewcoloredtheoremstyle{thmproofbox}{proof}{qed=\qedsymbol}{backgroundcolor=white}
  \createnewcoloredtheoremstyle{thmexplanationbox}{example}{qed=\qedsymbol}{backgroundcolor=white}
}
\makeatother

%%%%%%%%%%%%%%%%%%%%%%%%%%%%%
%  Create the Environments  %
%%%%%%%%%%%%%%%%%%%%%%%%%%%%%

\declaretheorem[numberwithin=section, style=thmtheorembox,     name=Theorem]{theorem}
\declaretheorem[numbered=no,          style=thmexamplebox,     name=Example]{example}
\declaretheorem[numberwithin=section, style=thmclaimbox,       name=Claim]{claim}
\declaretheorem[numberwithin=section, style=thmcorollarybox,   name=Corollary]{corollary}
\declaretheorem[numberwithin=section, style=thmpropbox,        name=Proposition]{prop}
\declaretheorem[numberwithin=section, style=thmlemmabox,       name=Lemma]{lemma}
\declaretheorem[numberwithin=section, style=thmexercisebox,    name=Exercise]{exercise}
\declaretheorem[numbered=no,          style=thmproofbox,       name=Proof]{replacementproof}
\declaretheorem[numbered=no,          style=thmexplanationbox, name=Explanation]{expl}

\makeatletter
\@ifclasswith\class{nocolor}{
  % Environments without color.

  \newtheorem*{note}{Note}

  \declaretheorem[numberwithin=section, style=thmdefinitionbox, name=Definition]{definition}
  \declaretheorem[numberwithin=section, style=thmquestionbox,   name=Question]{question}
  \declaretheorem[numberwithin=section, style=thmsolutionbox,   name=Solution]{solution}
}{
  % Environments with color.

  \newtcbtheorem[number within=section]{Definition}{Definition}{
    enhanced,
    before skip=2mm,
    after skip=2mm,
    colback=red!5,
    colframe=red!80!black,
    colbacktitle=red!75!black,
    boxrule=0.5mm,
    attach boxed title to top left={
      xshift=1cm,
      yshift*=1mm-\tcboxedtitleheight
    },
    varwidth boxed title*=-3cm,
    boxed title style={
      interior engine=empty,
      frame code={
        \path[fill=tcbcolback]
        ([yshift=-1mm,xshift=-1mm]frame.north west)
        arc[start angle=0,end angle=180,radius=1mm]
        ([yshift=-1mm,xshift=1mm]frame.north east)
        arc[start angle=180,end angle=0,radius=1mm];
        \path[left color=tcbcolback!60!black,right color=tcbcolback!60!black,
        middle color=tcbcolback!80!black]
        ([xshift=-2mm]frame.north west) -- ([xshift=2mm]frame.north east)
        [rounded corners=1mm]-- ([xshift=1mm,yshift=-1mm]frame.north east)
        -- (frame.south east) -- (frame.south west)
        -- ([xshift=-1mm,yshift=-1mm]frame.north west)
        [sharp corners]-- cycle;
      },
    },
    fonttitle=\bfseries,
    title={#2},
    #1
  }{def}

  \NewDocumentEnvironment{definition}{O{}O{}}
    {\begin{Definition}{#1}{#2}}{\end{Definition}}

  \newtcolorbox{note}[1][]{%
    enhanced jigsaw,
    colback=gray!20!white,%
    colframe=gray!80!black,
    size=small,
    boxrule=1pt,
    title=\textbf{Note:-},
    halign title=flush center,
    coltitle=black,
    breakable,
    drop shadow=black!50!white,
    attach boxed title to top left={xshift=1cm,yshift=-\tcboxedtitleheight/2,yshifttext=-\tcboxedtitleheight/2},
    minipage boxed title=1.5cm,
    boxed title style={%
      colback=white,
      size=fbox,
      boxrule=1pt,
      boxsep=2pt,
      underlay={%
        \coordinate (dotA) at ($(interior.west) + (-0.5pt,0)$);
        \coordinate (dotB) at ($(interior.east) + (0.5pt,0)$);
        \begin{scope}
          \clip (interior.north west) rectangle ([xshift=3ex]interior.east);
          \filldraw [white, blur shadow={shadow opacity=60, shadow yshift=-.75ex}, rounded corners=2pt] (interior.north west) rectangle (interior.south east);
        \end{scope}
        \begin{scope}[gray!80!black]
          \fill (dotA) circle (2pt);
          \fill (dotB) circle (2pt);
        \end{scope}
      },
    },
    #1,
  }

  \newtcbtheorem{Question}{Question}{enhanced,
    breakable,
    colback=white,
    colframe=myblue!80!black,
    attach boxed title to top left={yshift*=-\tcboxedtitleheight},
    fonttitle=\bfseries,
    title=\textbf{Question:-},
    boxed title size=title,
    boxed title style={%
      sharp corners,
      rounded corners=northwest,
      colback=tcbcolframe,
      boxrule=0pt,
    },
    underlay boxed title={%
      \path[fill=tcbcolframe] (title.south west)--(title.south east)
      to[out=0, in=180] ([xshift=5mm]title.east)--
      (title.center-|frame.east)
      [rounded corners=\kvtcb@arc] |-
      (frame.north) -| cycle;
    },
    #1
  }{def}

  \NewDocumentEnvironment{question}{O{}O{}}
  {\begin{Question}{#1}{#2}}{\end{Question}}

  \newtcolorbox{Solution}{enhanced,
    breakable,
    colback=white,
    colframe=mygreen!80!black,
    attach boxed title to top left={yshift*=-\tcboxedtitleheight},
    title=\textbf{Solution:-},
    boxed title size=title,
    boxed title style={%
      sharp corners,
      rounded corners=northwest,
      colback=tcbcolframe,
      boxrule=0pt,
    },
    underlay boxed title={%
      \path[fill=tcbcolframe] (title.south west)--(title.south east)
      to[out=0, in=180] ([xshift=5mm]title.east)--
      (title.center-|frame.east)
      [rounded corners=\kvtcb@arc] |-
      (frame.north) -| cycle;
    },
  }

  \NewDocumentEnvironment{solution}{O{}O{}}
  {\vspace{-10pt}\begin{Solution}{#1}{#2}}{\end{Solution}}
}
\makeatother

%%%%%%%%%%%%%%%%%%%%%%%%%%%%
%  Edit Proof Environment  %
%%%%%%%%%%%%%%%%%%%%%%%%%%%%

\renewenvironment{proof}[1][\proofname]{\vspace{-10pt}\begin{replacementproof}}{\end{replacementproof}}
\newenvironment{explanation}[1][\proofname]{\vspace{-10pt}\begin{expl}}{\end{expl}}

\theoremstyle{definition}

\newtheorem*{notation}{Notation}
\newtheorem*{previouslyseen}{As previously seen}
\newtheorem*{problem}{Problem}
\newtheorem*{observe}{Observe}
\newtheorem*{property}{Property}
\newtheorem*{intuition}{Intuition}

\setuptoc{toc}{leveldown}

\begin{document}
\pagenumbering{gobble}
\vspace{-10pt}
\setcounter{chapter}{2}

\chapter{Multinomial Response Models}

\section{The Random Utility Model}
A decision maker $i$ faces a finite choice set $\Omega_J$ of dimension J. The decision maker would obtain a certain level of utility from each alternative. The utility that decision maker $i$ derives from alternative $j$ is $U_ij$, which is known to them but \textbf{unknown to the researcher}. Thus alternative $j$ is chosen iff $U_{ij} > U_{ik}$ for all $k \neq j$. \\
The researcher doesn't observe the decision-maker's utility, only some attributes of the alternatives $\mathbf{v_j}$ (sometimes $\mathbf{v_{ij}}$) $\forall j$, and some attributes of the decision maker $\mathbf{x_i}$. The function is denoted $V_{ij} = V(\mathbf{v_j},\mathbf{x_i}, \mathbf{\theta})$ where $\mathbf{\theta}$ is a vector of unknown parameters. Since there are aspects of utility that are not observed, $V_{ij} \not = U_{ij}$, and we decompose utility as $U_{ij} = V_{ij} + \epsilon_{ij}$, where $\epsilon_{ij}$ captures the unobserved aspects of utility. For example:
\begin{itemize}
    \item $U_{ij} = \mathbf{v_j'\omega} + \epsilon_{ij}$, where $\mathbf{v_j}$ is an $L\times1$ vector of attributes for alternative $j$, and $\mathbf{\omega}$ is an $L\times1$ vector of parameters.
    \item $U_{ij} = \mathbf{x_i ' \beta} + \alpha_j + \epsilon_{ij}$, where $\mathbf{x_i}$ is an $K\times1$ vector of attributes for decision maker $i$, $\mathbf{\beta}$ is a $K\times1$ vector of parameters and $\alpha_j$ is a scalar alternative-specific constant (capturing average utility of alternative $j$).
    \item $U_{ij} = \frac{\mathbf{v_j'\omega}}{\mathbf{x_i'}} + \alpha_j + \epsilon_{ij}$ denoting interactions between attributes and characteristics.
\end{itemize}
The researcher doesn't know $\epsilon_{ij}$ and therefore treats it as random, with the joint density of random vector $\mathbf{\epsilon_i} = (\epsilon_{i1}, \dots, \epsilon_{iJ})$ denoted $f(\mathbf{\epsilon_i})$. 
\begin{note}
    How should we interpret $\epsilon_{ij}$? It is not defined for a choice situation per se, rather it is defined relative to a researcher's representation of that choice situation. What does it mean for this to have a distribution?\\
    Consider a population of people who face the same observed utility as person $i$. Among these people, the values of the unobserved factors differ. The density $f(\epsilon_{ij})$ is the distribution of the unobserved portion of utility within the population of people who face the same observed utility as person $i$. It can also just be thought of as a subjective probability imposed by the researcher.
\end{note}
With this density, we can make probabilistic statements about choices. The probability that decision maker $i$ chooses alternative $j$ is:
\begin{align*}
    P_{ij} &= P(U_{ij} > U_{ik} \forall k \neq j) \\
    &= P(V_{ij} + \epsilon_{ij} > V_{ik} + \epsilon_{ik} \forall k \neq j) \\
    &= P(\epsilon_{ij} - \epsilon_{ik} < V_{ik} - V_{ij} \forall k \neq j) 
\end{align*}
This probability is the CDF of $\epsilon_{ij} - \epsilon_{ik}$ evaluated at $V_{ik} - V_{ij}$. Using the density of $\epsilon_{ij}$, we can write this as:
\begin{align*}
    P_{ij} &= P(\epsilon_{ij} - \epsilon_{ik} < V_{ik} - V_{ij} \forall k \neq j) \\
    &= \int_\epsilon I(\epsilon_{ij} - \epsilon_{ik} < V_{ik} - V_{ij} \forall k \neq j) f(\epsilon_{ij}) d\epsilon_{i} 
\end{align*}
where $I(\cdot)$ is the indicator function. This is a multidimensional integral over the density of the unobserved portion of utility.
\subsection{Identification}
\textbf{"Only differences in utility matter"}\\
The absolute level of utility is irrelevant to both the decision maker and the researcher. If a constant is added to all utilities, the ordering of alternatives is unchanged. We can see this from the decomposition of the choice probability:
\[
    P_{ij} = P(\epsilon_{ij} - \epsilon_{ik} < V_{ik} - V_{ij} \forall k \neq j)
\]
which only depends on differences. In general this means that the only parameters that can be estimated (identified) are those that capture differences across alternatives. This general statement takes several forms:
\begin{itemize}
    \item \textbf{Alternative-specific constants}: $\alpha_j$ captures the average utility of alternative $j$. When they are included $\epsilon_{ij}$ has zero mean by construction. However, since only differences in utility matter, only differences in $\alpha_j$ matter, not the absolute level. To reflect this, the researcher must set the level of one constant. \textbf{With $J$ alternatives, at most $J-1$ constants are identified, with one of the constants normalised to zero}\footnote{The researcher could normalize to some value other than zero, of course;
    however, there would be no point in doing so, since normalizing to
    zero is easier (the constant is simply left out of the model) and has the
    same effect.}. It's irrelevant which constant is normalised to zero, just be careful of the interpretation of the other constants.
    \item \textbf{Number of independent error terms}. The choice probabilities take the form of a $J$ dimensional integral over the density of the $J$ error terms. However, recognising that only differences in utility matter, we can reduce the dimension by defining $\tilde \epsilon_{ijk} = \epsilon_{ij} - \epsilon_{ik}$, giving:
    \[
        P_{ij} = \int I(\tilde \epsilon_{ijk} < V_{ik} - V_{ij} \forall k \neq j) g(\tilde \epsilon_{ijk}) d\tilde \epsilon_{ijk}
    \]
    This is a $J-1$ dimensional integral!\footnote{Since $\mathbf{\epsilon_i}$ has more elements than $\mathbf{\tilde \epsilon_{ij}}$, g is consistent with an infinite number of different f's. One dimension of f is not identified, and must be normalised.}
\end{itemize}
\textbf{"The overall scale of utility is irrelevant "}\\
Just as adding a constant to the utility of all alternatives doesn't change the choice, neither does multiplying each alternative's utility by a constant. To account for this, the researcher must normalise the scale of utility.\\
The standard way to do this is to normalise the variance of the error terms. When utility is multiplied by $\kappa$, the variance of $\epsilon_{ij}$ changes by $\kappa^2$: $Var(\kappa \epsilon_{ij}) = \kappa^2 Var(\epsilon_{ij})$. Therefore \textbf{normalising the variance of the error term is equivalent to normalising the scale of utility}.\\
When errors are assumed i.i.d., normalising is straightforward. We set the error variance to some number, usually something convenient. Since all errors have the same variance by assumption, normalising the variance of any of them sets the variance for them all. The original model becomes equivalent to $U_{ij}^1 = \mathbf{v_j'\frac{\omega}{\sigma}} + \epsilon_{ij}^1$, with $Var(\epsilon_{ij}^1) = 1$. The new coefficients reflect the impact of observed variables relative to standard deviation of the unobserved factors.
\begin{example}[Probit]
    $Var(\epsilon_{ij}) = \sigma^2$, so we can set variance to 1 by dividing by $\sigma$. 
\end{example}
\begin{example}[Logit]
    $Var(\epsilon_{ij}) = \sigma^2 \pi^2/6$, so we can normalise by setting variance to $\pi^2/6$ (and dividing by $\sigma$).
    Thus 
    \[
        Pr(y_i = j|\mathbf{v}) = \frac{e^{\mathbf{v_{ij}'\omega}/\sigma}}{\sum_{j=1}^J e^{\mathbf{v_{ij}'\omega}/\sigma}}
    \]
\end{example}
\begin{question}
    Show that for a logit parametrisation which sets the normalised variance to 1.6, to convert the probit coefficients to the same scale as logit, they should be multiplied by $\sqrt{1.6}$.
\end{question}
\begin{solution}
\begin{align*}
    U_{ij} &= \mathbf{v_{ij}'\omega} + \epsilon_{ij} \text{ where } Var(\epsilon_{ij}) = 1.6 \\
    \implies U_{ij}^1 &= \mathbf{v_{ij}'}\frac{\omega}{\sqrt{1.6}} + \epsilon_{ij}^1 \text{ where } Var(\epsilon_{ij}^1) = 1
\end{align*}
Since probit sets the variance to 1:
\begin{align*}
    \omega^p &= \frac{\omega^l}{\sqrt{1.6}} \\
    \implies \sqrt{1.6}\omega^p &= \omega^l
\end{align*}
\end{solution}
\section{Multinomial Probit}
We write the conditional probability of choosing alternative j' as 
\[
    Pr(y_i = j'|\mathbf{v_i,\theta}) = \int_{-\infty}^{V_{j'}-V_1} \cdots \int_{-\infty}^{V_{j'}-V_{J}} g(\tilde \epsilon_{1j'} \cdots \tilde \epsilon_{Jj'}, \Xi_{\epsilon J-1}) d\tilde \epsilon_{1j'} \cdots d\tilde \epsilon_{Jj'}
\]
g is a multivariate normal density of dimension $J-1$, $\tilde \epsilon_{ij'} = \epsilon_{ij'} - \epsilon_{ij}$ as before, and $\Xi_{\epsilon J-1}$ is the covariance matrix for the error differences.\\
There is a curse of dimensionality here, there are no closed form expressions for such high dimensional integrals, and for $J>2$  we have to use simulation methods (SMLE).
\begin{lemma}[Cholesky decomposition]
Let $A$ be a $K \times K$ matrix. We say that A possesses a Cholesky decomposition if and only if there exists a lower triangular $K \times K$ matrix $L$ such that its diagonal entries are strictly positive real numbers and
\[
    A = LL'
\]
\end{lemma}
\begin{example}[Trinomial probit]
    \[
        y^*_j = \mathbf{x' \beta} + \epsilon_j \quad j = 1,2,3
    \]
    The deterministic component of utility (for alternative $j$) is given by $V_j = \mathbf{x' \beta}$ and $\epsilon = (\epsilon_1, \epsilon_2, \epsilon_3)' \sim N(0, \Sigma)$. We can redefine using error difference to reduce dimensionality to 2:
    \[
        Pr(y = 1|\mathbf{x}) = \int_{-\infty}^{V_1-V_2} \int_{-\infty}^{V_1-V_3} g(\tilde \epsilon_{21}, \tilde \epsilon_{31}, \rho) d\tilde \epsilon_{21} d\tilde \epsilon_{31}
    \]
    where g is bivariate normal and $\rho$ is the correlation between $\tilde \epsilon_{21}$ and $\tilde \epsilon_{31}$.\\
    We can write the probability above as the product of conditionals as below:
    \begin{align*}
        Pr(y=1|\mathbf{x}) &= Pr(\tilde \epsilon_{21} < V_1 - V_2, \tilde \epsilon_{31} < V_1 - V_3|\mathbf{x})\\
        &= Pr(\tilde \epsilon_{21} < V_1 - V_2|\mathbf{x}) Pr(\tilde \epsilon_{31} < V_1 - V_3 |\mathbf{x}, \tilde \epsilon_{21} < V_1 - V_2) 
    \end{align*}

    The errors $\tilde \epsilon$ are bivariate normal with covariance matrix $\Sigma$. Since $\Sigma$ is positive definite, it can be decomposed as $\Sigma = LL'$, where L is a lower triangular matrix.
    \[ L =
        \begin{bmatrix}
        \ell_{11} & 0 \\
        \ell_{21} & \ell_{22}
        \end{bmatrix} 
    \]
    Using this decomposition, we can construct $\tilde \epsilon$ using \textit{independent} standard normal random variables $e_1$ and $e_2$:
    \[
        \begin{bmatrix}
            \tilde \epsilon_{21} \\
            \tilde \epsilon_{31}
        \end{bmatrix} =
        \begin{bmatrix}
            \ell_{11} & 0 \\
            \ell_{21} & \ell_{22}
        \end{bmatrix}
        \begin{bmatrix}
            e_1 \\
            e_2
        \end{bmatrix}
    \]
    Thus we can represent the probability $Pr(y = 1|\mathbf{x})$ as:
    \[ Pr \left(
        \underbrace{\begin{bmatrix}
            -\infty\\
            - \infty
        \end{bmatrix}}_A
        \leq
        \underbrace{\begin{bmatrix}
            \ell_{11} & 0 \\
            \ell_{21} & \ell_{22}
        \end{bmatrix}}_L
        \underbrace{\begin{bmatrix}
            e_1 \\
            e_2
        \end{bmatrix}}_e
        \leq 
        \underbrace{\begin{bmatrix}
            V_1 - V_2 \\
            V_1 - V_3
        \end{bmatrix}}_B |\mathbf{x}\right)
    \]
    where A and B are the respective lower and upper limits of the integrals given above. The two components of $Pr(y=1|\mathbf{x})$ are:
    \begin{align*}
        Pr(\tilde \epsilon_{21} < V_1 - V_2|\mathbf{x}) &= Pr(\ell_{11}e_1 < V_1 - V_2|\mathbf{x}) \\
        &= Pr(e_1 < \frac{V_1 - V_2}{\ell_{11}}|\mathbf{x}) \\
        Pr( \tilde \epsilon_{31} < V_1 - V_3 |\mathbf{x}, \tilde \epsilon_{21} < V_1 - V_2) &= Pr(\ell_{21}e_1 + \ell_{22}e_2 < V_1 - V_3|\mathbf{x}, e_1 < \frac{V_1 - V_2}{\ell_{11}}) \\
        &= Pr(e_2 < \frac{V_1 - V_3 - \ell_{21}e_1}{\ell_{22}}|\mathbf{x}, e_1 < \frac{V_1 - V_2}{\ell_{11}})
    \end{align*}
    Going forward all probabilities are conditional on $\mathbf{x}$ and I define $b_1 = V_1 - V_2$ and $b_2 = V_1 - V_3$. From before:
    \[
        Pr(y=1) = Pr\left(e_1 < \frac{b_1}{\ell_{11}}\right) Pr\left(e_2 < \frac{b_2 - \ell_{21}e_1}{\ell_{22}}\middle|e_1 < \frac{b_1}{\ell_{11}}\right) 
    \]
    Suppose now we draw a random variable $e^*_1$ from a truncated standard normal density with upper truncation point of $b_1/\ell_{11}$. Then the probability can be rewritten as:
    \[
        Pr(y=1) = Pr\left(e_1 < \frac{b_1}{\ell_{11}}\right) Pr\left(e_2 < \frac{b_2 - \ell_{21}e^*_1}{\ell_{22}}\right) 
    \]
    The GHK simulator is the arithmetic mean of the probabilities given by the above for R random draws of $e^*_1$:
    \[
        Pr^{GHK}(y=1|\mathbf{x}) = \frac{1}{R}\sum_{r=1}^R \Phi \left\{\frac{b_1}{\ell_{11}}\right\} \Phi \left\{\frac{b_2 - \ell_{21}e^*_1}{\ell_{22}}\right\}
    \]
    where $\Phi$ is the standard normal CDF.\\
    \textbf{If $\ell_{21} = 0$ then simulation methods are pointless, the errors are already uncorrelated}
\end{example}
The advantage of this expression is the fact that the e’s are independent normal distributed random variables and hence the probability can be equivalently expressed as a product of independent but conditioned univariate cumulative density functions.
\section{Logit}
In Multinomial Probit, for large $J$ we need to solve a high dimensional integral. Logit is a highly tractable alternative, where the joint density $\epsilon_i = (\epsilon_{i1}, \dots, \epsilon_{iJ})$ is assumed to be i.i.d. extreme value type I. The density for each unobserved component is 
\[
  f(\epsilon_{ij}) = e^{-\epsilon_{ij}}e^{-e^{-\epsilon_{ij}}},
\quad
  F(\epsilon_{ij}) = e^{-e^{-\epsilon_{ij}}}.
\]
As we used before, Var($\epsilon_{ij}$) = $\pi^2/6$ \footnote{The mean of this distribution is not zero, but since only differences in utility matter this is not a problem}.\\
The difference between two extreme value type I random variables is a logistic random variable, then $\tilde \epsilon_{ijk} = \epsilon_{ij} - \epsilon_{ik}$ has density
\[
    F(\tilde \epsilon_{ijk}) = \frac{e^{\tilde \epsilon_{ijk}}}{1 + e^{\tilde \epsilon_{ijk}}}
\]
Using this density, we can write the choice probability as:
\[
    Pr(y_i = j | \mathbf{v_i}) = \frac{e^{V_{ij}}}{\sum_{k=1}^J e^{V_{ik}}}
\]
Clearly all probabilities are positive and sum to 1.
\subsection{Power and limitations of logit}
\begin{itemize}
\item \textbf{Taste variation} - Logit can represent systematic taste variation (that is, taste
variation that relates to observed characteristics of the decision-maker) but not random taste variation (differences in tastes that cannot be linked to observed characteristics).
\item \textbf{Substitution patterns} - The logit model implies proportional substitution across alternatives, given the researcher’s specification of representative utility (see later sections on IIA).
\end{itemize}
\textbf{Independence of irrelevant alternatives (IIA)}\\
For any two alternatives j and k, the ratio of the logit probabilities is:
\begin{align*}
    \frac{P_{ij}}{P_{ik}} &= \frac{e^{V_{ij}}}{\sum_{k=1}^J e^{V_{ik}}} \frac{\sum_{k=1}^J e^{V_{ik}}}{e^{V_{ik}}} \\
    &= \frac{e^{V_{ij}}}{e^{V_{ik}}} \\
    &= e^{V_{ij} - V_{ik}}
\end{align*}
This ratio does not depend on any alternatives other than j and k. That is, the relative odds of choosing j over k is the same no matter what other alternatives are available or what the attributes of the other alternatives are. Since the ratio is independent from alternatives other than j and k, it is said to be independent from “irrelevant” alternatives. 
\begin{example}[IIA and Three Buses]
   A car and blue bus are initially available. Assume $Pr_c = Pr_{bB} = 0.5$, thus $Pr_c/Pr_{bB} = 1$.\\
   Now add a red bus to the choice set $\Omega_J$, and assume the buses are perfect substitutes ($P_{rB} = P_{bB}$). Given IIA, $Pr_c/Pr_{bB} =1$ holds before and after the introduction of the red bus.\\
   $\implies Pr_c/Pr_{bB} = Pr_{rB}/Pr_{bB} = 1$\\
   $\implies Pr_c = Pr_{rB} = Pr_{bB} = 1/3$\\
   We expect the original probability $Pr_c$ not to change when we introduce the red bus, such that $Pr_c = 0.5$ and $Pr_{bB} = 0.25$. We overestimate $Pr_{rB}$ and $Pr_{bB}$, and underestimate $Pr_c$.
\end{example}
\begin{question}
    Let $U_{ij}$ denote the unobserved utility for individual $i\in{1,\dots,n}$ who chooses from large gas cars (lgc), small gas cars (sgc) and small electric cars (sec) $j\in{\text{lgc,sgc,sec}}$. A model which is additive in a linear index and error term is given by:
    \[
        U_{ij} = \mathbf{x_{ij}'\beta} + \epsilon_{ij}
    \]
    where $\epsilon_{ij}$ is an unobserved error term. The error terms are assumed to be i.i.d. extreme value type I. The choice probabilities are:
    \[
        P_{lgc} = 0.66, \quad P_{sgc} = 0.33, \quad P_{sec} = 0.01
    \]
    Suppose a subsidy for electric cars is introduced, and this subsidy raises $P_{sec}$ to 0.1. What are the new values of $P_{lgc}$ and $P_{sgc}$?
\end{question}
\begin{solution}
By the logit model, the probability for each of the gas cars would be predicted to drop by the same percentage:\\
The probability for the large gas car would drop by 10\% of its original value, from 0.66 to 0.594 and the probability for the small gas car would drop by 10\% of its original value, from 0.33 to 0.297.\\
In terms of absolute numbers, the increase in demand for the small electric car (0.09) is predicted to come twice as much from the large gas car (0.06) as from the small gas car (0.03).\\
\textbf{This is precisely the notion of proportional substitution.}
\end{solution}
\subsection{Logit variants}
\textbf{Conditional Logit}\\
This is the normal logit we've seen before, \textit{conditional} on the observed characteristics/attributes we get a choice probability:
\begin{definition}[Conditional Logit]
    \begin{align*}
        U_{ij} &= V_{ij} + \epsilon_{ij} \quad \text{where } \epsilon_{ij} \sim \text{EV1} \\
        V_{ij} &= \mathbf{v_{ij}'\omega} \\
        Pr(y_i &= j|\mathbf{v}) = \frac{e^V_{ij}}{\sum_{k=1}^J e^V_{ik}}\\
        &= \frac{e^{\mathbf{v_{ij}'\omega}}}{\sum_{k=1}^J e^{\mathbf{v_{ik}'\omega}}}
    \end{align*}  
\end{definition}
For example, when there are two alternatives, the conditional logit model is:
\begin{align*}
    Pr(y_i = 1|\mathbf{v}) &= \frac{e^{\mathbf{v_{i1}'\omega}}}{e^{\mathbf{v_{i1}'\omega}} + e^{\mathbf{v_{i2}'\omega}}} = \frac{e^{V_{i1}}}{e^{V_{i1}} + e^{V_{i2}}} \\
    &= \frac{1}{1 + e^{\mathbf{v_{i2}'\omega} - \mathbf{v_{i1}'\omega}}} \\
\end{align*}
The analyst may observe both alternative specific and alternative invariant characteristics, models including both are referred to as \textbf{hybrid models}. This could be done by specifying an additively separable model: $V_{ij} = \mathbf{v_{ij}'\omega} + \mathbf{x_i'\beta}$, or alternatively allowing a bunch of interactions: $V_{ij} = \mathbf{v_{ij}'\omega} + \sum_{k}^K \mathbf{v_{ij}}\times \mathbf{x_{ik}}\delta_{jk}$.\\
\subsection{Identification - Aside}
A consumer facing the choice between J alternatives will choose the alternative in choice set $\Omega_J$ with the highest utility. As before, this implies only differences in utility matter.
\begin{example}
    \[
        U_{i1} = \alpha_1 + \mathbf{x_i'\beta_1}+\epsilon_{i1} \quad U_{i2} = \alpha_2 + \mathbf{x_i'\beta_2}+\epsilon_{i2} 
    \]
    With J alternatives, only J-1 ASC's are identified (as earlier). Similarly only $\beta_{12}= \beta_1 - \beta_2$ is identified. Normalisations: set $\alpha_1 = 0$ and $\beta_1 = 0$.
\end{example}
\subsection{Elasticities}
Below we derive own and cross elasticities for the logit model: \[U_{ij} = V_{ij} + \epsilon_{ij}, \quad P_{ij} = \frac{e^{V_{ij}}}{\sum_l e^{V_{il}}}\]
\textbf{Own Partial effects} - We first calculate the partial effect of a change in attribute $k$ of alternative $j$ for individual $i$ on the probability of choosing alternative $j$:
\begin{align*}
    \frac{\partial P_{ij}}{\partial v_{ij}^k} &= \frac{\partial}{\partial v_{ij}^k} \left(\frac{e^{V_{ij}}}{\sum_l e^{V_{il}}}\right) \\
    &= \frac{\sum_l e^{V_{il}} \frac{\partial V_{ij}}{\partial v_{ij}^k} e^{V_{ij}} - e^{V_{ij}} \frac{\partial V_{ij}}{\partial v_{ij}^k} e^{V_{ij}}}{\left(\sum_l e^{V_{il}}\right)^2} \\
    &= \frac{\partial V_{ij}}{\partial v_{ij}^k} \left( \frac{e^{V_{ij}}}{\sum_l e^{V_{il}}} - \left(\frac{e^{V_{ij}}}{\sum_l e^{V_{il}}}\right)^2 \right) \\
    &= \frac{\partial V_{ij}}{\partial v_{ij}^k} (P_{ij} - P_{ij}^2)\\
    &= \frac{\partial V_{ij}}{\partial v_{ij}^k} P_{ij} (1 - P_{ij})
\end{align*}
If representative utility is linear in $\mathbf{v_{ij}}$, then $\frac{\partial V_{ij}}{\partial v_{ij}^k} = \omega_k$. Thus: 
\[
   \frac{\partial P_{ij}}{\partial v_{ij}^k} = \omega_k P_{ij} (1 - P_{ij})
\]
\textbf{Own elasticity} - Denote the elasticity of $P_{ij}$ with respect to $v_{ij}^k$ as $E_{jv_{ij}^k}$.
\begin{align*}
    E_{jv_{ij}^k} &= \frac{\partial P_{ij}}{\partial v_{ij}^k} \frac{v_{ij}^k}{P_{ij}} \\
    &= \frac{\partial V_{ij}}{\partial v_{ij}^k} P_{ij}(1 - P_{ij}) \frac{v_{ij}^k}{P_{ij}} \\
    &= \frac{\partial V_{ij}}{\partial v_{ij}^k} v_{ij}^k (1 - P_{ij}) 
\end{align*}
If representative utility is linear in $\mathbf{v_{ij}}$, then:
\[
    E_{jv_{ij}^k} = \omega_k v_{ij}^k (1 - P_{ij})
\]
\textbf{Cross partial effects} - We now calculate the partial effect of a change in attribute $h$ of alternative $j$ for individual $i$ on the probability of choosing alternative $l$:
\begin{align*}
    \frac{\partial P_{ij}}{\partial v_{ih}^k} &= \frac{\partial}{\partial v_{ih}^k} \left(\frac{e^{V_{ij}}}{\sum_l e^{V_{il}}}\right) \\
    &= \frac{- e^{V_{ij}}e^{V_{ih}} \frac{\partial V_{ih}}{\partial v_{ih}^k}}{\left(\sum_l e^{V_{il}}\right)^2} \\
    &= - \frac{\partial V_{ih}}{\partial v_{ih}^k} \left( \frac{e^{V_{ij}}}{\sum_l e^{V_{il}}} \right) \left( \frac{e^{V_{ih}}}{\sum_l e^{V_{il}}} \right) \\
    &= - \frac{\partial V_{ih}}{\partial v_{ih}^k} P_{ij} P_{ih}
\end{align*}
If representative utility is linear in $\mathbf{v_{ij}}$, then:
\[
    \frac{\partial P_{ij}}{\partial v_{ih}^k} = - \omega_k P_{ij} P_{ih}
\]
\textbf{Cross elasticity} - Denote the elasticity of $P_{ij}$ with respect to $v_{ih}^k$ as $E_{jv_{ih}^k}$.
\begin{align*}
    E_{jv_{ih}^k} &= \frac{\partial P_{ij}}{\partial v_{ih}^k} \frac{v_{ih}^k}{P_{ij}} \\
    &= - \frac{\partial V_{ih}}{\partial v_{ih}^k} P_{ij} P_{ih} \frac{v_{ih}^k}{P_{ij}} \\
    &= - \frac{\partial V_{ih}}{\partial v_{ih}^k} v_{ih}^k P_{ih}
\end{align*}
If representative utility is linear in $\mathbf{v_{ij}}$, then:
\[
    E_{jv_{ih}^k} = - \omega_k v_{ih}^k P_{ih}
\]
\textbf{Interpretation of cross elasticity}\\
The cross-elasticity of demand is the same for all choices, i.e. the formula for the cross-elasticity doesn't contain anything relating to the $j^{th}$ alternative.\\
If the $k^{th}$ attribute of $h$ changes, then the effect on substitution probabilities to other alternatives (here alternative $j$) is independent of the share originally with alternative $j$.\\
\begin{itemize}
    \item An improvement in one alternative draws proportionately from all other alternatives.
    \item A worsening in one alternative propels proportionately to all other alternatives, i.e. increase in price, lost demand is redistributed in equal proportions.
\end{itemize}
\begin{example}
    Assume there is a single attribute denoting cost $v^c$ per alternative, $V_{ij} = \omega v^c_{j}$. Then for 3 alternatives we can display $E_{jv^c_{h}}^L$ (the cross elasticity for alternative $j$ - rows - with respect to the cost of alternative $h$ - columns) as:
    \[ E_{jv^c_{h}}^L =
    \begin{bmatrix}
        (1-P_{i1})\omega v^c_1 & -P_{i2}\omega v^c_2 & -P_{i3}\omega v^c_3 \\
        -P_{i1}\omega v^c_1 & (1-P_{i2})\omega v^c_2 & -P_{i3}\omega v^c_3 \\
        -P_{i1}\omega v^c_1 & -P_{i2}\omega v^c_2 & (1-P_{i3})\omega v^c_3
    \end{bmatrix}
    \]
    For $\omega<1$ all own price (cross) elasticities are negative (positive).
\end{example}
\end{document}
