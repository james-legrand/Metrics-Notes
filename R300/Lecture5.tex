\documentclass[DIV=14,titlepage=false]{scrreprt}

%%%%%%%%%%%%%%%%%%%%%%%%%%%%%%%%%%%%%%%%%%%%%%%%%%%%%%%%%%%%%%%%%%%%%%%%%%%%%%%
%                                Basic Packages                               %
%%%%%%%%%%%%%%%%%%%%%%%%%%%%%%%%%%%%%%%%%%%%%%%%%%%%%%%%%%%%%%%%%%%%%%%%%%%%%%%
% Gives us multiple colors.
\usepackage[usenames,dvipsnames,pdftex]{xcolor}
% Lets us style link colors.
\usepackage{hyperref}
% Lets us import images and graphics.
\usepackage{graphicx}
% Lets us use figures in floating environments.
\usepackage{float}
% Lets us create multiple columns.
\usepackage{multicol}
% Gives us better math syntax.
\usepackage{amsmath,amsfonts,mathtools,amsthm,amssymb}
% Lets us strikethrough text.
\usepackage{cancel}
% Lets us edit the caption of a figure.
\usepackage{caption}
% Lets us import pdf directly in our tex code.
\usepackage{pdfpages}
% Lets us do algorithm stuff.
\usepackage[ruled,vlined,linesnumbered]{algorithm2e}
% Gets rid of some errors.
\usepackage{scrhack}
\def\class{article}
\usepackage{geometry}
\geometry{margin=0.9in}
%%%%%%%%%%%%%%%%%%%%%%%%%%%%%%%%%%%%%%%%%%%%%%%%%%%%%%%%%%%%%%%%%%%%%%%%%%%%%%%
%                                Basic Settings                               %
%%%%%%%%%%%%%%%%%%%%%%%%%%%%%%%%%%%%%%%%%%%%%%%%%%%%%%%%%%%%%%%%%%%%%%%%%%%%%%%

%%%%%%%%%%%%%
%  Symbols  %
%%%%%%%%%%%%%

\let\implies\Rightarrow
\let\impliedby\Leftarrow
\let\iff\Leftrightarrow
\let\epsilon\varepsilon

%%%%%%%%%%%%
%  Tables  %
%%%%%%%%%%%%

\setlength{\tabcolsep}{5pt}
\renewcommand\arraystretch{1.5}

%%%%%%%%%%%%%%%%%%%%%%%
%  Center Title Page  %
%%%%%%%%%%%%%%%%%%%%%%%

\usepackage{titling}
\renewcommand\maketitlehooka{\null\mbox{}\vfill}
\renewcommand\maketitlehookd{\vfill\null}

%%%%%%%%%%%%%%%%%%%%%%%%%%%%%%%%%%%%%%%%%%%%%%%%%%%%%%%
%  Create a grey background in the middle of the PDF  %
%%%%%%%%%%%%%%%%%%%%%%%%%%%%%%%%%%%%%%%%%%%%%%%%%%%%%%%

\usepackage{eso-pic}
\newcommand\definegraybackground{
  \definecolor{reallylightgray}{HTML}{FAFAFA}
  \AddToShipoutPicture{
    \ifthenelse{\isodd{\thepage}}{
      \AtPageLowerLeft{
        \put(\LenToUnit{\dimexpr\paperwidth-222pt},0){
          \color{reallylightgray}\rule{222pt}{297mm}
        }
      }
    }
    {
      \AtPageLowerLeft{
        \color{reallylightgray}\rule{222pt}{297mm}
      }
    }
  }
}

%%%%%%%%%%%%%%%%%%%%%%%%
%  Modify Links Color  %
%%%%%%%%%%%%%%%%%%%%%%%%

\hypersetup{
  % Enable highlighting links.
  colorlinks,
  % Change the color of links to blue.
  linkcolor={black},
  % Change the color of citations to black.
  citecolor={black},
  % Change the color of url's to blue with some black.
  urlcolor=blue
}

%%%%%%%%%%%%%%%%%%
% Fix WrapFigure %
%%%%%%%%%%%%%%%%%%

\newcommand{\wrapfill}{\par\ifnum\value{WF@wrappedlines}>0
    \parskip=0pt
    \addtocounter{WF@wrappedlines}{-1}%
    \null\vspace{\arabic{WF@wrappedlines}\baselineskip}%
    \WFclear
\fi}

%%%%%%%%%%%%%%%%%
% Multi Columns %
%%%%%%%%%%%%%%%%%

\let\multicolmulticols\multicols
\let\endmulticolmulticols\endmulticols

\RenewDocumentEnvironment{multicols}{mO{}}
{%
  \ifnum#1=1
    #2%
  \else % More than 1 column
    \multicolmulticols{#1}[#2]
  \fi
}
{%
  \ifnum#1=1
\else % More than 1 column
  \endmulticolmulticols
\fi
}

\newlength{\thickarrayrulewidth}
\setlength{\thickarrayrulewidth}{5\arrayrulewidth}

%%%%%%%%%%%%%%%%%%%%
%  Import Figures  %
%%%%%%%%%%%%%%%%%%%%

\usepackage{import}
\pdfminorversion=7

% EXAMPLE:
% 1. \incfig{limit-graph}
% 2. \incfig[0.4]{limit-graph}
% Parameters:
% 1. The figure name. It should be located in figures/NAME.tex_pdf.
% 2. (Optional) The width of the figure. Example: 0.5, 0.35.
\newcommand\incfig[2][1]{%
  \def\svgwidth{#1\columnwidth}
  \import{./figures/}{#2.pdf_tex}
}

\begingroup\expandafter\expandafter\expandafter\endgroup
\expandafter\ifx\csname pdfsuppresswarningpagegroup\endcsname\relax
\else
  \pdfsuppresswarningpagegroup=1\relax
\fi

%%%%%%%%%%%%%
%  Correct  %
%%%%%%%%%%%%%

% EXAMPLE:
% 1. \correct{INCORRECT}{CORRECT}
% Parameters:
% 1. The incorrect statement.
% 2. The correct statement.
\definecolor{correct}{HTML}{009900}
\newcommand\correct[2]{{\color{red}{#1 }}\ensuremath{\to}{\color{correct}{ #2}}}



\newcommand{\R}{\mathbb{R}}
\newcommand{\Z}{\mathbb{Z}}
\newcommand{\E}{\mathbb{E}}
\newcommand{\B}{\ensuremath{\mathcal{B}}}
\newcommand{\X}{\ensuremath{\mathcal{X}}}
\newcommand{\Y}{\ensuremath{\mathcal{Y}}}
\newcommand{\mA}{\ensuremath{\mathbf{A}}}
\newcommand{\mB}{\ensuremath{\mathbf{B}}}
\newcommand{\mC}{\ensuremath{\mathbf{C}}}
\newcommand{\mD}{\ensuremath{\mathbf{D}}}
\newcommand{\mX}{\ensuremath{\mathbf{X}}}
\newcommand{\mY}{\ensuremath{\mathbf{Y}}}
\newcommand{\mx}{\ensuremath{\mathbf{x}}}
\newcommand{\my}{\ensuremath{\mathbf{y}}}
\newcommand{\mI}{\ensuremath{\mathbf{I}}}
\newcommand{\mi}{\ensuremath{\mathbf{\iota}}}
\newcommand{\mmu}{\ensuremath{\mathbf{\mu}}}
\newcommand{\mc}{\ensuremath{\mathbf{c}}}
\newcommand{\mSigma}{\ensuremath{\mathbf{\Sigma}}}
\newcommand{\mzero}{\ensuremath{\mathbf{0}}}
\newcommand{\independent}{\perp\!\!\!\!\perp} 
\setlength{\parindent}{0pt}
%%%%%%%%%%%%%%%%%%%%%%%%%%%%%%%%%%%%%%%%%%%%%%%%%%%%%%%%%%%%%%%%%%%%%%%%%%%%%%%
%                                 Environments                                %
%%%%%%%%%%%%%%%%%%%%%%%%%%%%%%%%%%%%%%%%%%%%%%%%%%%%%%%%%%%%%%%%%%%%%%%%%%%%%%%

\usepackage{varwidth}
\usepackage{thmtools}
\usepackage[most,many,breakable]{tcolorbox}

\tcbuselibrary{theorems,skins,hooks}
\usetikzlibrary{arrows,calc,shadows.blur}

%%%%%%%%%%%%%%%%%%%
%  Define Colors  %
%%%%%%%%%%%%%%%%%%%

\definecolor{myblue}{RGB}{45, 111, 177}
\definecolor{mygreen}{RGB}{56, 140, 70}
\definecolor{myred}{RGB}{199, 68, 64}
\definecolor{mypurple}{RGB}{197, 92, 212}

\definecolor{definition}{HTML}{228b22}
\definecolor{theorem}{HTML}{00007B}
\definecolor{example}{HTML}{2A7F7F}
\definecolor{definition}{HTML}{228b22}
\definecolor{prop}{HTML}{191971}
\definecolor{lemma}{HTML}{983b0f}
\definecolor{exercise}{HTML}{88D6D1}

\colorlet{definition}{mygreen!85!black}
\colorlet{claim}{mygreen!85!black}
\colorlet{corollary}{mypurple!85!black}
\colorlet{proof}{theorem}

%%%%%%%%%%%%%%%%%%%%%%
%  Helpful Commands  %
%%%%%%%%%%%%%%%%%%%%%%

% EXAMPLE:
% 1. \createnewtheoremstyle{thmdefinitionbox}{}{}
% 2. \createnewtheoremstyle{thmtheorembox}{}{}
% 3. \createnewtheoremstyle{thmproofbox}{qed=\qedsymbol}{
%       rightline=false, topline=false, bottomline=false
%    }
% Parameters:
% 1. Theorem name.
% 2. Any extra parameters to pass directly to declaretheoremstyle.
% 3. Any extra parameters to pass directly to mdframed.
\newcommand\createnewtheoremstyle[3]{
  \declaretheoremstyle[
  headfont=\bfseries\sffamily, bodyfont=\normalfont, #2,
  mdframed={
    #3,
  },
  ]{#1}
}

% EXAMPLE:
% 1. \createnewcoloredtheoremstyle{thmdefinitionbox}{definition}{}{}
% 2. \createnewcoloredtheoremstyle{thmexamplebox}{example}{}{
%       rightline=true, leftline=true, topline=true, bottomline=true
%     }
% 3. \createnewcoloredtheoremstyle{thmproofbox}{proof}{qed=\qedsymbol}{backgroundcolor=white}
% Parameters:
% 1. Theorem name.
% 2. Color of theorem.
% 3. Any extra parameters to pass directly to declaretheoremstyle.
% 4. Any extra parameters to pass directly to mdframed.
\newcommand\createnewcoloredtheoremstyle[4]{
  \declaretheoremstyle[
  headfont=\bfseries\sffamily\color{#2}, bodyfont=\normalfont, #3,
  mdframed={
    linewidth=2pt,
    rightline=false, leftline=true, topline=false, bottomline=false,
    linecolor=#2, backgroundcolor=#2!5, #4,
  },
  ]{#1}
}

%%%%%%%%%%%%%%%%%%%%%%%%%%%%%%%%%%%
%  Create the Environment Styles  %
%%%%%%%%%%%%%%%%%%%%%%%%%%%%%%%%%%%

\makeatletter
\@ifclasswith\class{nocolor}{
  % Environments without color.

  \createnewtheoremstyle{thmdefinitionbox}{}{}
  \createnewtheoremstyle{thmtheorembox}{}{}
  \createnewtheoremstyle{thmexamplebox}{}{}
  \createnewtheoremstyle{thmclaimbox}{}{}
  \createnewtheoremstyle{thmcorollarybox}{}{}
  \createnewtheoremstyle{thmpropbox}{}{}
  \createnewtheoremstyle{thmlemmabox}{}{}
  \createnewtheoremstyle{thmexercisebox}{}{}
  \createnewtheoremstyle{thmdefinitionbox}{}{}
  \createnewtheoremstyle{thmquestionbox}{}{}
  \createnewtheoremstyle{thmsolutionbox}{}{}

  \createnewtheoremstyle{thmproofbox}{qed=\qedsymbol}{}
  \createnewtheoremstyle{thmexplanationbox}{}{}
}{
  % Environments with color.

  \createnewcoloredtheoremstyle{thmdefinitionbox}{definition}{}{}
  \createnewcoloredtheoremstyle{thmtheorembox}{theorem}{}{}
  \createnewcoloredtheoremstyle{thmexamplebox}{example}{}{
    rightline=true, leftline=true, topline=true, bottomline=true
  }
  \createnewcoloredtheoremstyle{thmclaimbox}{claim}{}{}
  \createnewcoloredtheoremstyle{thmcorollarybox}{corollary}{}{}
  \createnewcoloredtheoremstyle{thmpropbox}{prop}{}{}
  \createnewcoloredtheoremstyle{thmlemmabox}{lemma}{}{}
  \createnewcoloredtheoremstyle{thmexercisebox}{exercise}{}{}

  \createnewcoloredtheoremstyle{thmproofbox}{proof}{qed=\qedsymbol}{backgroundcolor=white}
  \createnewcoloredtheoremstyle{thmexplanationbox}{example}{qed=\qedsymbol}{backgroundcolor=white}
}
\makeatother

%%%%%%%%%%%%%%%%%%%%%%%%%%%%%
%  Create the Environments  %
%%%%%%%%%%%%%%%%%%%%%%%%%%%%%

\declaretheorem[numberwithin=section, style=thmtheorembox,     name=Theorem]{theorem}
\declaretheorem[numbered=no,          style=thmexamplebox,     name=Example]{example}
\declaretheorem[numberwithin=section, style=thmclaimbox,       name=Claim]{claim}
\declaretheorem[numberwithin=section, style=thmcorollarybox,   name=Corollary]{corollary}
\declaretheorem[numberwithin=section, style=thmpropbox,        name=Proposition]{prop}
\declaretheorem[numberwithin=section, style=thmlemmabox,       name=Lemma]{lemma}
\declaretheorem[numberwithin=section, style=thmexercisebox,    name=Exercise]{exercise}
\declaretheorem[numbered=no,          style=thmproofbox,       name=Proof]{replacementproof}
\declaretheorem[numbered=no,          style=thmexplanationbox, name=Explanation]{expl}

\makeatletter
\@ifclasswith\class{nocolor}{
  % Environments without color.

  \newtheorem*{note}{Note}

  \declaretheorem[numberwithin=section, style=thmdefinitionbox, name=Definition]{definition}
  \declaretheorem[numberwithin=section, style=thmquestionbox,   name=Question]{question}
  \declaretheorem[numberwithin=section, style=thmsolutionbox,   name=Solution]{solution}
}{
  % Environments with color.

  \newtcbtheorem[number within=section]{Definition}{Definition}{
    enhanced,
    before skip=2mm,
    after skip=2mm,
    colback=red!5,
    colframe=red!80!black,
    colbacktitle=red!75!black,
    boxrule=0.5mm,
    attach boxed title to top left={
      xshift=1cm,
      yshift*=1mm-\tcboxedtitleheight
    },
    varwidth boxed title*=-3cm,
    boxed title style={
      interior engine=empty,
      frame code={
        \path[fill=tcbcolback]
        ([yshift=-1mm,xshift=-1mm]frame.north west)
        arc[start angle=0,end angle=180,radius=1mm]
        ([yshift=-1mm,xshift=1mm]frame.north east)
        arc[start angle=180,end angle=0,radius=1mm];
        \path[left color=tcbcolback!60!black,right color=tcbcolback!60!black,
        middle color=tcbcolback!80!black]
        ([xshift=-2mm]frame.north west) -- ([xshift=2mm]frame.north east)
        [rounded corners=1mm]-- ([xshift=1mm,yshift=-1mm]frame.north east)
        -- (frame.south east) -- (frame.south west)
        -- ([xshift=-1mm,yshift=-1mm]frame.north west)
        [sharp corners]-- cycle;
      },
    },
    fonttitle=\bfseries,
    title={#2},
    #1
  }{def}

  \NewDocumentEnvironment{definition}{O{}O{}}
    {\begin{Definition}{#1}{#2}}{\end{Definition}}

  \newtcolorbox{note}[1][]{%
    enhanced jigsaw,
    colback=gray!20!white,%
    colframe=gray!80!black,
    size=small,
    boxrule=1pt,
    title=\textbf{Note:-},
    halign title=flush center,
    coltitle=black,
    breakable,
    drop shadow=black!50!white,
    attach boxed title to top left={xshift=1cm,yshift=-\tcboxedtitleheight/2,yshifttext=-\tcboxedtitleheight/2},
    minipage boxed title=1.5cm,
    boxed title style={%
      colback=white,
      size=fbox,
      boxrule=1pt,
      boxsep=2pt,
      underlay={%
        \coordinate (dotA) at ($(interior.west) + (-0.5pt,0)$);
        \coordinate (dotB) at ($(interior.east) + (0.5pt,0)$);
        \begin{scope}
          \clip (interior.north west) rectangle ([xshift=3ex]interior.east);
          \filldraw [white, blur shadow={shadow opacity=60, shadow yshift=-.75ex}, rounded corners=2pt] (interior.north west) rectangle (interior.south east);
        \end{scope}
        \begin{scope}[gray!80!black]
          \fill (dotA) circle (2pt);
          \fill (dotB) circle (2pt);
        \end{scope}
      },
    },
    #1,
  }

  \newtcbtheorem{Question}{Question}{enhanced,
    breakable,
    colback=white,
    colframe=myblue!80!black,
    attach boxed title to top left={yshift*=-\tcboxedtitleheight},
    fonttitle=\bfseries,
    title=\textbf{Question:-},
    boxed title size=title,
    boxed title style={%
      sharp corners,
      rounded corners=northwest,
      colback=tcbcolframe,
      boxrule=0pt,
    },
    underlay boxed title={%
      \path[fill=tcbcolframe] (title.south west)--(title.south east)
      to[out=0, in=180] ([xshift=5mm]title.east)--
      (title.center-|frame.east)
      [rounded corners=\kvtcb@arc] |-
      (frame.north) -| cycle;
    },
    #1
  }{def}

  \NewDocumentEnvironment{question}{O{}O{}}
  {\begin{Question}{#1}{#2}}{\end{Question}}

  \newtcolorbox{Solution}{enhanced,
    breakable,
    colback=white,
    colframe=mygreen!80!black,
    attach boxed title to top left={yshift*=-\tcboxedtitleheight},
    title=\textbf{Solution:-},
    boxed title size=title,
    boxed title style={%
      sharp corners,
      rounded corners=northwest,
      colback=tcbcolframe,
      boxrule=0pt,
    },
    underlay boxed title={%
      \path[fill=tcbcolframe] (title.south west)--(title.south east)
      to[out=0, in=180] ([xshift=5mm]title.east)--
      (title.center-|frame.east)
      [rounded corners=\kvtcb@arc] |-
      (frame.north) -| cycle;
    },
  }

  \NewDocumentEnvironment{solution}{O{}O{}}
  {\vspace{-10pt}\begin{Solution}{#1}{#2}}{\end{Solution}}
}
\makeatother

%%%%%%%%%%%%%%%%%%%%%%%%%%%%
%  Edit Proof Environment  %
%%%%%%%%%%%%%%%%%%%%%%%%%%%%

\renewenvironment{proof}[1][\proofname]{\vspace{-10pt}\begin{replacementproof}}{\end{replacementproof}}
\newenvironment{explanation}[1][\proofname]{\vspace{-10pt}\begin{expl}}{\end{expl}}

\theoremstyle{definition}

\newtheorem*{notation}{Notation}
\newtheorem*{previouslyseen}{As previously seen}
\newtheorem*{problem}{Problem}
\newtheorem*{observe}{Observe}
\newtheorem*{property}{Property}
\newtheorem*{intuition}{Intuition}

\setuptoc{toc}{leveldown}

\begin{document}
\vspace{-10pt}
\setcounter{chapter}{4}
\pagenumbering{gobble}
\chapter{Finite sample tests of linear hypotheses.}
\vspace{-10pt}
\section{Linear hypotheses}
The t-test is appropriate when the null hypothesis is a real valued restriction. However, more generally there may be multiple restrictions on the coefficient vector $\boldsymbol{\beta}$. Suppose we have $p>1$ restrictions, we can express a linear hypothesis about $\boldsymbol{\beta}$ in the form $\boldsymbol{R}_{p \times k}\boldsymbol{\beta}_{k \times 1}=\boldsymbol{q}_{p \times 1}$.

\begin{example}[Nerlove's returns to scale]
    Nerlove studied the regression of the total cost of electricity production on demand ($Q_i$) and factor prices (capital, labour and fuel):
    \[ \log TC_i = \beta_1 + \beta_2 \log Q_i + \beta_3 \log p_{C_i} + \beta_4 \log p_{L_i}+ \beta_5 \log p_{F_i} + \epsilon_i \]
    Economic theory suggests that $\beta_2=\frac{1}{r}$ where $r$ is the degree of returns to scale. To test constant returns we can use $H_0: \beta_2=1$, which is trivially linear in components of $\boldsymbol{\beta}$. Alternatively we can write \[R\beta=q\] with $R=(0,1,0,0,0)$ and $q=1$.\\
    Further the total cost must be homogenous of degree 1 with respect to factor prices (doubling cost of all inputs doubles total cost). To test this we can consider $H_0: \beta_3 + \beta_4 +\beta_5 =1$. If we were to reject this it would suggest model misspecification.\\
    To test these hypotheses simultaneously consider:
    \[
        R\beta = q \quad \text{with} \quad R = \begin{pmatrix}
        0 & 1 & 0 & 0 & 0 \\
        0 & 0 & 1 & 1 & 1 \\
        \end{pmatrix}
        \quad \text{and} \quad q = \begin{pmatrix}
        1 \\
        1 \\
        \end{pmatrix}
        \]
\end{example}



To test $H_0$: $\boldsymbol{R\beta}=\boldsymbol{q}$ vs. $H_1$: $\boldsymbol{R\beta}\not = \boldsymbol{q}$ we compute the vector $\boldsymbol{R \hat \beta}=\boldsymbol{q}$ and reject the null if this vector is "too large" depending on the distribution of $\boldsymbol{\hat\beta}$ under $H_0$. 

\begin{definition}[Wald statistic]
    When restrictions are a linear function of coefficients $\boldsymbol{\beta}$, we can write the Wald statistic as \[W = (R\hat \beta-q)'(R\hat V_{\hat\beta}R')^{-1}(R\hat \beta-q) \] i.e. a weighted Euclidean measure of the length of the vector $R\hat \beta-q$.
\end{definition}
\begin{note}
    As the Wald statistic is symmetric in the argument $R\hat \beta-q$ it treats positive and negative alternatives symmetrically. Thus the inherent alternative is always two-sided.\\
    The Wald statistic is not-invariant to a non-linear transformation/reparametrisation of the hypothesis. For example, asking whether $\beta_1 = 1$ is the same as asking whether $\log \beta_1 = 0$; but the Wald statistic for $\beta_1 = 1$ is not the same as the Wald statistic for $\log \beta_1 = 0$. This is because there is in general no neat relationship between the standard errors of $\beta_1$ and $\log \beta_1$, so it needs to be approximated.
\end{note}

Assuming normal regression:
\begin{align*}
\hat\beta|X &\sim N(\beta, \sigma^2(X'X)^{-1})\\
R \hat\beta|X &\sim N(R\beta, \sigma^2R(X'X)^{-1}R')\\
R \hat\beta-q|X &\sim N(R\beta-q, \sigma^2R(X'X)^{-1}R')\\ 
&\overset{H_0}{\sim} N(0, \sigma^2R(X'X)^{-1}R')
\end{align*}
We can thus standardise:
\[
    (\sigma^2R(X'X)^{-1}R')^{-\frac{1}{2}}(R \hat\beta-q)|X \overset{H_0}{\sim} N(0, I_P)
\]
\begin{equation}
    (R \hat\beta-q)'(\sigma^2R(X'X)^{-1}R')^{-1}(R \hat\beta-q)|X \overset{H_0}{\sim} \chi^2(p)  
    \label{eq:WaldStatistic}
\end{equation}
However, the true variance $\sigma^2$ is unknown, we thus replace it with the estimated $\hat \sigma^2$ to obtain the Wald statistic:
\begin{align*}
    W &= (R \hat\beta-q)'(\hat \sigma^2R(X'X)^{-1}R')^{-1}(R \hat\beta-q)\\
    &= \frac{(R \hat\beta-q)'(\sigma^2R(X'X)^{-1}R')^{-1}(R \hat\beta-q)}{\hat\sigma^2/\sigma^2}
\end{align*}



Note that this distribution is not $\chi^2(p)$ since $\hat\sigma^2$ is itself a random variable. We must consider the joint distribution of $\boldsymbol{\hat\sigma^2}$ and $\boldsymbol{\hat\beta}$ to make progress.

\section{The joint distribution of $\boldsymbol{\hat\sigma^2}$ and $\boldsymbol{\hat\beta}$}
Recall the definition of the variance estimator: \[\hat\sigma^2 = \frac{\boldsymbol{\hat\epsilon}'\boldsymbol{\hat\epsilon}}{n-k}\]
To express this in terms of the population $\boldsymbol{\epsilon}$'s examine the following, where we denote the residual maker matrix by $\mathbf{M_X}=\mathbf{I}-\mathbf{X}(\mathbf{X}'\mathbf{X})^{-1}\mathbf{X}'$:
\begin{align*}
    (n-k)\hat\sigma^2&=\boldsymbol{\hat\epsilon}'\boldsymbol{\hat\epsilon}\\
    &= (\mathbf{M_X}\mathbf{y})'\mathbf{M_X}\mathbf{y} \\
    &= (\mathbf{M_X}(\mathbf{X}\boldsymbol{\beta}+\boldsymbol{\epsilon}))'\mathbf{M_X}(\mathbf{X}\boldsymbol{\beta}+\boldsymbol{\epsilon})\\
    &= \boldsymbol{\epsilon}'\mathbf{M'_X}\mathbf{M_X}\boldsymbol{\epsilon} \hspace{20pt} \text{(since $\mathbf{M_XX}=\mathbf{0}$)}\\
    &= \boldsymbol{\epsilon}'\mathbf{M_X}\boldsymbol{\epsilon} \hspace{39pt} \text{(since $\mathbf{M'_XM_X}=\mathbf{M_XM_X}=\mathbf{M_X}$)}
\end{align*}
 Since $\mathbf{M_X}$ is symmetric, it is positive definite when all eigenvalues are positive. Since it is also idempotent, $\mathbf{M^2_X}=\mathbf{M_X}$, all eigenvalues are either zero or one, meaning $\mathbf{M_X}$ is positive semi-definite.\footnote[1]{Alternatively since \( \mathbf{M_X^2} = \mathbf{M_X} \) and \( \mathbf{M_X}' = \mathbf{M_X} \), note that \( \mathbf{v}'\mathbf{M_Xv} = \mathbf{v}'\mathbf{M_X^2v} = \mathbf{v}'\mathbf{M_X}'\mathbf{M_Xv} = (\mathbf{v}'\mathbf{M_X})'(\mathbf{M_Xv}) = \|\mathbf{M_Xv}\|^2 \) for all \( \mathbf{v} \in \mathbb{R}^n \).
}

\begin{lemma}[Spectral decomposition]
    For every $n \times n$ real symmetric matrix, the eigenvalues are real and the eigenvectors can be chosen real and orthonormal. Thus a real symmetric matrix $\mathbf {A}$ can be decomposed as
    \[ \mathbf {A} =\mathbf {Q} \mathbf {\Lambda } \mathbf {Q'}\]
    where $\mathbf {Q}$ is an orthogonal matrix whose columns are the real, orthonormal eigenvectors of $\mathbf {A}$, and $\mathbf {\Lambda}$ is a diagonal matrix whose entries are the eigenvalues of $\mathbf {A}$. 
\end{lemma}

The spectral decomposition of $\mathbf{M_X}$ is $\mathbf{M_X}=\mathbf{H}\mathbf{\Lambda}\mathbf{H'}$ where $\mathbf{HH'}=\mathbf{I_n}$ and $\mathbf{\Lambda}$ is diagonal with the eigenvalues of $\mathbf{M_X}$ along the diagonal. Since $\mathbf {M_X}$ is idempotent with rank $n-k$, it has $n-k$ eigenvalues equalling 1 and $k$ eigenvalues equalling 0, so: 
\[\mathbf{\Lambda} = \begin{bmatrix}
    \mathbf{I}_{n-k} & \mathbf{0} \\
    \mathbf{0} & \mathbf{0}_k
    \end{bmatrix} \]
    
In the normal regression $\boldsymbol{\epsilon} \sim N(0,\mathbf{I_n} \sigma^2)$, we want to find the distribution of $\mathbf{H'}\boldsymbol{\epsilon}$. A linear combination of normals is also normal, meaning $\mathbf{H'}\boldsymbol{\epsilon}$ is normal with mean $\E[\mathbf{H'}\boldsymbol{\epsilon}]=\mathbf{H'}\E[\boldsymbol{\epsilon}]=0$ and variance Var$(\mathbf{H'}\boldsymbol{e})=\mathbf{H'}\mathbf{I_n} \sigma^2 \mathbf{H} = \sigma^2 \mathbf{H'}\mathbf{H} = \mathbf{I_n} \sigma^2$. Thus $\mathbf{H'}\boldsymbol{\epsilon} \sim N(0,\mathbf{I_n} \sigma^2)$.

Let $\mathbf{u}=\mathbf{H'}\boldsymbol{\epsilon}$, and partition $\underset{n \times 1}{\mathbf{u}}=\begin{bmatrix}
    \underset{(n-k) \times 1}{\mathbf{u_1}}\\
    \underset{k \times 1}{\mathbf{u_2}}
\end{bmatrix}$ where $\mathbf{u_1} \sim N(0,\mathbf{I_n} \sigma^2)$, then we have

\begin{align*}
    (n-k)\hat\sigma^2 &= \boldsymbol{\epsilon}'\mathbf{M_X} \boldsymbol{\epsilon}\\
    &= \boldsymbol{\epsilon}' \mathbf{H}\mathbf{\Lambda} \mathbf{H'} \boldsymbol{\epsilon}\\
    &= \mathbf{u}' \begin{bmatrix}
        \mathbf{I}_{n-k} & \mathbf{0} \\
        \mathbf{0} & \mathbf{0}_k
        \end{bmatrix} \mathbf{u}\\
    &= [\mathbf{u'_1} \hspace{5pt} \mathbf{u'_2}] \begin{bmatrix}
        \mathbf{I}_{n-k} & \mathbf{0} \\
        \mathbf{0} & \mathbf{0}_k
        \end{bmatrix} \begin{bmatrix}
            \mathbf{u_1}\\
            \mathbf{u_2}
        \end{bmatrix}\\
    &=\mathbf{u'_1} \mathbf{u_1}
\end{align*}
where $\mathbf{u'_1} \mathbf{u_1}$ is the sum of $n-k$ squared standard normals , thus it is distributed $\chi^2_{n-k}$. Since $\boldsymbol{\epsilon}$ is independent of $\boldsymbol{\hat\beta}$ it follows that $\hat\sigma^2$ is independent of $\boldsymbol{\hat\beta}$ as well.
\begin{theorem} 
In normal regression, \[\frac{(n-k)\hat\sigma^2}{\sigma^2}\sim\chi^2_{n-k}\] and is independent of $\boldsymbol{\hat\beta}$.
\label{thm:distributionOfVariance}
\end{theorem}
\begin{corollary}
    In normal regression satisfying GM1-3, the normalised Wald statistic $\frac{W}{p}$, is distributed as $F(p,n-k)$ under the null.
\end{corollary}
\begin{proof}
    \[
\frac{W}{p} = \frac{(R \hat\beta-q)'(\sigma^2R(X'X)^{-1}R')^{-1}(R \hat\beta-q)/p}{\hat\sigma^2/\sigma^2} \sim \frac{\chi^2(p)/p}{\chi^2(n - k)/(n - k)} \sim F(p, n - k).
\]
Where we have used \ref{eq:WaldStatistic} in the numerator, and Theorem \ref{thm:distributionOfVariance} in the denominator.
\end{proof}
Consider a special case of testing a single restriction, that the $j$-th coefficient is zero. Then $R\hat \beta_j -q=\beta_j$:
\begin{align*}
\hat \beta_j|X &\overset{H_0}{\sim}N(0,\sigma^2(X'X)^{-1}_{ij})\\
\frac{\hat \beta_j}{\sqrt{\sigma^2(X'X)^{-1}_{jj}}}|X&\overset{H_0}{\sim}N(0,1)
\end{align*}
As before $\sigma^2$ is unknown, we can substitute in $\hat\sigma^2$, but the distribution will change:
\begin{align*}
t&=\frac{\hat\beta_j}{\sqrt{\hat\sigma^2(X'X)^{-1}_{jj}}}\\
 &= \frac{\hat\beta_j/{\sqrt{\sigma^2(X'X)^{-1}_{jj}}}}{\sqrt{\frac{(n-k)\hat\sigma^2}{\sigma^2}/(n-k)}}\\
 t|X &\overset{H_0}{\sim} \frac{N(0,1)}{\sqrt{\chi^2(n - k)/(n - k)}}\\
 &\overset{H_0}{\sim} t(n-k)
\end{align*}
Where we are using the fact that the numerator and denominator are independent conditional on X. Note that the square of the $t$-statistic equals the F-statistic for testing the single restriction.
\begin{align*}
t^2(n-k) &= \left(\frac{N(0,1)}{\sqrt{\chi^2(n - k)/(n - k)}}\right)^2\\
&= \frac{\chi^2(1)/1}{\chi^2(n - k)/(n - k)}\\
&= F(1,n-k)
\end{align*}
It is preferable to use the t-statistic since we can test one-sided alternatives, by squaring it we kill the sign of $\hat\beta_j$, making it impossible to differentiate between left and right sided alternatives.
\section{The familiar form of the F-statistic}
Consider the following test:\[ H_0: R\beta = q \text{ vs. } H_1: R\beta \not = q.\]

\begin{prop}
    The normalised Wald statistic is equivalent to the following formula for the F-statistic when testing linear restrictions:
    \[ F=\frac{W}{p} = \frac{(RSS_r-RSS_u)/p}{RSS_u/(n-k)} \]
\end{prop}
\begin{proof}
Let us impose the null hypothesis $R\beta = q$ when minimising the sum of squared residuals, denote the solution as the restricted least squares estimator $\tilde{\beta}$:
\[ 
    \min_{\beta} (Y - X\beta)'(Y - X\beta) \quad \text{s.t.} \quad R\beta = q
\]
\[
    \mathcal{L}(\beta) = (Y - X\beta)'(Y - X\beta) + \lambda'(R\beta - q)    
\]
\begin{align*}
        \frac{\partial \mathcal{L}}{\partial \beta} &= -2X'(Y - X\tilde{\beta}) + R'\lambda = 0\\
         &\implies X'Y - X'X\tilde{\beta} = R'\left(\frac{\lambda}{2}\right)\\
         &\implies (X'X)^{-1}X'Y - (X'X)^{-1}X'X\tilde{\beta} = (X'X)^{-1} R'\left(\frac{\lambda}{2}\right)
\end{align*}
Define the usual (unrestricted) OLS estimate as $\hat\beta = \hat\beta_{OLS}= (X'X)X'Y$ 
\begin{align*}
\implies& \hat\beta - \tilde{\beta} = (X'X)^{-1} R'\left(\frac{\lambda}{2}\right)\\
\implies& \tilde{\beta} = \hat\beta - (X'X)^{-1}R'\left(\frac{\lambda}{2}\right)\\
\implies& R\tilde{\beta} = R\hat\beta - R(X'X)^{-1}R'\left(\frac{\lambda}{2}\right)
\end{align*}
Since \(R\tilde{\beta} = q\):
\begin{align*}
    q &= R\hat\beta - R(X'X)^{-1}R'\left(\frac{\lambda}{2}\right)\\
    R\hat\beta -q &= R(X'X)^{-1}R'\left(\frac{\lambda}{2}\right)\\
    \Rightarrow (R(X'&X)^{-1}R')^{-1}(R\hat{\beta} - q)=\frac{\lambda}{2} 
\end{align*}

Thus,
\[
\tilde{\beta} = \hat{\beta} - (X'X)^{-1}R'\left(R(X'X)^{-1}R'\right)^{-1}(R\hat{\beta} - q)
\]
Now from the corresponding restricted and unrestricted residuals,
\[
\hat{\epsilon} = Y - X\hat{\beta}
\]
\[
\tilde{\epsilon} = Y - X\tilde{\beta} = X\hat{\beta} + \hat{\epsilon} - X\tilde{\beta} = \hat{\epsilon} + X(\hat\beta - \tilde{\beta})
\]
Since \(\hat\epsilon'X = 0\) \footnote[1]{I.e.: Unrestricted OLS residuals uncorrelated with regressors, see lecture 2 for an explanation}
\begin{align*}
\tilde{\epsilon}'\tilde{\epsilon} &= (\hat{\epsilon} + X(\hat\beta - \tilde{\beta}))'(\hat{\epsilon} + X(\hat\beta - \tilde{\beta}))\\
&= \hat{\epsilon}'\hat{\epsilon} + \hat{\epsilon}'X(\hat\beta - \tilde{\beta}) + (\hat\beta - \tilde{\beta})'X'\hat{\epsilon}+(\hat{\beta} - \tilde{\beta})'X'X(\hat{\beta} - \tilde{\beta})\\
&= \hat{\epsilon}'\hat{\epsilon} + (\hat{\beta} - \tilde{\beta})'X'X(\hat{\beta} - \tilde{\beta})
\end{align*}
and substituting \(\hat{\beta} - \tilde{\beta} = (X'X)^{-1}R'\left(R(X'X)^{-1}R'\right)^{-1}(R\hat{\beta} - q)\),


\begin{align*}
\tilde{\epsilon}'\tilde{\epsilon} - \hat{\epsilon}'\hat{\epsilon} &= ((X'X)^{-1}R'\left(R(X'X)^{-1}R'\right)^{-1}(R\hat{\beta} - q))'\bcancel{X'X(X'X)^{-1}}R'\left(R(X'X)^{-1}R'\right)^{-1}(R\hat{\beta} - q)\\
&=(R\hat{\beta} - q)'\left(R(X'X)^{-1}R'\right)^{-1}\bcancel{R(X'X)^{-1}R'\left(R(X'X)^{-1}R'\right)^{-1}}(R\hat{\beta} - q)\\
&=(R\hat{\beta} - q)'\left(R(X'X)^{-1}R'\right)^{-1}(R\hat{\beta} - q)
\end{align*}
Finally,
\[
\frac{W}{p} = \frac{(R\hat{\beta} - q)'\left(R(X'X)^{-1}R'\right)^{-1}(R\hat{\beta} - q)/p}{\hat\sigma^2} = \frac{(\tilde{\epsilon}'\tilde{\epsilon} - \hat{\epsilon}'\hat{\epsilon})/p}{\frac{\hat{\epsilon}'\hat{\epsilon}}{n-k}} = \frac{(RSS_r - RSS_u)/p}{RSS_u/(n - k)}
\]


\end{proof}


\end{document}
