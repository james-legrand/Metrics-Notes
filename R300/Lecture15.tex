\documentclass[DIV=14,titlepage=false]{scrreprt}
\usepackage{parskip}
%%%%%%%%%%%%%%%%%%%%%%%%%%%%%%%%%%%%%%%%%%%%%%%%%%%%%%%%%%%%%%%%%%%%%%%%%%%%%%%
%                                Basic Packages                               %
%%%%%%%%%%%%%%%%%%%%%%%%%%%%%%%%%%%%%%%%%%%%%%%%%%%%%%%%%%%%%%%%%%%%%%%%%%%%%%%
% Gives us multiple colors.
\usepackage[usenames,dvipsnames,pdftex]{xcolor}
% Lets us style link colors.
\usepackage{hyperref}
% Lets us import images and graphics.
\usepackage{graphicx}
% Lets us use figures in floating environments.
\usepackage{float}
% Lets us create multiple columns.
\usepackage{multicol}
% Gives us better math syntax.
\usepackage{amsmath,amsfonts,mathtools,amsthm,amssymb}
% Lets us strikethrough text.
\usepackage{cancel}
% Lets us edit the caption of a figure.
\usepackage{caption}
% Lets us import pdf directly in our tex code.
\usepackage{pdfpages}
% Lets us do algorithm stuff.
\usepackage[ruled,vlined,linesnumbered]{algorithm2e}
% Gets rid of some errors.
\usepackage{scrhack}
\def\class{article}
\usepackage{geometry}
\geometry{margin=0.9in}
%%%%%%%%%%%%%%%%%%%%%%%%%%%%%%%%%%%%%%%%%%%%%%%%%%%%%%%%%%%%%%%%%%%%%%%%%%%%%%%
%                                Basic Settings                               %
%%%%%%%%%%%%%%%%%%%%%%%%%%%%%%%%%%%%%%%%%%%%%%%%%%%%%%%%%%%%%%%%%%%%%%%%%%%%%%%

%%%%%%%%%%%%%
%  Symbols  %
%%%%%%%%%%%%%

\let\implies\Rightarrow
\let\impliedby\Leftarrow
\let\iff\Leftrightarrow
\let\epsilon\varepsilon

%%%%%%%%%%%%
%  Tables  %
%%%%%%%%%%%%

\setlength{\tabcolsep}{5pt}
\renewcommand\arraystretch{1.5}

%%%%%%%%%%%%%%%%%%%%%%%
%  Center Title Page  %
%%%%%%%%%%%%%%%%%%%%%%%

\usepackage{titling}
\renewcommand\maketitlehooka{\null\mbox{}\vfill}
\renewcommand\maketitlehookd{\vfill\null}

%%%%%%%%%%%%%%%%%%%%%%%%%%%%%%%%%%%%%%%%%%%%%%%%%%%%%%%
%  Create a grey background in the middle of the PDF  %
%%%%%%%%%%%%%%%%%%%%%%%%%%%%%%%%%%%%%%%%%%%%%%%%%%%%%%%

\usepackage{eso-pic}
\newcommand\definegraybackground{
  \definecolor{reallylightgray}{HTML}{FAFAFA}
  \AddToShipoutPicture{
    \ifthenelse{\isodd{\thepage}}{
      \AtPageLowerLeft{
        \put(\LenToUnit{\dimexpr\paperwidth-222pt},0){
          \color{reallylightgray}\rule{222pt}{297mm}
        }
      }
    }
    {
      \AtPageLowerLeft{
        \color{reallylightgray}\rule{222pt}{297mm}
      }
    }
  }
}

%%%%%%%%%%%%%%%%%%%%%%%%
%  Modify Links Color  %
%%%%%%%%%%%%%%%%%%%%%%%%

\hypersetup{
  % Enable highlighting links.
  colorlinks,
  % Change the color of links to blue.
  linkcolor={black},
  % Change the color of citations to black.
  citecolor={black},
  % Change the color of url's to blue with some black.
  urlcolor=blue
}

%%%%%%%%%%%%%%%%%%
% Fix WrapFigure %
%%%%%%%%%%%%%%%%%%

\newcommand{\wrapfill}{\par\ifnum\value{WF@wrappedlines}>0
    \parskip=0pt
    \addtocounter{WF@wrappedlines}{-1}%
    \null\vspace{\arabic{WF@wrappedlines}\baselineskip}%
    \WFclear
\fi}

%%%%%%%%%%%%%%%%%
% Multi Columns %
%%%%%%%%%%%%%%%%%

\let\multicolmulticols\multicols
\let\endmulticolmulticols\endmulticols

\RenewDocumentEnvironment{multicols}{mO{}}
{%
  \ifnum#1=1
    #2%
  \else % More than 1 column
    \multicolmulticols{#1}[#2]
  \fi
}
{%
  \ifnum#1=1
\else % More than 1 column
  \endmulticolmulticols
\fi
}

\newlength{\thickarrayrulewidth}
\setlength{\thickarrayrulewidth}{5\arrayrulewidth}

%%%%%%%%%%%%%%%%%%%%
%  Import Figures  %
%%%%%%%%%%%%%%%%%%%%

\usepackage{import}
\pdfminorversion=7

% EXAMPLE:
% 1. \incfig{limit-graph}
% 2. \incfig[0.4]{limit-graph}
% Parameters:
% 1. The figure name. It should be located in figures/NAME.tex_pdf.
% 2. (Optional) The width of the figure. Example: 0.5, 0.35.
\newcommand\incfig[2][1]{%
  \def\svgwidth{#1\columnwidth}
  \import{./figures/}{#2.pdf_tex}
}

\begingroup\expandafter\expandafter\expandafter\endgroup
\expandafter\ifx\csname pdfsuppresswarningpagegroup\endcsname\relax
\else
  \pdfsuppresswarningpagegroup=1\relax
\fi

%%%%%%%%%%%%%
%  Correct  %
%%%%%%%%%%%%%

% EXAMPLE:
% 1. \correct{INCORRECT}{CORRECT}
% Parameters:
% 1. The incorrect statement.
% 2. The correct statement.
\definecolor{correct}{HTML}{009900}
\newcommand\correct[2]{{\color{red}{#1 }}\ensuremath{\to}{\color{correct}{ #2}}}



\newcommand{\R}{\mathbb{R}}
\newcommand{\Z}{\mathbb{Z}}
\newcommand{\E}{\mathbb{E}}
\newcommand{\B}{\ensuremath{\mathcal{B}}}
\newcommand{\X}{\ensuremath{\mathcal{X}}}
\newcommand{\Y}{\ensuremath{\mathcal{Y}}}
\newcommand{\mA}{\ensuremath{\mathbf{A}}}
\newcommand{\mB}{\ensuremath{\mathbf{B}}}
\newcommand{\mC}{\ensuremath{\mathbf{C}}}
\newcommand{\mD}{\ensuremath{\mathbf{D}}}
\newcommand{\mX}{\ensuremath{\mathbf{X}}}
\newcommand{\mY}{\ensuremath{\mathbf{Y}}}
\newcommand{\mx}{\ensuremath{\mathbf{x}}}
\newcommand{\my}{\ensuremath{\mathbf{y}}}
\newcommand{\mI}{\ensuremath{\mathbf{I}}}
\newcommand{\mi}{\ensuremath{\mathbf{\iota}}}
\newcommand{\mmu}{\ensuremath{\mathbf{\mu}}}
\newcommand{\mc}{\ensuremath{\mathbf{c}}}
\newcommand{\mSigma}{\ensuremath{\mathbf{\Sigma}}}
\newcommand{\mzero}{\ensuremath{\mathbf{0}}}
\newcommand{\independent}{\perp\!\!\!\!\perp} 
\setlength{\parindent}{0pt}
%%%%%%%%%%%%%%%%%%%%%%%%%%%%%%%%%%%%%%%%%%%%%%%%%%%%%%%%%%%%%%%%%%%%%%%%%%%%%%%
%                                 Environments                                %
%%%%%%%%%%%%%%%%%%%%%%%%%%%%%%%%%%%%%%%%%%%%%%%%%%%%%%%%%%%%%%%%%%%%%%%%%%%%%%%

\usepackage{varwidth}
\usepackage{thmtools}
\usepackage[most,many,breakable]{tcolorbox}

\tcbuselibrary{theorems,skins,hooks}
\usetikzlibrary{arrows,calc,shadows.blur}

%%%%%%%%%%%%%%%%%%%
%  Define Colors  %
%%%%%%%%%%%%%%%%%%%

\definecolor{myblue}{RGB}{45, 111, 177}
\definecolor{mygreen}{RGB}{56, 140, 70}
\definecolor{myred}{RGB}{199, 68, 64}
\definecolor{mypurple}{RGB}{197, 92, 212}

\definecolor{definition}{HTML}{228b22}
\definecolor{theorem}{HTML}{00007B}
\definecolor{example}{HTML}{2A7F7F}
\definecolor{definition}{HTML}{228b22}
\definecolor{prop}{HTML}{191971}
\definecolor{lemma}{HTML}{983b0f}
\definecolor{exercise}{HTML}{88D6D1}

\colorlet{definition}{mygreen!85!black}
\colorlet{claim}{mygreen!85!black}
\colorlet{corollary}{mypurple!85!black}
\colorlet{proof}{theorem}

%%%%%%%%%%%%%%%%%%%%%%
%  Helpful Commands  %
%%%%%%%%%%%%%%%%%%%%%%

% EXAMPLE:
% 1. \createnewtheoremstyle{thmdefinitionbox}{}{}
% 2. \createnewtheoremstyle{thmtheorembox}{}{}
% 3. \createnewtheoremstyle{thmproofbox}{qed=\qedsymbol}{
%       rightline=false, topline=false, bottomline=false
%    }
% Parameters:
% 1. Theorem name.
% 2. Any extra parameters to pass directly to declaretheoremstyle.
% 3. Any extra parameters to pass directly to mdframed.
\newcommand\createnewtheoremstyle[3]{
  \declaretheoremstyle[
  headfont=\bfseries\sffamily, bodyfont=\normalfont, #2,
  mdframed={
    #3,
  },
  ]{#1}
}

% EXAMPLE:
% 1. \createnewcoloredtheoremstyle{thmdefinitionbox}{definition}{}{}
% 2. \createnewcoloredtheoremstyle{thmexamplebox}{example}{}{
%       rightline=true, leftline=true, topline=true, bottomline=true
%     }
% 3. \createnewcoloredtheoremstyle{thmproofbox}{proof}{qed=\qedsymbol}{backgroundcolor=white}
% Parameters:
% 1. Theorem name.
% 2. Color of theorem.
% 3. Any extra parameters to pass directly to declaretheoremstyle.
% 4. Any extra parameters to pass directly to mdframed.
\newcommand\createnewcoloredtheoremstyle[4]{
  \declaretheoremstyle[
  headfont=\bfseries\sffamily\color{#2}, bodyfont=\normalfont, #3,
  mdframed={
    linewidth=2pt,
    rightline=false, leftline=true, topline=false, bottomline=false,
    linecolor=#2, backgroundcolor=#2!5, #4,
  },
  ]{#1}
}

%%%%%%%%%%%%%%%%%%%%%%%%%%%%%%%%%%%
%  Create the Environment Styles  %
%%%%%%%%%%%%%%%%%%%%%%%%%%%%%%%%%%%

\makeatletter
\@ifclasswith\class{nocolor}{
  % Environments without color.

  \createnewtheoremstyle{thmdefinitionbox}{}{}
  \createnewtheoremstyle{thmtheorembox}{}{}
  \createnewtheoremstyle{thmexamplebox}{}{}
  \createnewtheoremstyle{thmclaimbox}{}{}
  \createnewtheoremstyle{thmcorollarybox}{}{}
  \createnewtheoremstyle{thmpropbox}{}{}
  \createnewtheoremstyle{thmlemmabox}{}{}
  \createnewtheoremstyle{thmexercisebox}{}{}
  \createnewtheoremstyle{thmdefinitionbox}{}{}
  \createnewtheoremstyle{thmquestionbox}{}{}
  \createnewtheoremstyle{thmsolutionbox}{}{}

  \createnewtheoremstyle{thmproofbox}{qed=\qedsymbol}{}
  \createnewtheoremstyle{thmexplanationbox}{}{}
}{
  % Environments with color.

  \createnewcoloredtheoremstyle{thmdefinitionbox}{definition}{}{}
  \createnewcoloredtheoremstyle{thmtheorembox}{theorem}{}{}
  \createnewcoloredtheoremstyle{thmexamplebox}{example}{}{
    rightline=true, leftline=true, topline=true, bottomline=true
  }
  \createnewcoloredtheoremstyle{thmclaimbox}{claim}{}{}
  \createnewcoloredtheoremstyle{thmcorollarybox}{corollary}{}{}
  \createnewcoloredtheoremstyle{thmpropbox}{prop}{}{}
  \createnewcoloredtheoremstyle{thmlemmabox}{lemma}{}{}
  \createnewcoloredtheoremstyle{thmexercisebox}{exercise}{}{}

  \createnewcoloredtheoremstyle{thmproofbox}{proof}{qed=\qedsymbol}{backgroundcolor=white}
  \createnewcoloredtheoremstyle{thmexplanationbox}{example}{qed=\qedsymbol}{backgroundcolor=white}
}
\makeatother

%%%%%%%%%%%%%%%%%%%%%%%%%%%%%
%  Create the Environments  %
%%%%%%%%%%%%%%%%%%%%%%%%%%%%%

\declaretheorem[numberwithin=section, style=thmtheorembox,     name=Theorem]{theorem}
\declaretheorem[numbered=no,          style=thmexamplebox,     name=Example]{example}
\declaretheorem[numberwithin=section, style=thmclaimbox,       name=Claim]{claim}
\declaretheorem[numberwithin=section, style=thmcorollarybox,   name=Corollary]{corollary}
\declaretheorem[numberwithin=section, style=thmpropbox,        name=Proposition]{prop}
\declaretheorem[numberwithin=section, style=thmlemmabox,       name=Lemma]{lemma}
\declaretheorem[numberwithin=section, style=thmexercisebox,    name=Exercise]{exercise}
\declaretheorem[numbered=no,          style=thmproofbox,       name=Proof]{replacementproof}
\declaretheorem[numbered=no,          style=thmexplanationbox, name=Proof]{expl}

\makeatletter
\@ifclasswith\class{nocolor}{
  % Environments without color.

  \newtheorem*{note}{Note}

  \declaretheorem[numberwithin=section, style=thmdefinitionbox, name=Definition]{definition}
  \declaretheorem[numberwithin=section, style=thmquestionbox,   name=Question]{question}
  \declaretheorem[numberwithin=section, style=thmsolutionbox,   name=Solution]{solution}
}{
  % Environments with color.

  \newtcbtheorem[number within=section]{Definition}{Definition}{
    enhanced,
    before skip=2mm,
    after skip=2mm,
    colback=red!5,
    colframe=red!80!black,
    colbacktitle=red!75!black,
    boxrule=0.5mm,
    attach boxed title to top left={
      xshift=1cm,
      yshift*=1mm-\tcboxedtitleheight
    },
    varwidth boxed title*=-3cm,
    boxed title style={
      interior engine=empty,
      frame code={
        \path[fill=tcbcolback]
        ([yshift=-1mm,xshift=-1mm]frame.north west)
        arc[start angle=0,end angle=180,radius=1mm]
        ([yshift=-1mm,xshift=1mm]frame.north east)
        arc[start angle=180,end angle=0,radius=1mm];
        \path[left color=tcbcolback!60!black,right color=tcbcolback!60!black,
        middle color=tcbcolback!80!black]
        ([xshift=-2mm]frame.north west) -- ([xshift=2mm]frame.north east)
        [rounded corners=1mm]-- ([xshift=1mm,yshift=-1mm]frame.north east)
        -- (frame.south east) -- (frame.south west)
        -- ([xshift=-1mm,yshift=-1mm]frame.north west)
        [sharp corners]-- cycle;
      },
    },
    fonttitle=\bfseries,
    title={#2},
    #1
  }{def}

  \NewDocumentEnvironment{definition}{O{}O{}}
    {\begin{Definition}{#1}{#2}}{\end{Definition}}

  \newtcolorbox{note}[1][]{%
    enhanced jigsaw,
    colback=gray!20!white,%
    colframe=gray!80!black,
    size=small,
    boxrule=1pt,
    title=\textbf{Note:-},
    halign title=flush center,
    coltitle=black,
    breakable,
    drop shadow=black!50!white,
    attach boxed title to top left={xshift=1cm,yshift=-\tcboxedtitleheight/2,yshifttext=-\tcboxedtitleheight/2},
    minipage boxed title=1.5cm,
    boxed title style={%
      colback=white,
      size=fbox,
      boxrule=1pt,
      boxsep=2pt,
      underlay={%
        \coordinate (dotA) at ($(interior.west) + (-0.5pt,0)$);
        \coordinate (dotB) at ($(interior.east) + (0.5pt,0)$);
        \begin{scope}
          \clip (interior.north west) rectangle ([xshift=3ex]interior.east);
          \filldraw [white, blur shadow={shadow opacity=60, shadow yshift=-.75ex}, rounded corners=2pt] (interior.north west) rectangle (interior.south east);
        \end{scope}
        \begin{scope}[gray!80!black]
          \fill (dotA) circle (2pt);
          \fill (dotB) circle (2pt);
        \end{scope}
      },
    },
    #1,
  }

  \newtcbtheorem{Question}{Question}{enhanced,
    breakable,
    colback=white,
    colframe=myblue!80!black,
    attach boxed title to top left={yshift*=-\tcboxedtitleheight},
    fonttitle=\bfseries,
    title=\textbf{Question:-},
    boxed title size=title,
    boxed title style={%
      sharp corners,
      rounded corners=northwest,
      colback=tcbcolframe,
      boxrule=0pt,
    },
    underlay boxed title={%
      \path[fill=tcbcolframe] (title.south west)--(title.south east)
      to[out=0, in=180] ([xshift=5mm]title.east)--
      (title.center-|frame.east)
      [rounded corners=\kvtcb@arc] |-
      (frame.north) -| cycle;
    },
    #1
  }{def}

  \NewDocumentEnvironment{question}{O{}O{}}
  {\begin{Question}{#1}{#2}}{\end{Question}}

  \newtcolorbox{Solution}{enhanced,
    breakable,
    colback=white,
    colframe=mygreen!80!black,
    attach boxed title to top left={yshift*=-\tcboxedtitleheight},
    title=\textbf{Solution:-},
    boxed title size=title,
    boxed title style={%
      sharp corners,
      rounded corners=northwest,
      colback=tcbcolframe,
      boxrule=0pt,
    },
    underlay boxed title={%
      \path[fill=tcbcolframe] (title.south west)--(title.south east)
      to[out=0, in=180] ([xshift=5mm]title.east)--
      (title.center-|frame.east)
      [rounded corners=\kvtcb@arc] |-
      (frame.north) -| cycle;
    },
  }

  \NewDocumentEnvironment{solution}{O{}O{}}
  {\vspace{-10pt}\begin{Solution}{#1}{#2}}{\end{Solution}}
}
\makeatother

%%%%%%%%%%%%%%%%%%%%%%%%%%%%
%  Edit Proof Environment  %
%%%%%%%%%%%%%%%%%%%%%%%%%%%%

\renewenvironment{proof}[1][\proofname]{\vspace{-10pt}\begin{replacementproof}}{\end{replacementproof}}
\newenvironment{explanation}[1][\proofname]{\vspace{-10pt}\begin{expl}}{\end{expl}}

\theoremstyle{definition}

\newtheorem*{notation}{Notation}
\newtheorem*{previouslyseen}{As previously seen}
\newtheorem*{problem}{Problem}
\newtheorem*{observe}{Observe}
\newtheorem*{property}{Property}
\newtheorem*{intuition}{Intuition}



\title{%
R300 Econometrics}
\author{Metrics Enjoyers}
\date{Michaelmas Term, 2023-2024}

\setuptoc{toc}{leveldown}
\setcounter{chapter}{14}

\begin{document}
\pagenumbering{gobble}
\chapter{Irrelevant and Weak Instruments}

\section{Irrelevant Instruments}
Occurs when \(E(w_ix_i')\) is not of full column rank, violating IV1.
In this case, parameter \(\beta\) is not identified. Consider:
\[y_i=x_i\beta+\epsilon_i\]
\[x_i-w_i\gamma+e_i\]
with one-dimernsional endogenous variable \(x_i\), and instrument \(w_i\).
Satisfying \(E(w_i\epsilon_i)=0\), but fails the relevance assumption, so that \(\gamma=0\) and hence \(E(w_ix_i)=0\).
\\ The system of equations (moment conditions):
\[E(w_i\epsilon_i)=0\]
\[E(w_ix_i)=0\]
is equivalent to
\[\begin{cases} E(w_i(y_i-x_i\beta))=0 \\ E(w_ix_i)=0 \end{cases} \iff \begin{cases}E(w_iy_i) \\ E(w_ix_i)\end{cases}\]
which tells us nothing about \(\beta\) and so it is not identified.
\\
We can still compute IV estimator as unlikely \(W'X\) exactly zero in a finite sample:
\vspace{5mm}
\begin{prop}
    Under non-identifiability:
    \begin{itemize}
        \item \(\hat\beta_{IV}\) does not converge in probability to a limit. Instead it converges in distribution to a RV. In particular, \(\hat\beta_{IV}\) is not consistent.
        \item The limiting distribution of \(\hat\beta_{IV}\) is not centered at \(\beta\) but instead has its median at \(\beta+\rho\), like \(\beta_{OLS}\).
        \item \(\hat \beta_{IV}\) will have very wild fluctuations in finite samples, since the ratio \(\xi_0/\xi_2\) is Cauchy distributed and has no moments, due its fat tails.
    \end{itemize}
\end{prop}

\vspace{5mm}
\begin{proof}
For simplicity we assume homoskedasticity and suppose:
\[E(e_i|w_i)=E(\epsilon_i|w_i)=0\]
\[Var(e_i|w_i)=Var(\epsilon_i|w_i)=1, Cov(e_i,\epsilon_i|w_i)=\rho\neq0\]
\[Ew_i=0, Ew_i^2=1\] 

By CLT:
\[\frac{1}{\sqrt{n}}\sum_{i=1}^n\begin{pmatrix}w_i\epsilon_i \\ w_ie_i \end{pmatrix}\xrightarrow{d}\begin{pmatrix}\xi_1 \\ \xi_2\end{pmatrix}\sim N\left(0,\begin{pmatrix}1&\rho \\\rho & 1\end{pmatrix}\right)\]
Note: \(\xi_0=\xi_1-\rho\xi_2\) is a normal random variable, independent from \(\xi_2\)
\[E(\xi_0|\xi_2)=E(\xi_1-\rho\xi_2|\xi_2)=E(\xi_1|\xi_2)-\rho\xi_2=0\]
\\
Then (using \(\gamma=0 \implies x_i=e_i\))
\[\hat\beta_{OLS}-\beta=\frac{\frac{1}{n}\sum x_i\epsilon_i}{\frac{1}{n}\sum x_i^2}=\frac{\frac{1}{n}\sum e_i\epsilon_i}{\frac{1}{n}\sum e_i^2}\xrightarrow{p}\rho\neq0\]
\[\hat\beta_{IV}-\beta= \frac{\frac{1}{\sqrt{n}}\sum w_i\epsilon_i}{\frac{1}{\sqrt{n}}\sum w_ix_i}=\frac{\frac{1}{\sqrt{n}}\sum w_i\epsilon_i}{\frac{1}{\sqrt{n}}\sum w_ie_i}\xrightarrow{d}\frac{\xi_1}{\xi_2}=\rho+\frac{\xi_0}{\xi_2}\]
\end{proof}
\vspace{5mm}
\begin{prop}
    Under non-identifiability, the \(\beta\) t-statistics will diverge, such that we may conclude statistically significant estimates when they are in fact useless. (we prove this explicitly below in the case when \(\rho\rightarrow1\), i.e. lots of endogeneity)
    \[|t|\xrightarrow{p}\infty\]
\end{prop}
\vspace{5mm}
\begin{proof}
    \[\hat \sigma^2=\frac{1}{n}\sum(y_i-x_i\hat\beta_{IV})^2=\frac{1}{n}\sum (\epsilon_i-x_i(\hat\beta_{IV}-\beta))^2\]
    \[=\frac{1}{n}\sum(\epsilon_i-e_i(\hat\beta_{IV}-\beta))^2\]
    \[\frac{1}{n}\sum \epsilon_i^2 - 2(\hat\beta_{IV}-\beta)\frac{1}{n}\sum \epsilon_ie_i+(\hat\beta_{IV}-\beta)^2\frac{1}{n}\sum e_i^2\]
    \[\xrightarrow{d}1-2\rho\left(\rho+\frac{\xi_0}{\xi_2}\right)+\left(\rho+\frac{\xi_0}{\xi_2}\right)^2\]
    \[=1-\rho^2+\left(\frac{\xi_0}{\xi_2}\right)^2\]

Therefore, the \(t\)-statistic for \(H_0: \beta=\beta_0, H1:\beta\neq\beta_0\) has the asymptotic distribution:
\[t=\frac{\hat \beta_{IV}-\beta_0}{\sqrt{\hat{Var}(\hat \beta_{IV})}}; \qquad {Var}(\hat \beta_{IV})=\sigma^2(D'C^{-1}D)^{-1}\]
Replacing with sample analogues:
\[\hat{Var}(\hat \beta_{IV})=\hat\sigma^2(\hat D'\hat C^{-1}\hat D)^{-1}\]
In the 1D case:
\[\hat D=\frac{1}{\sqrt{n}}\sum w_ix_i\]
\[\hat C=\frac{1}{n}\sum w_i^2\]
\[\therefore t=\frac{\hat \beta_{IV}-\beta}{\sqrt{\hat\sigma^2\frac{1}{n}\sum w_i^2}/\frac{1}{\sqrt{n}}|\sum w_ix_i|}=\frac{\hat \beta_{IV}-\beta}{\sqrt{\hat \sigma^2\frac{1}{n}\sum w_i^2}/\frac{1}{\sqrt{n}}|\sum w_ie_i|}\]
\[\xrightarrow{d}\frac{\rho+\frac{\xi_0}{\xi_2}}{\sqrt{1-\rho^2+\left(\frac{\xi_0}{\xi_2}\right)^2}/|\xi_2|}\]

This distribution is non-normal.
Note when \(\rho \rightarrow 1\)
\[Var(\xi_0)=Var(\xi_1-\rho\xi_2)=1-2\rho^2+\rho^2\xrightarrow 0\]
so that
\[\xi_0\xrightarrow{p}0 \text{ and } 1-\rho^2+(\frac{xi_0}{\xi_2})^2\xrightarrow{p}0\]
This implies that \[|t|\xrightarrow{p}\infty\]
\end{proof}

\section{Weak Instruments}
When \(E(w_ix_i')\) is of full column rank but \(E(w_iw_i')^{-1}E(w_ix_i')\) (the coefficient matrix in the first stage regression) is small, the instruments, although relevant, are \textbf{\underline{weak}}.

\begin{prop}
    Under weak instruments, \(\hat\beta_{IV}\) will still not be consistent for \(\beta\) and will again have a non normal distribution.
\end{prop}
\vspace{5mm}
\begin{proof}
Consider the same 1D model as previous, except instead of \(E(w_ix_i)=0\) We assume it is 'small', modelled by the 'local-to-zero' assumption:
\[E(w_ix_i)=\gamma=\frac{1}{\sqrt{n}}\mu\]
where \(\mu\) is fixed, which will yield useful asymptotic approximations for \(\hat\beta_{IV}\).
Large \(\mu\) corresponds to relatively strong instruments, whereas small \(\mu\) corresponds to almost irrelevant instruments.
\\ For \(\hat\beta_{OLS}\) as before:
\[\hat\beta_{OLS}-\beta=\frac{\frac{1}{n}\sum x_i\epsilon_i}{\frac{1}{n}\sum x_i^2}=\frac{\frac{1}{n}\sum (\frac{\mu}{\sqrt{n}}w_i+e_i)\epsilon_i}{\frac{1}{n}\sum (e_i^2+\frac{2\mu}{\sqrt{n}}w_ie_i+\frac{\mu^2}{n}w_i^2)}\xrightarrow{p}\rho\neq0\]
\\ For \(\hat\beta_{IV}\):
\[\hat\beta_{IV}-\beta=\frac{\sum w_i\epsilon_i}{\sum w_ix_i}=\frac{\frac{1}{\sqrt{n}}\sum w_i\epsilon_i}{\frac{1}{\sqrt{n}}\sum w_i(\frac{\mu}{\sqrt{n}}w_i+e_i)}\]
\[=\frac{\frac{1}{\sqrt{n}}\sum w_i\epsilon_i}{\mu\frac{1}{n}\sum w_i^2+\frac{1}{\sqrt{n}}\sum w_ie_i}\xrightarrow{d}\frac{\xi_1}{\mu+\xi_2}=\frac{\xi_0}{\mu+\xi_2}+\rho\frac{\xi_2}{\mu+\xi_2}\]
Thus as in the case of irrelevant instruments, \(\hat\beta_{IV}\) is not consistent for \(\beta\) and has a non-normal asymptotic distribution.
When \(\mu\rightarrow\infty\), so the instruments become strong, the consistency is restored because \(\frac{\xi_1}{\mu+\xi_2}\xrightarrow{p}0\text{ as }\mu\rightarrow\infty\).
\end{proof}
\vspace{5mm}
\begin{prop}
The t-statistic for \(\beta\) will inflate as instruments become weaker and the distribution will become closer to \(\chi^2(1)/\|\mu|\). 
\end{prop}
\vspace{5mm}
\begin{proof}
First consider \(\hat\sigma^2\):
\[\hat \sigma^2=\frac{1}{n}\sum(y_i-x_i\hat\beta_{IV})^2=\frac{1}{n}\sum (\epsilon_i-x_i(\hat\beta_{IV}-\beta))^2\]
\[\frac{1}{n}\sum \epsilon_i^2 - 2(\hat\beta_{IV}-\beta)\frac{1}{n}\sum \epsilon_ix_i+(\hat\beta_{IV}-\beta)^2\frac{1}{n}\sum x_i^2\]
But
\[\frac{1}{n}\sum \epsilon_ix_i=\frac{1}{n}\sum \epsilon_i\left(\frac{\mu}{\sqrt{n}}w_i+e_i\right)\xrightarrow{p}\rho \text{ and}\]
\[\frac{1}{n}\sum x_i^2\xrightarrow{p}1\]

Hence,
\[\hat\sigma^2\xrightarrow{d}1-2\rho\frac{\xi_1}{\mu+\xi_2}+\left(\frac{\xi_1}{\mu+\xi_2}\right)^2\]
\[=1-\rho^2+\left(\rho-\frac{\xi_1}{\mu+\xi_2}\right)^2\]
\[=1-\rho^2+\left(\frac{\rho\mu-\xi_0}{\mu+\xi_2}\right)^2\]

Thus for the t-statistic:
\[t=\frac{\hat \beta_{IV}-\beta}{\sqrt{\hat\sigma^2\frac{1}{n}\sum w_i^2}/\frac{1}{\sqrt{n}}|\sum w_ix_i|}\xrightarrow{d}\frac{\xi_1/(\mu+\xi_2)}{\sqrt{1-\rho^2+(\frac{\rho\mu-\xi_0}{\mu+\xi_2})^2}/|\mu+\xi_2|}\]
\[=\frac{\xi_1}{sgn(\mu+\xi_2)\sqrt{(1-\rho^2)+(\frac{\rho\mu-\xi_0}{\mu+\xi_2})^2}}\]

This has a non-normal distribution.
\\ Again we consider the extreme case where \(\rho=1\) (lots of endogeneity in \(x_i\)):
\\ \(\rho=1 \implies \xi_1=\xi_2=\xi\sim N(0,1)\), \(\xi_0=\xi-\rho\xi=0\)
\[t\xrightarrow{d}\frac{\xi(\mu+\xi)}{|\mu|}\]
With \(|\mu|\) very large (strong instruments), \(t\) is almost normal, but when \(|\mu|\) is small, \(t\) is almost \(\chi^2(1)/|\mu|\).
\\ As \(\mu\rightarrow0\), \(t\xrightarrow{p}\infty\).
Thus there is a multiplicity of possibilities when \(\mu\) varies.
\end{proof}
\vspace{5mm}
\subsection{Classifying Weak Instruments}
Stock and Yogo (2005) classify strength of instruments based on the size distortion of the nominal 5\% significance asymptotic t-test, which rejects the null hypothesis iff \(|t|>1.96\).
\\ \\ No distortion implies:
\[Pr(|t|>1.96)\rightarrow0.05\]
But with \(\mu<\infty\), such convergence will not take place:
\[Pr(|t|>1.96)=Pr\left(|\frac{\xi(\mu+\xi)}{|\mu|}|>1.96\right)\nrightarrow0.05\]
Thus the actual (as opposed to nominal(i.e. intended)) size of the test will not be 5\%.
\\ \\ Stock and Yogo suggested that a 'tolerable' actual size should be perhaps \underline{\smash{not larger than 15\%}}.
\\ \\ Let \(\tau^2\) be such that, whenever \(\mu^2\geq\tau^2\) the actual size of the t-test is below 15\%. \(\tau^2\) is found through simulating the \(\frac{\xi(\mu+\xi)}{|\mu|}\) distribution.
\\ \\ Then proposed to use the F-stat from first stage regression (normally used to test hypothesis \(\mu=0\)) to test hypothesis that \(\mu^2\leq\tau^2\).
\\ \\ By simulation they found the appropriate critical value is \textbf{approx 10}, though this is dependent on choosing 15\% as the benchmark distortion and number of instruments used.
\\ But for not too many instruemnts critical value remains around 10 so this is used in empirical literature often to determine if instruments weak.

\end{document}