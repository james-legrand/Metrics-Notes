\documentclass[DIV=14,titlepage=false]{scrreprt}

%%%%%%%%%%%%%%%%%%%%%%%%%%%%%%%%%%%%%%%%%%%%%%%%%%%%%%%%%%%%%%%%%%%%%%%%%%%%%%%
%                                Basic Packages                               %
%%%%%%%%%%%%%%%%%%%%%%%%%%%%%%%%%%%%%%%%%%%%%%%%%%%%%%%%%%%%%%%%%%%%%%%%%%%%%%%
% Gives us multiple colors.
\usepackage[usenames,dvipsnames,pdftex]{xcolor}
% Lets us style link colors.
\usepackage{hyperref}
% Lets us import images and graphics.
\usepackage{graphicx}
% Lets us use figures in floating environments.
\usepackage{float}
% Lets us create multiple columns.
\usepackage{multicol}
% Gives us better math syntax.
\usepackage{amsmath,amsfonts,mathtools,amsthm,amssymb}
% Lets us strikethrough text.
\usepackage{cancel}
% Lets us edit the caption of a figure.
\usepackage{caption}
% Lets us import pdf directly in our tex code.
\usepackage{pdfpages}
% Lets us do algorithm stuff.
\usepackage[ruled,vlined,linesnumbered]{algorithm2e}
% Gets rid of some errors.
\usepackage{scrhack}
\def\class{article}
\usepackage{geometry}
\geometry{margin=0.9in}
%%%%%%%%%%%%%%%%%%%%%%%%%%%%%%%%%%%%%%%%%%%%%%%%%%%%%%%%%%%%%%%%%%%%%%%%%%%%%%%
%                                Basic Settings                               %
%%%%%%%%%%%%%%%%%%%%%%%%%%%%%%%%%%%%%%%%%%%%%%%%%%%%%%%%%%%%%%%%%%%%%%%%%%%%%%%

%%%%%%%%%%%%%
%  Symbols  %
%%%%%%%%%%%%%

\let\implies\Rightarrow
\let\impliedby\Leftarrow
\let\iff\Leftrightarrow
\let\epsilon\varepsilon

%%%%%%%%%%%%
%  Tables  %
%%%%%%%%%%%%

\setlength{\tabcolsep}{5pt}
\renewcommand\arraystretch{1.5}

%%%%%%%%%%%%%%%%%%%%%%%
%  Center Title Page  %
%%%%%%%%%%%%%%%%%%%%%%%

\usepackage{titling}
\renewcommand\maketitlehooka{\null\mbox{}\vfill}
\renewcommand\maketitlehookd{\vfill\null}

%%%%%%%%%%%%%%%%%%%%%%%%%%%%%%%%%%%%%%%%%%%%%%%%%%%%%%%
%  Create a grey background in the middle of the PDF  %
%%%%%%%%%%%%%%%%%%%%%%%%%%%%%%%%%%%%%%%%%%%%%%%%%%%%%%%

\usepackage{eso-pic}
\newcommand\definegraybackground{
  \definecolor{reallylightgray}{HTML}{FAFAFA}
  \AddToShipoutPicture{
    \ifthenelse{\isodd{\thepage}}{
      \AtPageLowerLeft{
        \put(\LenToUnit{\dimexpr\paperwidth-222pt},0){
          \color{reallylightgray}\rule{222pt}{297mm}
        }
      }
    }
    {
      \AtPageLowerLeft{
        \color{reallylightgray}\rule{222pt}{297mm}
      }
    }
  }
}

%%%%%%%%%%%%%%%%%%%%%%%%
%  Modify Links Color  %
%%%%%%%%%%%%%%%%%%%%%%%%

\hypersetup{
  % Enable highlighting links.
  colorlinks,
  % Change the color of links to blue.
  linkcolor={black},
  % Change the color of citations to black.
  citecolor={black},
  % Change the color of url's to blue with some black.
  urlcolor=blue
}

%%%%%%%%%%%%%%%%%%
% Fix WrapFigure %
%%%%%%%%%%%%%%%%%%

\newcommand{\wrapfill}{\par\ifnum\value{WF@wrappedlines}>0
    \parskip=0pt
    \addtocounter{WF@wrappedlines}{-1}%
    \null\vspace{\arabic{WF@wrappedlines}\baselineskip}%
    \WFclear
\fi}

%%%%%%%%%%%%%%%%%
% Multi Columns %
%%%%%%%%%%%%%%%%%

\let\multicolmulticols\multicols
\let\endmulticolmulticols\endmulticols

\RenewDocumentEnvironment{multicols}{mO{}}
{%
  \ifnum#1=1
    #2%
  \else % More than 1 column
    \multicolmulticols{#1}[#2]
  \fi
}
{%
  \ifnum#1=1
\else % More than 1 column
  \endmulticolmulticols
\fi
}

\newlength{\thickarrayrulewidth}
\setlength{\thickarrayrulewidth}{5\arrayrulewidth}

%%%%%%%%%%%%%%%%%%%%
%  Import Figures  %
%%%%%%%%%%%%%%%%%%%%

\usepackage{import}
\pdfminorversion=7

% EXAMPLE:
% 1. \incfig{limit-graph}
% 2. \incfig[0.4]{limit-graph}
% Parameters:
% 1. The figure name. It should be located in figures/NAME.tex_pdf.
% 2. (Optional) The width of the figure. Example: 0.5, 0.35.
\newcommand\incfig[2][1]{%
  \def\svgwidth{#1\columnwidth}
  \import{./figures/}{#2.pdf_tex}
}

\begingroup\expandafter\expandafter\expandafter\endgroup
\expandafter\ifx\csname pdfsuppresswarningpagegroup\endcsname\relax
\else
  \pdfsuppresswarningpagegroup=1\relax
\fi

%%%%%%%%%%%%%
%  Correct  %
%%%%%%%%%%%%%

% EXAMPLE:
% 1. \correct{INCORRECT}{CORRECT}
% Parameters:
% 1. The incorrect statement.
% 2. The correct statement.
\definecolor{correct}{HTML}{009900}
\newcommand\correct[2]{{\color{red}{#1 }}\ensuremath{\to}{\color{correct}{ #2}}}



\newcommand{\R}{\mathbb{R}}
\newcommand{\Z}{\mathbb{Z}}
\newcommand{\E}{\mathbb{E}}
\newcommand{\B}{\ensuremath{\mathcal{B}}}
\newcommand{\X}{\ensuremath{\mathcal{X}}}
\newcommand{\Y}{\ensuremath{\mathcal{Y}}}
\newcommand{\mA}{\ensuremath{\mathbf{A}}}
\newcommand{\mB}{\ensuremath{\mathbf{B}}}
\newcommand{\mC}{\ensuremath{\mathbf{C}}}
\newcommand{\mD}{\ensuremath{\mathbf{D}}}
\newcommand{\mX}{\ensuremath{\mathbf{X}}}
\newcommand{\mY}{\ensuremath{\mathbf{Y}}}
\newcommand{\mx}{\ensuremath{\mathbf{x}}}
\newcommand{\my}{\ensuremath{\mathbf{y}}}
\newcommand{\mI}{\ensuremath{\mathbf{I}}}
\newcommand{\mi}{\ensuremath{\mathbf{\iota}}}
\newcommand{\mmu}{\ensuremath{\mathbf{\mu}}}
\newcommand{\mc}{\ensuremath{\mathbf{c}}}
\newcommand{\mSigma}{\ensuremath{\mathbf{\Sigma}}}
\newcommand{\mzero}{\ensuremath{\mathbf{0}}}
\newcommand{\independent}{\perp\!\!\!\!\perp} 
\setlength{\parindent}{0pt}
%%%%%%%%%%%%%%%%%%%%%%%%%%%%%%%%%%%%%%%%%%%%%%%%%%%%%%%%%%%%%%%%%%%%%%%%%%%%%%%
%                                 Environments                                %
%%%%%%%%%%%%%%%%%%%%%%%%%%%%%%%%%%%%%%%%%%%%%%%%%%%%%%%%%%%%%%%%%%%%%%%%%%%%%%%

\usepackage{varwidth}
\usepackage{thmtools}
\usepackage[most,many,breakable]{tcolorbox}

\tcbuselibrary{theorems,skins,hooks}
\usetikzlibrary{arrows,calc,shadows.blur}

%%%%%%%%%%%%%%%%%%%
%  Define Colors  %
%%%%%%%%%%%%%%%%%%%

\definecolor{myblue}{RGB}{45, 111, 177}
\definecolor{mygreen}{RGB}{56, 140, 70}
\definecolor{myred}{RGB}{199, 68, 64}
\definecolor{mypurple}{RGB}{197, 92, 212}

\definecolor{definition}{HTML}{228b22}
\definecolor{theorem}{HTML}{00007B}
\definecolor{example}{HTML}{2A7F7F}
\definecolor{definition}{HTML}{228b22}
\definecolor{prop}{HTML}{191971}
\definecolor{lemma}{HTML}{983b0f}
\definecolor{exercise}{HTML}{88D6D1}

\colorlet{definition}{mygreen!85!black}
\colorlet{claim}{mygreen!85!black}
\colorlet{corollary}{mypurple!85!black}
\colorlet{proof}{theorem}

%%%%%%%%%%%%%%%%%%%%%%
%  Helpful Commands  %
%%%%%%%%%%%%%%%%%%%%%%

% EXAMPLE:
% 1. \createnewtheoremstyle{thmdefinitionbox}{}{}
% 2. \createnewtheoremstyle{thmtheorembox}{}{}
% 3. \createnewtheoremstyle{thmproofbox}{qed=\qedsymbol}{
%       rightline=false, topline=false, bottomline=false
%    }
% Parameters:
% 1. Theorem name.
% 2. Any extra parameters to pass directly to declaretheoremstyle.
% 3. Any extra parameters to pass directly to mdframed.
\newcommand\createnewtheoremstyle[3]{
  \declaretheoremstyle[
  headfont=\bfseries\sffamily, bodyfont=\normalfont, #2,
  mdframed={
    #3,
  },
  ]{#1}
}

% EXAMPLE:
% 1. \createnewcoloredtheoremstyle{thmdefinitionbox}{definition}{}{}
% 2. \createnewcoloredtheoremstyle{thmexamplebox}{example}{}{
%       rightline=true, leftline=true, topline=true, bottomline=true
%     }
% 3. \createnewcoloredtheoremstyle{thmproofbox}{proof}{qed=\qedsymbol}{backgroundcolor=white}
% Parameters:
% 1. Theorem name.
% 2. Color of theorem.
% 3. Any extra parameters to pass directly to declaretheoremstyle.
% 4. Any extra parameters to pass directly to mdframed.
\newcommand\createnewcoloredtheoremstyle[4]{
  \declaretheoremstyle[
  headfont=\bfseries\sffamily\color{#2}, bodyfont=\normalfont, #3,
  mdframed={
    linewidth=2pt,
    rightline=false, leftline=true, topline=false, bottomline=false,
    linecolor=#2, backgroundcolor=#2!5, #4,
  },
  ]{#1}
}

%%%%%%%%%%%%%%%%%%%%%%%%%%%%%%%%%%%
%  Create the Environment Styles  %
%%%%%%%%%%%%%%%%%%%%%%%%%%%%%%%%%%%

\makeatletter
\@ifclasswith\class{nocolor}{
  % Environments without color.

  \createnewtheoremstyle{thmdefinitionbox}{}{}
  \createnewtheoremstyle{thmtheorembox}{}{}
  \createnewtheoremstyle{thmexamplebox}{}{}
  \createnewtheoremstyle{thmclaimbox}{}{}
  \createnewtheoremstyle{thmcorollarybox}{}{}
  \createnewtheoremstyle{thmpropbox}{}{}
  \createnewtheoremstyle{thmlemmabox}{}{}
  \createnewtheoremstyle{thmexercisebox}{}{}
  \createnewtheoremstyle{thmdefinitionbox}{}{}
  \createnewtheoremstyle{thmquestionbox}{}{}
  \createnewtheoremstyle{thmsolutionbox}{}{}

  \createnewtheoremstyle{thmproofbox}{qed=\qedsymbol}{}
  \createnewtheoremstyle{thmexplanationbox}{}{}
}{
  % Environments with color.

  \createnewcoloredtheoremstyle{thmdefinitionbox}{definition}{}{}
  \createnewcoloredtheoremstyle{thmtheorembox}{theorem}{}{}
  \createnewcoloredtheoremstyle{thmexamplebox}{example}{}{
    rightline=true, leftline=true, topline=true, bottomline=true
  }
  \createnewcoloredtheoremstyle{thmclaimbox}{claim}{}{}
  \createnewcoloredtheoremstyle{thmcorollarybox}{corollary}{}{}
  \createnewcoloredtheoremstyle{thmpropbox}{prop}{}{}
  \createnewcoloredtheoremstyle{thmlemmabox}{lemma}{}{}
  \createnewcoloredtheoremstyle{thmexercisebox}{exercise}{}{}

  \createnewcoloredtheoremstyle{thmproofbox}{proof}{qed=\qedsymbol}{backgroundcolor=white}
  \createnewcoloredtheoremstyle{thmexplanationbox}{example}{qed=\qedsymbol}{backgroundcolor=white}
}
\makeatother

%%%%%%%%%%%%%%%%%%%%%%%%%%%%%
%  Create the Environments  %
%%%%%%%%%%%%%%%%%%%%%%%%%%%%%

\declaretheorem[numberwithin=section, style=thmtheorembox,     name=Theorem]{theorem}
\declaretheorem[numbered=no,          style=thmexamplebox,     name=Example]{example}
\declaretheorem[numberwithin=section, style=thmclaimbox,       name=Claim]{claim}
\declaretheorem[numberwithin=section, style=thmcorollarybox,   name=Corollary]{corollary}
\declaretheorem[numberwithin=section, style=thmpropbox,        name=Proposition]{prop}
\declaretheorem[numberwithin=section, style=thmlemmabox,       name=Lemma]{lemma}
\declaretheorem[numberwithin=section, style=thmexercisebox,    name=Exercise]{exercise}
\declaretheorem[numbered=no,          style=thmproofbox,       name=Proof]{replacementproof}
\declaretheorem[numbered=no,          style=thmexplanationbox, name=Explanation]{expl}

\makeatletter
\@ifclasswith\class{nocolor}{
  % Environments without color.

  \newtheorem*{note}{Note}

  \declaretheorem[numberwithin=section, style=thmdefinitionbox, name=Definition]{definition}
  \declaretheorem[numberwithin=section, style=thmquestionbox,   name=Question]{question}
  \declaretheorem[numberwithin=section, style=thmsolutionbox,   name=Solution]{solution}
}{
  % Environments with color.

  \newtcbtheorem[number within=section]{Definition}{Definition}{
    enhanced,
    before skip=2mm,
    after skip=2mm,
    colback=red!5,
    colframe=red!80!black,
    colbacktitle=red!75!black,
    boxrule=0.5mm,
    attach boxed title to top left={
      xshift=1cm,
      yshift*=1mm-\tcboxedtitleheight
    },
    varwidth boxed title*=-3cm,
    boxed title style={
      interior engine=empty,
      frame code={
        \path[fill=tcbcolback]
        ([yshift=-1mm,xshift=-1mm]frame.north west)
        arc[start angle=0,end angle=180,radius=1mm]
        ([yshift=-1mm,xshift=1mm]frame.north east)
        arc[start angle=180,end angle=0,radius=1mm];
        \path[left color=tcbcolback!60!black,right color=tcbcolback!60!black,
        middle color=tcbcolback!80!black]
        ([xshift=-2mm]frame.north west) -- ([xshift=2mm]frame.north east)
        [rounded corners=1mm]-- ([xshift=1mm,yshift=-1mm]frame.north east)
        -- (frame.south east) -- (frame.south west)
        -- ([xshift=-1mm,yshift=-1mm]frame.north west)
        [sharp corners]-- cycle;
      },
    },
    fonttitle=\bfseries,
    title={#2},
    #1
  }{def}

  \NewDocumentEnvironment{definition}{O{}O{}}
    {\begin{Definition}{#1}{#2}}{\end{Definition}}

  \newtcolorbox{note}[1][]{%
    enhanced jigsaw,
    colback=gray!20!white,%
    colframe=gray!80!black,
    size=small,
    boxrule=1pt,
    title=\textbf{Note:-},
    halign title=flush center,
    coltitle=black,
    breakable,
    drop shadow=black!50!white,
    attach boxed title to top left={xshift=1cm,yshift=-\tcboxedtitleheight/2,yshifttext=-\tcboxedtitleheight/2},
    minipage boxed title=1.5cm,
    boxed title style={%
      colback=white,
      size=fbox,
      boxrule=1pt,
      boxsep=2pt,
      underlay={%
        \coordinate (dotA) at ($(interior.west) + (-0.5pt,0)$);
        \coordinate (dotB) at ($(interior.east) + (0.5pt,0)$);
        \begin{scope}
          \clip (interior.north west) rectangle ([xshift=3ex]interior.east);
          \filldraw [white, blur shadow={shadow opacity=60, shadow yshift=-.75ex}, rounded corners=2pt] (interior.north west) rectangle (interior.south east);
        \end{scope}
        \begin{scope}[gray!80!black]
          \fill (dotA) circle (2pt);
          \fill (dotB) circle (2pt);
        \end{scope}
      },
    },
    #1,
  }

  \newtcbtheorem{Question}{Question}{enhanced,
    breakable,
    colback=white,
    colframe=myblue!80!black,
    attach boxed title to top left={yshift*=-\tcboxedtitleheight},
    fonttitle=\bfseries,
    title=\textbf{Question:-},
    boxed title size=title,
    boxed title style={%
      sharp corners,
      rounded corners=northwest,
      colback=tcbcolframe,
      boxrule=0pt,
    },
    underlay boxed title={%
      \path[fill=tcbcolframe] (title.south west)--(title.south east)
      to[out=0, in=180] ([xshift=5mm]title.east)--
      (title.center-|frame.east)
      [rounded corners=\kvtcb@arc] |-
      (frame.north) -| cycle;
    },
    #1
  }{def}

  \NewDocumentEnvironment{question}{O{}O{}}
  {\begin{Question}{#1}{#2}}{\end{Question}}

  \newtcolorbox{Solution}{enhanced,
    breakable,
    colback=white,
    colframe=mygreen!80!black,
    attach boxed title to top left={yshift*=-\tcboxedtitleheight},
    title=\textbf{Solution:-},
    boxed title size=title,
    boxed title style={%
      sharp corners,
      rounded corners=northwest,
      colback=tcbcolframe,
      boxrule=0pt,
    },
    underlay boxed title={%
      \path[fill=tcbcolframe] (title.south west)--(title.south east)
      to[out=0, in=180] ([xshift=5mm]title.east)--
      (title.center-|frame.east)
      [rounded corners=\kvtcb@arc] |-
      (frame.north) -| cycle;
    },
  }

  \NewDocumentEnvironment{solution}{O{}O{}}
  {\vspace{-10pt}\begin{Solution}{#1}{#2}}{\end{Solution}}
}
\makeatother

%%%%%%%%%%%%%%%%%%%%%%%%%%%%
%  Edit Proof Environment  %
%%%%%%%%%%%%%%%%%%%%%%%%%%%%

\renewenvironment{proof}[1][\proofname]{\vspace{-10pt}\begin{replacementproof}}{\end{replacementproof}}
\newenvironment{explanation}[1][\proofname]{\vspace{-10pt}\begin{expl}}{\end{expl}}

\theoremstyle{definition}

\newtheorem*{notation}{Notation}
\newtheorem*{previouslyseen}{As previously seen}
\newtheorem*{problem}{Problem}
\newtheorem*{observe}{Observe}
\newtheorem*{property}{Property}
\newtheorem*{intuition}{Intuition}

\setuptoc{toc}{leveldown}

\begin{document}
\vspace{-10pt}
\setcounter{chapter}{7}
\pagenumbering{gobble}
\chapter{Heteroskedasticity and serial correlation. HAC standard errors.}
\vspace{-10pt}
The homoskedasticity and no serial correlation assumption (GM3) can be violated in three ways:
\begin{itemize}
    \item Heteroskedasticity only  (B)- $Var(\epsilon|X)$ is diagonal with unequal elements along the diagonal.
    \item Serial correlation only (C) - $Var(\epsilon|X)$ has non-zero off-diagonal elements, but all diagonal elements are the same.
    \item Heteroskedasticity and serial correlation (D) - $Var(\epsilon|X)$ is a general non-diagonal matrix with unequal elements along the diagonal.
\end{itemize}
\[ A = \sigma^2 \begin{bmatrix}
    1&0&0\\
    0&1&0\\
    0&0&1\\
\end{bmatrix} \quad 
B = \sigma^2 \begin{bmatrix}
    1&0&0\\
    0&2&0\\
    0&0&3\\
\end{bmatrix} \quad 
C = \sigma^2 \begin{bmatrix}
    1&\rho&\rho^2\\
    \rho&1&\rho\\
    \rho^2&\rho&1\\
\end{bmatrix} \quad
D = \sigma^2 \begin{bmatrix}
    1&\rho&\rho^2\\
    \rho&2&\rho\\
    \rho^2&\rho&3\\
\end{bmatrix}
\]
\section{Heteroskedasticity}
Under heteroskedasticity OLS is still consistent and asymptotically normal, although is no longer efficient and has a different asymptotic covariance matrix. Thus the default standard errors will be wrong. Recall the large sample OLS assumptions, now consider the weaker assumptions OLS2' and OLS3'

\begin{minipage}{.45\textwidth}
    \begin{align*}
        &(\text{OLS0}) \quad (y_i, x_i)\text{ is an i.i.d. sequence} \\
        &(\text{OLS1}) \quad E(x_i x_i') \text{ is finite non-singular} \\
        &(\text{OLS2}) \quad E(y_i|x_i) = x_i'\beta \\
        &(\text{OLS3}) \quad \text{Var}(y_i|x_i)= \sigma^2 \\
        &(\text{OLS4}) \quad E\epsilon_i^4 < \infty, \quad E\|x_i\|^4 < \infty
    \end{align*}
\end{minipage}
\begin{minipage}{.55\textwidth}
    \begin{align*}
        &(\text{OLS2'}) \quad E(\epsilon_ix_i)= 0 \\
        &(\text{OLS3'}) \quad \text{Var}(\epsilon_ix_i)=V<\infty \text{ and is non-singular} 
    \end{align*}
\end{minipage}

\begin{theorem}
    Under OLS0,1,2',3',4
    \begin{enumerate}
        \item $\hat \beta_{OLS} \overset{p}{\to} \beta$ (OLS is consistent)
        \item $\sqrt{n}\left(\hat \beta_{OLS} -\beta\right) \overset{d}{\to} N\left(0,(\E(x_ix_i'))^{-1}V (\E(x_ix_i'))^{-1}\right)$
    \end{enumerate}
\end{theorem}

\begin{proof}
    1. We only require OLS0,1,2' for consistency
    \begin{align*}
        \hat\beta_{OLS} &= \beta + (X'X)^{-1}X'\epsilon\\
        &= \beta + \left(\frac{1}{n}\sum_{i=1}^{n} x_i x_i'\right)^{-1} \frac{1}{n}\sum_{i=1}^{n} x_i \epsilon_i\\
        &\overset{p}{\to} \beta + [\E(x_ix_i')]^{-1} \E(\epsilon_i x_i) \\
        &= \beta
    \end{align*}
    2.
    \[\sqrt{n}\left(\hat\beta_{OLS} - \beta\right) = \left(\frac{1}{n}\sum_{i=1}^{n} x_i x_i'\right)^{-1} \frac{1}{\sqrt{n}}\sum_{i=1}^{n} x_i \epsilon_i \]
    
    Using the CLT: 
    \[ \frac{1}{\sqrt{n}}\sum_{i=1}^{n} x_i \epsilon_i \overset{d}{\to} N(0,V) \]
    Using the CMT: 
    \begin{align*}
        \left(\frac{1}{n}\sum_{i=1}^{n} x_i x_i'\right)^{-1} \frac{1}{\sqrt{n}}\sum_{i=1}^{n} x_i \epsilon_i & \overset{d}{\to} [\E(x_ix_i')]^{-1} N(0,V)\\
        \implies \sqrt{n}\left(\hat\beta_{OLS} - \beta\right) & \overset{d}{\to} N\left(0,(\E(x_ix_i'))^{-1}V (\E(x_ix_i'))^{-1}\right)
    \end{align*}
    \end{proof}
When the errors are homoskedastic the variance is as in previous lectures: \[ \E[X'X]^{-1} \E[X'X \epsilon_i^2] \E[X'X]^{-1} = \E[X'X]^{-1} \sigma^2\E[X'X] \E[X'X]^{-1} = \sigma^2\E[X'X]^{-1} \]
The classic covariance matrix estimator can be highly biased if homoskedasticity fails, we now consider how to construct covariance matrix estimators which do not require homoskedasticity.\\
If $\epsilon_i$ were known, we could have estimated V as follows:
\[ \frac{1}{n} \sum_{i=1}^{n}x_ix_i'\hat \epsilon_i ^2 \overset{p}{\to} V\]
Of course $\epsilon_i$ is unknown, but since $\hat \beta_{OLS}$ remains consistent we can use the observed residuals $\hat \epsilon_i = Y_i - x_i'\hat\beta_{OLS}$:
\[ \hat V = \frac{1}{n} \sum_{i=1}^{n}x_ix_i'\hat \epsilon_i ^2 \]
To show this is a consistent estimator:
\begin{align*}
    \hat V &= \frac{1}{n} \sum_{i=1}^{n}x_ix_i'\hat \epsilon_i ^2\\
     &= \frac{1}{n} \sum_{i=1}^{n}x_ix_i' \left(\epsilon_i-x_i'\left( \hat \beta_{OLS} - \beta \right)\right)^2\\
     &=\frac{1}{n} \sum_{i=1}^{n}x_ix_i' \left( \epsilon_i^2 -2\epsilon_i x_i'\left( \hat \beta_{OLS} - \beta \right) + \left(x_i'\left( \hat \beta_{OLS} - \beta \right)\right)^2\right)\\
     &= \frac{1}{n} \sum_{i=1}^{n}x_ix_i' \epsilon_i^2 -\frac{2}{n}\sum_{i=1}^{n}(x_ix_i')\epsilon_i x_i'\left( \hat \beta_{OLS} - \beta \right) + \frac{1}{n} \sum_{i=1}^{n}x_ix_i' \left(x_i'\left( \hat \beta_{OLS} - \beta \right)\right)^2\\
     &\overset{p}{\to} V \quad \text{since $\hat \beta_{OLS}\overset{p}{\to} \beta$}
\end{align*}
\begin{definition}[White's heteroskedasticity robust covaraince matrix]
    \[\widehat{Var}(\hat\beta_{OLS}) = (X'X)^{-1}\left(\sum_{i=1}^{n}x_ix_i'\hat \epsilon_i ^2 \right)(X'X)^{-1} \]
\end{definition}
\begin{note}
Whilst this estimator is consistent, it is biased in finite samples. To see this, suppose the actual covariance matrix of the population regression residuals is given by $\E[\epsilon \epsilon'|X] = \Phi = diag(\phi_i)$. The covariance matrix of the OLS estimator is then \[V=(X'X)^{-1}(X'\Phi X) (X'X)^{-1}\] Denote the i-th column of the residual maker matrix M by $m_i$ then $\hat \epsilon_i = m_i ' \epsilon$. 
\[ \implies \E[\hat \epsilon_i^2] = \E[m_i' \epsilon \epsilon' m_i] = m_i'\Phi m_i \]
Notice that $m_i$ is the i-th column of the identity matrix (denoted as $e_i$) minus the i-th column of the projection matrix $X(X'X)^{-1}X'$ ($p_i$). Hence $m_i=e_i - p_i$ and 
\[ \E[\hat \epsilon_i^2] = (e_i-h_i)'\Phi (e_i-h_i) = \phi_i - 2\phi_i h_{ii}+h_i'\Phi h_i \]
where $h_{ii}$ is the i-th diagonal element of the projection matrix. Because this matrix is symmetric and idempotent, $h_{ii} = h_i'h_i$ so: 
\begin{align*}
    \E\left(\hat V - V\right) &= (X'X)^{-1}(X'\Phi X) (X'X)^{-1} - (X'X)^{-1}(X' \hat \Phi X) (X'X)^{-1}\\
    &= (X'X)^{-1}(X'(\Phi-\hat \Phi) X) (X'X)^{-1}\\
    &= (X'X)^{-1}(X' diag(\phi_i - (\phi_i - 2\phi_i h_{ii}+h_i'\Phi h_i) ) X) (X'X)^{-1}\\
    &= (X'X)^{-1}(X'diag(h_i'(\Phi - 2\phi_iI)h_i)X) (X'X)^{-1}
\end{align*}


Whilst $\hat V$ is biased, here we can see that it is also consistent. Notice that $ \hat \Phi $ is not consistent for $\Phi$, since there are more elements to estimate as the sample gets large. However, $\hat \epsilon_i$ is consistent for $\epsilon_i$. We know \[X'\hat \Phi X = \frac{1}{n}\sum_{i=1}^{n}  x_ix_i'\hat \epsilon_i^2 \] and since plim $\hat \epsilon_i ^2 = \phi_i$ we get plim $X'\hat \Phi X = X' \Phi X$.

In summary, $\hat V$ is biased since $\hat\epsilon^2_i$ is a biased estimate of $\epsilon^2$.
\end{note}

\section{Serial correlation (and heteroskedasticity)}
As with heteroskedasticity, OLS remains consistent and asymptotically normal, but the default standard errors are wrong. This cannot happen if the data are i.i.d. - if OLS0 holds it must be the case that $\Omega = Var(\epsilon|X)$ is diagonal. If the data are dependent, then $\Omega$ is typically no longer diagonal.

\begin{definition}[Strict Stationarity]
    A sequence of random variables $\{Z_t\}^{\infty}_{t=-\infty}$ is strictly stationary if, for any finite nonnegative integer $m$,
    \[
        f_{Z_t, Z_{t+1}, ..., Z_{t+m}}(x_0, x_1, ..., x_m)=  f_{Z_s, Z_{s+1}, ..., Z_{s+m}}(x_0, x_1, ..., x_m)
    \]
    which is to say that the joint distribution , $f$, does not depend on the index, $t$.
\end{definition}
Strict stationarity implies that the (marginal) distribution of $Z_t$ does not vary over time. It also implies that the bivariate distributions of$ (Z_t, Z_{t+1})$ and multivariate distributions of $ (Z_t, ..., Z_{t+m})$ are stable over time.

\begin{theorem}
    If $Z_t$ is i.i.d., then it is strictly stationary
\end{theorem}
\begin{proof}
    Let $F$ denote the joint distribution function, then:
    \begin{align*}
        F(x_{n+1}, ..., x_{n+m}) &= F(x_{n+1})\cdot ... \cdot F(x_{n+m})\\
        &= F(x_{n+k+1})\cdot ... \cdot F(x_{n+k+m})\\
        &= F(x_{n+k+1},...,x_{n+k+m})
    \end{align*}
    Lines 1 and 3 follow from the fact that the joint distribution function of a set of mutually independent variables is equal to the product of their marginal distribution functions. On line 2 we have used the fact that all the terms of the sequence have the same distribution. 
\end{proof}


\begin{definition}[Covarariance stationarity]
    A sequence of random variables $\{Z_t\}^{\infty}_{t=-\infty}$ is covariance (weakly) stationary if just the first two moments do not depend on $t$, e.g.
    \begin{align*}
        \E Z_1 &= \E Z_2 = \dots\\
        Var(Z_1)&=Var(Z_2)= \dots\\
        Cov(Z_1, Z_{1+m}) &= Cov(Z_2, Z_{2+m}) = \dots
    \end{align*}
\end{definition}
A strictly stationary process is covariance-stationary as long as the variance and covariances are finite.

Consider a new set of OLS assumptions:
\begin{align*}
    \text{(SC0)}& \quad \{(y_t, x_t)\}^T_{t=1} \text{ is strictly stationary}\\
    \text{(SC1)}& \quad \{(x_tx_t')\} \text{ satisfies LLN: }\tfrac{1}{T}\sum x_t x_t' \overset{p}{\to} \E(x_t x_t') < \infty \text{, positive definite}\\
    \text{(SC2)}& \quad \{(x_t \epsilon_t)\} \text{ satisfies LLN: }\tfrac{1}{T}\sum x_t \epsilon_t \overset{p}{\to} \E(x_t \epsilon_t) =0\\
    \text{(SC3)}& \quad \{(x_t \epsilon_t)\} \text{ satisfies CLT: }\tfrac{1}{\sqrt{T}}\sum x_t \epsilon_t \overset{d}{\to} N(0,V) \text{, where }
\end{align*}
\[ V= \E(\epsilon_t^2 x_t x_t')+\sum_{t=1}^{\infty}\left(\E(\epsilon_t \epsilon_{t-l} x_t x_{t-l}') + \E(\epsilon_t \epsilon_{t-l} x_{t-l} x_{t}')\right)\]
These assumptions further generalise our GM/OLS conditions, such that if the data were independent, we would have $V=\E(\epsilon_t^2x_tx_t')$ as in OLS3'.

\begin{theorem}
    Under SC0,1,2,3
    \begin{enumerate}
        \item $\hat \beta_{OLS} \overset{p}{\to} \beta$ (OLS is consistent)
        \item $\sqrt{T}\left(\hat \beta_{OLS} -\beta\right) \overset{d}{\to} N\left(0,(\E(x_tx_t'))^{-1}V (\E(x_tx_t'))^{-1}\right)$
    \end{enumerate}
\end{theorem}
The proof is identical to the heteroskedastic case in Theorem 8.2.1.\\

\textbf{Newey-West Method}\\
Under the SC assumptions, the conventional covariance matrix estimators are inconsistent as they do not capture the serial dependence in $x_te_t$. To consistently estimate the covariance matrix, we need a different estimator. The appropriate class of estimators are called Heteroskedasticity and Autocorrelation Consistent (HAC) covariance matrix estimators.\\

Define $V_T$ as follows:

\begin{align*}
    V_T &\equiv Var\left(\frac{1}{\sqrt{T}}\sum_{t=1}^{T}x_t\epsilon_t\right)\\
    &= \E \left[\frac{1}{T}\left(\sum_{t=1}^{T}x_t\epsilon_t\right)\left(\sum_{t=1}^{T}x_t\epsilon_t\right)'\right]\\
    &= \E\left[\frac{1}{T}(x_1\epsilon_1+x_2\epsilon_2+\dots+x_T\epsilon_T)(x_1'\epsilon_1+x_2'\epsilon_2+\dots+x_T'\epsilon_T)'\right]\\
    &= \E\left[\underbrace{\frac{1}{T}\sum_{t=1}^{T}\epsilon_t^2x_tx_t'}_{\text{variance at t}}+\underbrace{\frac{1}{T}\sum_{\ell=1}^{T-1}\sum_{t=\ell+1}^{T}\left(\epsilon_t\epsilon_{t-\ell} x_t x_{t-\ell}'+ \epsilon_t\epsilon_{t-\ell} x_{t-\ell}x_t'\right)}_{\text{all covariances with all other time periods}}\right]\\
    &= \frac{1}{T}\sum_{t=1}^{T}\E[\epsilon_t^2x_tx_t']+\frac{1}{T}\sum_{\ell=1}^{T-1}\sum_{t=\ell+1}^{T}\left(\E(\epsilon_t\epsilon_{t-\ell} x_t x_{t-\ell}')+ \E(\epsilon_t\epsilon_{t-\ell} x_{t-\ell}x_t')\right)\\
    &= \E[\epsilon_t^2x_tx_t']+\sum_{\ell=1}^{T-1}\frac{T-\ell}{T}\left(\E(\epsilon_t\epsilon_{t-\ell} x_t x_{t-\ell}')+ \E(\epsilon_t\epsilon_{t-\ell} x_{t-\ell}x_t')\right)\quad \text{Using SC0}\\
\end{align*}
As $T$ get large, $V_T \approx V$. Since we have $T$ data points, we can only estimate $G<T$ autocovariances of $x_t\epsilon_t$, where $G$ is the truncation lag.  Newey and West propose the following procedure:
\begin{enumerate}
    \item Choose $G$ such that: $G=O(T^{\alpha})$ for $0<\alpha<1/4$
    \item Estimate autocovariances of $x_t\epsilon_t$ of order $\ell$ by \[\hat \Gamma_{\ell} = \frac{1}{T}\sum_{t=\ell+1}^{T}\hat\epsilon_t \hat\epsilon_{t-\ell}x_tx_{t-\ell}'\]
    \item Estimate V by \[ \hat V_{nw} = \hat\Gamma_0+\sum_{\ell=1}^{G}\frac{G+1-\ell}{G+1}\left(\hat\Gamma_{\ell}+\hat\Gamma_{\ell}'\right)\]
\end{enumerate}

If we know a priori that autocovariances are zero in population beyond a certain finite lag $q$,  we can consistently estimate $V$ with \[\hat V = \hat\Gamma_0+\sum_{\ell=1}^{q}\left(\hat\Gamma_{\ell}+\hat\Gamma_{\ell}'\right)\] However in the case where we do not know $q$ (which is potentially infinite), we can use the weighted sum suggested by Newey and West. For example, for $q(n) = 3$ \[\hat V_{NW} = \hat \Gamma_0 + \frac{2}{3}(\hat \Gamma_1 + \hat \Gamma_1 ') + \frac{1}{3}(\hat \Gamma_2 + \hat \Gamma_2 ')\]

 The weighting term ensures $\hat V_{nw}$ is positive semi-definite. We can see the similarities between this and our expression for $V_T$ earlier, giving some intuition for its consistency. 
\begin{align*}
    V_T &= \mathbb{E}[\epsilon_t^2x_tx_t'] \hspace{15pt} +  \sum_{\ell=1}^{T-1} \frac{T-\ell}{T} \hspace{21pt}\Bigg[ \hspace{10pt}\mathbb{E}(\epsilon_t\epsilon_{t-\ell} x_t x_{t-\ell}') \hspace{22pt}+ \hspace{10pt} \mathbb{E}(\epsilon_t\epsilon_{t-\ell} x_{t-\ell}x_t')\hspace{23pt}\Bigg] \\
    \hat V_{nw} &= \frac{1}{T}\sum_{t=1}^{T} \hat\epsilon_t^2 x_tx_t' + \sum_{\ell=1}^{G}\frac{G+1-\ell}{G+1} \left[\frac{1}{T}\sum_{t=\ell+1}^{T}\left(\epsilon_t\epsilon_{t-\ell} x_t x_{t-\ell}'\right) + \frac{1}{T}\sum_{t=\ell+1}^{T}\left(\epsilon_t\epsilon_{t-\ell} x_{t-\ell}x_t'\right)\right]
\end{align*}

Now we can estimate the covariance matrix of $\hat \beta_{OLS}$ as
\[
\frac{1}{T}\left[ \frac{1}{T} \sum_{t=1}^{T}x_tx_t'\right]^{-1}\hat V_{nw}  \left[ \frac{1}{T} \sum_{t=1}^{T}x_tx_t'\right]^{-1}  
\]

\begin{lemma}
    The matrix of sample covariances for any process is positive semi-definite.
\end{lemma}
\begin{proof}
    Let $z_1, \dots, z_T$ be any sequence of $T$ numbers, and let $P$ be a $m\times m$ matrix of sample covariances:
    \[ P=
    \begin{bmatrix}
        \frac{1}{T} \sum_{t=1}^{T}z_t^2 & \frac{1}{T} \sum_{t=2}^{T}z_tz_{t-1} & \ddots & \frac{1}{T}\sum_{t=m+1}^{T}z_tz_{t-m}\\
        \frac{1}{T} \sum_{t=2}^{T}z_tz_{t-1} & \frac{1}{T} \sum_{t=1}^{T}z_t^2 & \ddots & \ddots\\
        \ddots & \ddots & \ddots & \frac{1}{T} \sum_{t=2}^{T}z_tz_{t-1}\\
        \frac{1}{T}\sum_{t=m+1}^{T}z_tz_{t-m} & \ddots & \frac{1}{T} \sum_{t=2}^{T}z_tz_{t-1} &  \frac{1}{T} \sum_{t=1}^{T}z_t^2
    \end{bmatrix}\]
    Consider the $m \times (2T-1)$ matrix:
    \[
    Z= \begin{bmatrix}
        z_1&z_2&\cdots&z_m&\cdots&z_T&0&\cdots&0\\
        0&z_1&z_2&\cdots&z_m&\cdots&z_T&\ddots&\vdots\\
        \vdots & \ddots&\ddots&\ddots&\ddots&\ddots&\ddots&\ddots&0\\
        0&\cdots&0&z_1&z_2&\cdots&\cdots&\cdots&z_T
    \end{bmatrix}    
    \]
    \begin{align*}
         \frac{1}{T}ZZ' &= \frac{1}{T}
         \underbrace{\begin{bmatrix}
            z_1&z_2&\cdots&z_m&\cdots&z_T&0&\cdots&0\\
            0&z_1&z_2&\cdots&z_m&\cdots&z_T&\ddots&\vdots\\
            \vdots & \ddots&\ddots&\ddots&\ddots&\ddots&\ddots&\ddots&0\\
            0&\cdots&0&z_1&z_2&\cdots&\cdots&\cdots&z_T
        \end{bmatrix}}_{m\times(2T-1)}
        \underbrace{\begin{bmatrix}
            z_1 & 0 & \cdots & 0 \\
            z_2 & z_1 & \cdots & \vdots \\
            \vdots & z_2 & \ddots & 0 \\
            z_m & \vdots & \ddots & z_1 \\
            \vdots & z_m & \ddots & z_2 \\
            z_T & \vdots & \ddots & \vdots \\
            0 & z_T & \cdots & \vdots \\
            \vdots & \ddots & \ddots & \vdots \\
            0 & \cdots & 0 & z_T \\
        \end{bmatrix}}_{(2T-1)\times m}\\
        &= \frac{1}{T}\begin{bmatrix}
             \sum_{t=1}^{T}z_t^2 &  \sum_{t=2}^{T}z_tz_{t-1} & \ddots & \sum_{t=m+1}^{T}z_tz_{t-m}\\
             \sum_{t=2}^{T}z_tz_{t-1} &  \sum_{t=1}^{T}z_t^2 & \ddots & \ddots\\
            \ddots & \ddots & \ddots &  \sum_{t=2}^{T}z_tz_{t-1}\\
            \sum_{t=m+1}^{T}z_tz_{t-m} & \ddots &  \sum_{t=2}^{T}z_tz_{t-1} &   \sum_{t=1}^{T}z_t^2
        \end{bmatrix}
        =P_{m\times m}
    \end{align*}
Thus P is p.s.d. since for any vector $v, v'ZZ'v=u'u=\sum_{i=1}^{m}u_i^2\geq 0$
\end{proof}



\begin{theorem}
    $\hat V_{nw}$ is positive semi-definite
\end{theorem}
\begin{proof}
    Let $c$ be any deterministic $k$-dimensional vector, we aim to show $c'\hat V_{nw} c \geq 0$. Consider the $G+1$ matrix
\[
    P = 
    \begin{bmatrix}
        c'\hat\Gamma_0c &  c'\hat\Gamma_1c& \ddots & c'\hat\Gamma_Gc\\
        c'\hat\Gamma_1c &  c'\hat\Gamma_0c& \ddots & \ddots \\
        \ddots & \ddots & \ddots & c'\hat\Gamma_1c\\
        c'\hat\Gamma_Gc & \ddots &c'\hat\Gamma_1c & c'\hat\Gamma_0c
    \end{bmatrix}
\]   
If $i$ is a $G+1$-dimensional vector of ones, then we have
\begin{align*}
    i'Pi&= 
    \begin{bmatrix}
        1 & 1 & \dots&1
    \end{bmatrix}
    \begin{bmatrix}
        c'\hat\Gamma_0c &  c'\hat\Gamma_1c& \ddots & c'\hat\Gamma_Gc\\
        c'\hat\Gamma'_1c &  c'\hat\Gamma_0c& \ddots & \ddots \\
        \ddots & \ddots & \ddots & c'\hat\Gamma_1c\\
        c'\hat\Gamma'_Gc & \ddots &c'\hat\Gamma'_1c & c'\hat\Gamma_0c
    \end{bmatrix}
    \begin{bmatrix}
        1 \\
        1 \\
        \vdots\\
        1
    \end{bmatrix}\\
    &= \begin{bmatrix}
        1 & 1 & \dots&1
    \end{bmatrix}
    \begin{bmatrix}
        \sum_{{\ell}=0}^{G}c' \hat\Gamma_{\ell} c\\
        \sum_{{\ell}=1}^{1}c' \hat\Gamma'_{\ell} c+ \sum_{{\ell}=0}^{G-1}c' \hat\Gamma_{\ell} c\\
        \vdots\\
        \sum_{{\ell}=0}^{G}c' \hat\Gamma'_{\ell} c
    \end{bmatrix} \quad \text{$m$-th row = $\sum_{{\ell}=1}^{m}c' \hat\Gamma'_{\ell} c+ \sum_{{\ell}=0}^{G-m}c' \hat\Gamma_{\ell} c$}\\
    &= \sum_{{\ell}=0}^{G}c' \hat\Gamma_{\ell} c + \sum_{{\ell}=1}^{1}c' \hat\Gamma'_{\ell} c+ \sum_{{\ell}=0}^{G-1}c' \hat\Gamma_{\ell} c + \dots + \sum_{{\ell}=0}^{G}c' \hat\Gamma'_{\ell} c \\
    &= (G+1)c' \hat\Gamma_0 c + G(c' \hat\Gamma'_1c + c' \hat\Gamma_1c) + (G-1)(c' \hat\Gamma'_2c + c' \hat\Gamma_2c) + \dots\\
    &= (G+1)c' \hat\Gamma_0 c + \sum_{{\ell}=1}^{G}(G+1-{\ell})(c' \hat\Gamma'_{\ell}c + c' \hat\Gamma_{\ell}c)\\
    \implies \frac{1}{G+1}i'Pi &= c' \hat\Gamma_0 c + \sum_{{\ell}=1}^{G}\frac{G+1-{\ell}}{G+1}(c' \hat\Gamma'_{\ell}c + c' \hat\Gamma_{\ell}c)\\
    &= c'\hat V_{nw}c
\end{align*}
Hence, it is sufficient to show that P is positive semi-definite. However, P is the matrix of sample covariances of the process $z_t=c'x_t\hat\epsilon_t$ with autocovariances:
\[\E[c'x_t\epsilon_t \epsilon_{t-j} x_{t-j}'c]=c'\E[\epsilon_t \epsilon_{t-j}x_t x_{t-j}']c= c'\Gamma_jc \quad \forall j\in\Z\]
The matrix of sample covariances for any process is positive semi-definite, thus $\hat V_{nw}$ is p.s.d.
\end{proof}
\begin{note}
    The population covariance matrix is always positive semi-definite, so it's desirable for its estimate to also be positive semi-definite. Thus in a time series context we define sample covariances as: 
    \[
    \frac{1}{T} \sum_{t=|i-j|+1}^{T} z_tz_{t-|i-j|}\quad \text{rather than as } \frac{1}{T-|i-j|} \sum_{t=|i-j|+1}^{T} z_tz_{t-|i-j|}
    \]
    Even though the former is biased and the latter unbiased, had we used the latter we might get an estimate that is not positive semi-definite.
\end{note}
\end{document}