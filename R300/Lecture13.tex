\documentclass[DIV=14,titlepage=false]{scrreprt}
\usepackage{parskip}
%%%%%%%%%%%%%%%%%%%%%%%%%%%%%%%%%%%%%%%%%%%%%%%%%%%%%%%%%%%%%%%%%%%%%%%%%%%%%%%
%                                Basic Packages                               %
%%%%%%%%%%%%%%%%%%%%%%%%%%%%%%%%%%%%%%%%%%%%%%%%%%%%%%%%%%%%%%%%%%%%%%%%%%%%%%%
% Gives us multiple colors.
\usepackage[usenames,dvipsnames,pdftex]{xcolor}
% Lets us style link colors.
\usepackage{hyperref}
% Lets us import images and graphics.
\usepackage{graphicx}
% Lets us use figures in floating environments.
\usepackage{float}
% Lets us create multiple columns.
\usepackage{multicol}
% Gives us better math syntax.
\usepackage{amsmath,amsfonts,mathtools,amsthm,amssymb}
% Lets us strikethrough text.
\usepackage{cancel}
% Lets us edit the caption of a figure.
\usepackage{caption}
% Lets us import pdf directly in our tex code.
\usepackage{pdfpages}
% Lets us do algorithm stuff.
\usepackage[ruled,vlined,linesnumbered]{algorithm2e}
% Gets rid of some errors.
\usepackage{scrhack}
\def\class{article}
\usepackage{geometry}
\geometry{margin=0.9in}
%%%%%%%%%%%%%%%%%%%%%%%%%%%%%%%%%%%%%%%%%%%%%%%%%%%%%%%%%%%%%%%%%%%%%%%%%%%%%%%
%                                Basic Settings                               %
%%%%%%%%%%%%%%%%%%%%%%%%%%%%%%%%%%%%%%%%%%%%%%%%%%%%%%%%%%%%%%%%%%%%%%%%%%%%%%%

%%%%%%%%%%%%%
%  Symbols  %
%%%%%%%%%%%%%

\let\implies\Rightarrow
\let\impliedby\Leftarrow
\let\iff\Leftrightarrow
\let\epsilon\varepsilon

%%%%%%%%%%%%
%  Tables  %
%%%%%%%%%%%%

\setlength{\tabcolsep}{5pt}
\renewcommand\arraystretch{1.5}

%%%%%%%%%%%%%%%%%%%%%%%
%  Center Title Page  %
%%%%%%%%%%%%%%%%%%%%%%%

\usepackage{titling}
\renewcommand\maketitlehooka{\null\mbox{}\vfill}
\renewcommand\maketitlehookd{\vfill\null}

%%%%%%%%%%%%%%%%%%%%%%%%%%%%%%%%%%%%%%%%%%%%%%%%%%%%%%%
%  Create a grey background in the middle of the PDF  %
%%%%%%%%%%%%%%%%%%%%%%%%%%%%%%%%%%%%%%%%%%%%%%%%%%%%%%%

\usepackage{eso-pic}
\newcommand\definegraybackground{
  \definecolor{reallylightgray}{HTML}{FAFAFA}
  \AddToShipoutPicture{
    \ifthenelse{\isodd{\thepage}}{
      \AtPageLowerLeft{
        \put(\LenToUnit{\dimexpr\paperwidth-222pt},0){
          \color{reallylightgray}\rule{222pt}{297mm}
        }
      }
    }
    {
      \AtPageLowerLeft{
        \color{reallylightgray}\rule{222pt}{297mm}
      }
    }
  }
}

%%%%%%%%%%%%%%%%%%%%%%%%
%  Modify Links Color  %
%%%%%%%%%%%%%%%%%%%%%%%%

\hypersetup{
  % Enable highlighting links.
  colorlinks,
  % Change the color of links to blue.
  linkcolor={black},
  % Change the color of citations to black.
  citecolor={black},
  % Change the color of url's to blue with some black.
  urlcolor=blue
}

%%%%%%%%%%%%%%%%%%
% Fix WrapFigure %
%%%%%%%%%%%%%%%%%%

\newcommand{\wrapfill}{\par\ifnum\value{WF@wrappedlines}>0
    \parskip=0pt
    \addtocounter{WF@wrappedlines}{-1}%
    \null\vspace{\arabic{WF@wrappedlines}\baselineskip}%
    \WFclear
\fi}

%%%%%%%%%%%%%%%%%
% Multi Columns %
%%%%%%%%%%%%%%%%%

\let\multicolmulticols\multicols
\let\endmulticolmulticols\endmulticols

\RenewDocumentEnvironment{multicols}{mO{}}
{%
  \ifnum#1=1
    #2%
  \else % More than 1 column
    \multicolmulticols{#1}[#2]
  \fi
}
{%
  \ifnum#1=1
\else % More than 1 column
  \endmulticolmulticols
\fi
}

\newlength{\thickarrayrulewidth}
\setlength{\thickarrayrulewidth}{5\arrayrulewidth}

%%%%%%%%%%%%%%%%%%%%
%  Import Figures  %
%%%%%%%%%%%%%%%%%%%%

\usepackage{import}
\pdfminorversion=7

% EXAMPLE:
% 1. \incfig{limit-graph}
% 2. \incfig[0.4]{limit-graph}
% Parameters:
% 1. The figure name. It should be located in figures/NAME.tex_pdf.
% 2. (Optional) The width of the figure. Example: 0.5, 0.35.
\newcommand\incfig[2][1]{%
  \def\svgwidth{#1\columnwidth}
  \import{./figures/}{#2.pdf_tex}
}

\begingroup\expandafter\expandafter\expandafter\endgroup
\expandafter\ifx\csname pdfsuppresswarningpagegroup\endcsname\relax
\else
  \pdfsuppresswarningpagegroup=1\relax
\fi

%%%%%%%%%%%%%
%  Correct  %
%%%%%%%%%%%%%

% EXAMPLE:
% 1. \correct{INCORRECT}{CORRECT}
% Parameters:
% 1. The incorrect statement.
% 2. The correct statement.
\definecolor{correct}{HTML}{009900}
\newcommand\correct[2]{{\color{red}{#1 }}\ensuremath{\to}{\color{correct}{ #2}}}



\newcommand{\R}{\mathbb{R}}
\newcommand{\Z}{\mathbb{Z}}
\newcommand{\E}{\mathbb{E}}
\newcommand{\B}{\ensuremath{\mathcal{B}}}
\newcommand{\X}{\ensuremath{\mathcal{X}}}
\newcommand{\Y}{\ensuremath{\mathcal{Y}}}
\newcommand{\mA}{\ensuremath{\mathbf{A}}}
\newcommand{\mB}{\ensuremath{\mathbf{B}}}
\newcommand{\mC}{\ensuremath{\mathbf{C}}}
\newcommand{\mD}{\ensuremath{\mathbf{D}}}
\newcommand{\mX}{\ensuremath{\mathbf{X}}}
\newcommand{\mY}{\ensuremath{\mathbf{Y}}}
\newcommand{\mx}{\ensuremath{\mathbf{x}}}
\newcommand{\my}{\ensuremath{\mathbf{y}}}
\newcommand{\mI}{\ensuremath{\mathbf{I}}}
\newcommand{\mi}{\ensuremath{\mathbf{\iota}}}
\newcommand{\mmu}{\ensuremath{\mathbf{\mu}}}
\newcommand{\mc}{\ensuremath{\mathbf{c}}}
\newcommand{\mSigma}{\ensuremath{\mathbf{\Sigma}}}
\newcommand{\mzero}{\ensuremath{\mathbf{0}}}
\newcommand{\independent}{\perp\!\!\!\!\perp} 
\setlength{\parindent}{0pt}
%%%%%%%%%%%%%%%%%%%%%%%%%%%%%%%%%%%%%%%%%%%%%%%%%%%%%%%%%%%%%%%%%%%%%%%%%%%%%%%
%                                 Environments                                %
%%%%%%%%%%%%%%%%%%%%%%%%%%%%%%%%%%%%%%%%%%%%%%%%%%%%%%%%%%%%%%%%%%%%%%%%%%%%%%%

\usepackage{varwidth}
\usepackage{thmtools}
\usepackage[most,many,breakable]{tcolorbox}

\tcbuselibrary{theorems,skins,hooks}
\usetikzlibrary{arrows,calc,shadows.blur}

%%%%%%%%%%%%%%%%%%%
%  Define Colors  %
%%%%%%%%%%%%%%%%%%%

\definecolor{myblue}{RGB}{45, 111, 177}
\definecolor{mygreen}{RGB}{56, 140, 70}
\definecolor{myred}{RGB}{199, 68, 64}
\definecolor{mypurple}{RGB}{197, 92, 212}

\definecolor{definition}{HTML}{228b22}
\definecolor{theorem}{HTML}{00007B}
\definecolor{example}{HTML}{2A7F7F}
\definecolor{definition}{HTML}{228b22}
\definecolor{prop}{HTML}{191971}
\definecolor{lemma}{HTML}{983b0f}
\definecolor{exercise}{HTML}{88D6D1}

\colorlet{definition}{mygreen!85!black}
\colorlet{claim}{mygreen!85!black}
\colorlet{corollary}{mypurple!85!black}
\colorlet{proof}{theorem}

%%%%%%%%%%%%%%%%%%%%%%
%  Helpful Commands  %
%%%%%%%%%%%%%%%%%%%%%%

% EXAMPLE:
% 1. \createnewtheoremstyle{thmdefinitionbox}{}{}
% 2. \createnewtheoremstyle{thmtheorembox}{}{}
% 3. \createnewtheoremstyle{thmproofbox}{qed=\qedsymbol}{
%       rightline=false, topline=false, bottomline=false
%    }
% Parameters:
% 1. Theorem name.
% 2. Any extra parameters to pass directly to declaretheoremstyle.
% 3. Any extra parameters to pass directly to mdframed.
\newcommand\createnewtheoremstyle[3]{
  \declaretheoremstyle[
  headfont=\bfseries\sffamily, bodyfont=\normalfont, #2,
  mdframed={
    #3,
  },
  ]{#1}
}

% EXAMPLE:
% 1. \createnewcoloredtheoremstyle{thmdefinitionbox}{definition}{}{}
% 2. \createnewcoloredtheoremstyle{thmexamplebox}{example}{}{
%       rightline=true, leftline=true, topline=true, bottomline=true
%     }
% 3. \createnewcoloredtheoremstyle{thmproofbox}{proof}{qed=\qedsymbol}{backgroundcolor=white}
% Parameters:
% 1. Theorem name.
% 2. Color of theorem.
% 3. Any extra parameters to pass directly to declaretheoremstyle.
% 4. Any extra parameters to pass directly to mdframed.
\newcommand\createnewcoloredtheoremstyle[4]{
  \declaretheoremstyle[
  headfont=\bfseries\sffamily\color{#2}, bodyfont=\normalfont, #3,
  mdframed={
    linewidth=2pt,
    rightline=false, leftline=true, topline=false, bottomline=false,
    linecolor=#2, backgroundcolor=#2!5, #4,
  },
  ]{#1}
}

%%%%%%%%%%%%%%%%%%%%%%%%%%%%%%%%%%%
%  Create the Environment Styles  %
%%%%%%%%%%%%%%%%%%%%%%%%%%%%%%%%%%%

\makeatletter
\@ifclasswith\class{nocolor}{
  % Environments without color.

  \createnewtheoremstyle{thmdefinitionbox}{}{}
  \createnewtheoremstyle{thmtheorembox}{}{}
  \createnewtheoremstyle{thmexamplebox}{}{}
  \createnewtheoremstyle{thmclaimbox}{}{}
  \createnewtheoremstyle{thmcorollarybox}{}{}
  \createnewtheoremstyle{thmpropbox}{}{}
  \createnewtheoremstyle{thmlemmabox}{}{}
  \createnewtheoremstyle{thmexercisebox}{}{}
  \createnewtheoremstyle{thmdefinitionbox}{}{}
  \createnewtheoremstyle{thmquestionbox}{}{}
  \createnewtheoremstyle{thmsolutionbox}{}{}

  \createnewtheoremstyle{thmproofbox}{qed=\qedsymbol}{}
  \createnewtheoremstyle{thmexplanationbox}{}{}
}{
  % Environments with color.

  \createnewcoloredtheoremstyle{thmdefinitionbox}{definition}{}{}
  \createnewcoloredtheoremstyle{thmtheorembox}{theorem}{}{}
  \createnewcoloredtheoremstyle{thmexamplebox}{example}{}{
    rightline=true, leftline=true, topline=true, bottomline=true
  }
  \createnewcoloredtheoremstyle{thmclaimbox}{claim}{}{}
  \createnewcoloredtheoremstyle{thmcorollarybox}{corollary}{}{}
  \createnewcoloredtheoremstyle{thmpropbox}{prop}{}{}
  \createnewcoloredtheoremstyle{thmlemmabox}{lemma}{}{}
  \createnewcoloredtheoremstyle{thmexercisebox}{exercise}{}{}

  \createnewcoloredtheoremstyle{thmproofbox}{proof}{qed=\qedsymbol}{backgroundcolor=white}
  \createnewcoloredtheoremstyle{thmexplanationbox}{example}{qed=\qedsymbol}{backgroundcolor=white}
}
\makeatother

%%%%%%%%%%%%%%%%%%%%%%%%%%%%%
%  Create the Environments  %
%%%%%%%%%%%%%%%%%%%%%%%%%%%%%

\declaretheorem[numberwithin=section, style=thmtheorembox,     name=Theorem]{theorem}
\declaretheorem[numbered=no,          style=thmexamplebox,     name=Example]{example}
\declaretheorem[numberwithin=section, style=thmclaimbox,       name=Claim]{claim}
\declaretheorem[numberwithin=section, style=thmcorollarybox,   name=Corollary]{corollary}
\declaretheorem[numberwithin=section, style=thmpropbox,        name=Proposition]{prop}
\declaretheorem[numberwithin=section, style=thmlemmabox,       name=Lemma]{lemma}
\declaretheorem[numberwithin=section, style=thmexercisebox,    name=Exercise]{exercise}
\declaretheorem[numbered=no,          style=thmproofbox,       name=Proof]{replacementproof}
\declaretheorem[numbered=no,          style=thmexplanationbox, name=Proof]{expl}

\makeatletter
\@ifclasswith\class{nocolor}{
  % Environments without color.

  \newtheorem*{note}{Note}

  \declaretheorem[numberwithin=section, style=thmdefinitionbox, name=Definition]{definition}
  \declaretheorem[numberwithin=section, style=thmquestionbox,   name=Question]{question}
  \declaretheorem[numberwithin=section, style=thmsolutionbox,   name=Solution]{solution}
}{
  % Environments with color.

  \newtcbtheorem[number within=section]{Definition}{Definition}{
    enhanced,
    before skip=2mm,
    after skip=2mm,
    colback=red!5,
    colframe=red!80!black,
    colbacktitle=red!75!black,
    boxrule=0.5mm,
    attach boxed title to top left={
      xshift=1cm,
      yshift*=1mm-\tcboxedtitleheight
    },
    varwidth boxed title*=-3cm,
    boxed title style={
      interior engine=empty,
      frame code={
        \path[fill=tcbcolback]
        ([yshift=-1mm,xshift=-1mm]frame.north west)
        arc[start angle=0,end angle=180,radius=1mm]
        ([yshift=-1mm,xshift=1mm]frame.north east)
        arc[start angle=180,end angle=0,radius=1mm];
        \path[left color=tcbcolback!60!black,right color=tcbcolback!60!black,
        middle color=tcbcolback!80!black]
        ([xshift=-2mm]frame.north west) -- ([xshift=2mm]frame.north east)
        [rounded corners=1mm]-- ([xshift=1mm,yshift=-1mm]frame.north east)
        -- (frame.south east) -- (frame.south west)
        -- ([xshift=-1mm,yshift=-1mm]frame.north west)
        [sharp corners]-- cycle;
      },
    },
    fonttitle=\bfseries,
    title={#2},
    #1
  }{def}

  \NewDocumentEnvironment{definition}{O{}O{}}
    {\begin{Definition}{#1}{#2}}{\end{Definition}}

  \newtcolorbox{note}[1][]{%
    enhanced jigsaw,
    colback=gray!20!white,%
    colframe=gray!80!black,
    size=small,
    boxrule=1pt,
    title=\textbf{Note:-},
    halign title=flush center,
    coltitle=black,
    breakable,
    drop shadow=black!50!white,
    attach boxed title to top left={xshift=1cm,yshift=-\tcboxedtitleheight/2,yshifttext=-\tcboxedtitleheight/2},
    minipage boxed title=1.5cm,
    boxed title style={%
      colback=white,
      size=fbox,
      boxrule=1pt,
      boxsep=2pt,
      underlay={%
        \coordinate (dotA) at ($(interior.west) + (-0.5pt,0)$);
        \coordinate (dotB) at ($(interior.east) + (0.5pt,0)$);
        \begin{scope}
          \clip (interior.north west) rectangle ([xshift=3ex]interior.east);
          \filldraw [white, blur shadow={shadow opacity=60, shadow yshift=-.75ex}, rounded corners=2pt] (interior.north west) rectangle (interior.south east);
        \end{scope}
        \begin{scope}[gray!80!black]
          \fill (dotA) circle (2pt);
          \fill (dotB) circle (2pt);
        \end{scope}
      },
    },
    #1,
  }

  \newtcbtheorem{Question}{Question}{enhanced,
    breakable,
    colback=white,
    colframe=myblue!80!black,
    attach boxed title to top left={yshift*=-\tcboxedtitleheight},
    fonttitle=\bfseries,
    title=\textbf{Question:-},
    boxed title size=title,
    boxed title style={%
      sharp corners,
      rounded corners=northwest,
      colback=tcbcolframe,
      boxrule=0pt,
    },
    underlay boxed title={%
      \path[fill=tcbcolframe] (title.south west)--(title.south east)
      to[out=0, in=180] ([xshift=5mm]title.east)--
      (title.center-|frame.east)
      [rounded corners=\kvtcb@arc] |-
      (frame.north) -| cycle;
    },
    #1
  }{def}

  \NewDocumentEnvironment{question}{O{}O{}}
  {\begin{Question}{#1}{#2}}{\end{Question}}

  \newtcolorbox{Solution}{enhanced,
    breakable,
    colback=white,
    colframe=mygreen!80!black,
    attach boxed title to top left={yshift*=-\tcboxedtitleheight},
    title=\textbf{Solution:-},
    boxed title size=title,
    boxed title style={%
      sharp corners,
      rounded corners=northwest,
      colback=tcbcolframe,
      boxrule=0pt,
    },
    underlay boxed title={%
      \path[fill=tcbcolframe] (title.south west)--(title.south east)
      to[out=0, in=180] ([xshift=5mm]title.east)--
      (title.center-|frame.east)
      [rounded corners=\kvtcb@arc] |-
      (frame.north) -| cycle;
    },
  }

  \NewDocumentEnvironment{solution}{O{}O{}}
  {\vspace{-10pt}\begin{Solution}{#1}{#2}}{\end{Solution}}
}
\makeatother

%%%%%%%%%%%%%%%%%%%%%%%%%%%%
%  Edit Proof Environment  %
%%%%%%%%%%%%%%%%%%%%%%%%%%%%

\renewenvironment{proof}[1][\proofname]{\vspace{-10pt}\begin{replacementproof}}{\end{replacementproof}}
\newenvironment{explanation}[1][\proofname]{\vspace{-10pt}\begin{expl}}{\end{expl}}

\theoremstyle{definition}

\newtheorem*{notation}{Notation}
\newtheorem*{previouslyseen}{As previously seen}
\newtheorem*{problem}{Problem}
\newtheorem*{observe}{Observe}
\newtheorem*{property}{Property}
\newtheorem*{intuition}{Intuition}



\title{%
R300 Econometrics}
\author{Metrics Enjoyers}
\date{Michaelmas Term, 2023-2024}

\setuptoc{toc}{leveldown}
\setcounter{chapter}{12}

\begin{document}
\pagenumbering{gobble}
\chapter{Errors in variables. Endogeneity. IV}

\subsection{Classical measurement error}
Suppose we observe noisy versions of the variables we would like to observe.
We obtain data \(y_i\) and \(x_i\) for \(i=1,...,n\) while the true values are \(y_i^*\) and \(x_i^*\).
Also assume:
\[x_i=x_i^*+\nu_i\]
\[y_i=y_i^*+\eta_i\]
where:
\[E(\nu_i)=E(\eta_i)=0\]
\[E(x_i^*\nu_i)=0, E(y_i^*\eta_i)=0\]
\[E(x_i^*\eta_i)=0, E(y_i^*\nu_i)=0\]
\[E(\nu_i\eta_i)=0\]

Given that \(E(y_i^*|x_i^*)=x_i^*\beta\), if we proceeded as if there were no measurement error we would estimate the following by OLS:

\[\hat\beta_{OLS}=(X'X)^{-1}X'Y\]
\[=(\frac{1}{n}\sum_{i=1}^nx_ix_i')^{-1}\frac{1}{n}\sum_{i=1}^nx_iy_i'\]
\[=(\frac{1}{n}\sum_{i=1}^n(x_i^*+\nu_i)(x_i^*+\nu_i)')^{-1}\frac{1}{n}\sum_{i=1}^n(x_i^*+\nu_i)(y_i^*+\eta_i)'\]
\[
=\left(\frac{1}{n}\sum_{i=1}^n \underset{\downarrow p}{x_i^*x_i'^{*}} + \underset{\downarrow p}{x_i^*\nu_i'} + \underset{\downarrow p}{\nu_ix_i'^{*}} + \underset{\downarrow p}{\nu_i\nu_i'}\right)^{-1} \frac{1}{n}\sum_{i=1}^n \underset{\downarrow p}{x_i^*y_i'^*} + \underset{\downarrow p}{x_i^*\eta_i'} + \underset{\downarrow p}{y_i^*\nu_i'} + \underset{\downarrow p}{\nu_i\eta_i'}
\]
\[\qquad \qquad E(x_i^*x_i'^*)\, E(x_i^*\nu_i')\, E(\nu_ix_i'^*)\, E(\nu_i\nu_i') \quad \quad E(x_i^*y_i'^*)\,E(x_i^*\eta_i')\, E(y_i^*\nu_i')\, E(\nu_i\eta_i')\]

\begin{lemma}
    Suppose we have sequences of random variables \(X_n\) and \(Y_n\) such that \(X_n\overset{p}{\to}X\) and \(Y_n\overset{p}{\to}Y\). Then \(X_nY_n\overset{p}{\to}XY\).

    Proof in Appendix
\end{lemma}
\vspace{5mm}
Thus as plim of a sum is the sum of plims, and \(1/x\) is a continuous function, we can use the CMT to argue:
\[\xrightarrow{p} (Ex_i^*x_i'^*+E\nu_i\nu_i')^{-1}(E(x_i^*(x_i'^*\beta+\epsilon_i)))\]
\[=(Ex_i^*x_i'^*+E\nu_i\nu_i')^{-1}(Ex_i^*x_i'^*)\beta\neq\beta\]

We have here an asymmetric bias. In the multiple regression case, the bias will depend on the interaction between explanatory variables.

In the univariate case the above reduces to :
\[\hat\beta_{OLS}\xrightarrow{p}\frac{\sigma^2_{x^*}}{\sigma^2_{x^*}+\sigma^2_{\nu}}\beta\]
This represents an attenuation bias of \(\hat\beta_{OLS}\) toward zero.

\subsection{Endogeneity with Errors}
Any measurement error in the dependent variable is subsumed in the error term as follows:
\[y_i=y_i^*+\eta_i=x_i^*\beta+(\epsilon_i+\eta_i)\]
The new error term \((\epsilon_i+\eta_i)\) creates no problem for estimation as \(\eta_i\) is uncorrelated with \(x_i^*\). However if there were measurement errors in \(x_i^*\) then we have:
\[y_i=x_i\beta+\underbrace{(\epsilon_i+\eta_i-\nu_i'\beta)}_{u_i}\]
In this case, \(u_i\) is correlated with \(x_i\)
\[E(x_iu_i')=E(x_iu_i)=E(x_i(\epsilon_i+\eta_i-\nu_i'\beta))=E(x_i\epsilon_i)+E(x_i\eta_i)-E(x_i\nu_i'\beta_i)\]
\[=-E(x_i\nu_i')\beta=-E(x_i^*+\nu_i)\nu_i'\beta\]
\[=\cancel{E(x_i^*\nu_i')\beta}+E(\nu_i\nu_i')\beta\neq\vec{0}\]

Thus as error term is correlated with \(x_i\), (OLS2') \(Ex_iu_i=0\) does not hold, and OLS is inconsistent.
Two solutions:

\underline{\smash{Solution 1}}:\\
If we can estimate \(E(\nu_i\nu_i')\) we can undo the error in estimation by using the following:
\[E[x_ix_i']=E[x_i^*x_i'^*]+E[\nu_i\nu_i']\]
But this is not usually possible.
\\ \underline{\smash{Solution 2}}:\\
Suppose we get another independent measure of \(x*\) such that
\[w_i=x_i^*+\tau_i\]
where \(\tau_i\) is uncorrelated with any of \(y_i^*,x_i^*,\eta_i,\nu_i\).
\[E[w_ix_i']=E[(x_i^*+\tau_i)(x_i^*+\nu_i)']=E[x_i^*x_i'^*]\]
\[E[w_iy_i]= E[(x_i^*+\tau_i)(y_i^*+\eta_i)]=E[x_i^*y_i^*]\]
Then if \(E[w_ix_i']\) is invertible:
\[E[w_ix_i']^{-1}E[w_iy_i]=E[x_i^*x_i'^*]^{-1}E[x_i^*y_i^*]=[x_i^*x_i'^*]^{-1}E[x_i^*x_i'^*]\beta=\beta\]
So \(\hat\beta_{IV} = (W'X)^{-1}W'Y\) is consistent for \(\beta\).

This can also be derived directly by multiplying \(w_i\) to the estimating equation and taking expectations:
\[w_iy_i=w_ix_i'\beta+w_ie_i\]
\[E[w_iy_i]=E[w_ix_i']\beta+\cancel{E[(x_i^*+\tau_i)(\epsilon_i+\eta_i-\nu_i'\beta)]}\]
\[\beta=E[w_ix_i']^{-1}E[w_iy_i]\]
\[\implies \hat\beta_{IV}=(\sum_{i=1}^nw_ix_i')^{-1}\sum_{i=1}^nw_iy_i=(W'X)^{-1}W'Y\]
where
\[w=\begin{pmatrix}
    w_1'\\
    \vdots\\
    w_n'
\end{pmatrix}, X=\begin{pmatrix}
    x_1'\\
    \vdots\\
    x_n'
\end{pmatrix}\]

\section{Endogeneity}

Consider the following model
\[y_i=x_i'\beta+\epsilon_i\]
where \(E(\epsilon_i|x_i)\neq0\) in violation of OLS2 and OLS2'.
To distinguish this from the \textit{regression}, we call the above equation a \underline{\smash{structural equation}} and \(\beta\) a \underline{\smash{structural parameter}}.
A structural equation represents a \underline{causal} link, rather than just an empirical association.

When \(E(\epsilon_i|x_i)\neq0\), we say that \(x_i\) is endogenous. When this occurs, usually this is only by a few components of \(x_i\) being correlated with \(\epsilon_i\). The components causing this are referred to as endogenous and the rest exogenous.
We then can partition \(x_i\) into the exogenous part \(x_{1i}\) and the endogenous part \(x_{2i}\) by rearranging.

The endogeneity problem may not only be caused through measurement error, but also joint determination, reverse causality or omitted variable bias.

\subsection{Joint Determination: Supply and Demand}
Consider the following model where \(q_i\) and \(p_i\) are determined jointly by the demand equation
\[\text{Demand: }q_i=-\beta_dp_i+\epsilon_{di}\]
\[\text{Supply: }q_i=-\beta_sp_i+\epsilon_{si}\]

In matrix notation:
\[\begin{pmatrix} 1 & \beta_d \\ 1 & \beta_s \end{pmatrix} \begin{pmatrix} q_i \\ p_i \end{pmatrix} = \begin{pmatrix} \epsilon_{di} \\ \epsilon_{si} \end{pmatrix}\]
\[\begin{pmatrix} q_i \\ p_i \end{pmatrix} = \frac{1}{-\beta_s-\beta_d} \begin{pmatrix} -\beta_s & -\beta_d \\ -1 & 1 \end{pmatrix} \begin{pmatrix} \epsilon_{di} \\ \epsilon_{si} \end{pmatrix}\]
\[=\begin{pmatrix} (\beta_s\epsilon_{di}+\beta_d\epsilon_{si})/(\beta_s+\beta_d) \\ (\epsilon_{si}-\epsilon_{di})/(\beta_s+\beta_d) \end{pmatrix}\]
Thus neither \(E[p_i\epsilon_{di}]\) nor \(E[p_i\epsilon_{si}]\) are zero, so \(p_i\) is endogenous.

Running an OLS of \(q_i\) on \(p_i\) will give a biased estimate of \(\beta_d\).
We estimate \(Cov(q_i,p_i)/Var(p_i)\), and assuming demanad and supply shocks uncorrelated:
\[Cov(q_i,p_i)/Var(p_i)=\frac{\frac{\beta_s}{(\beta_s+\beta_d)^2}Var(\epsilon_{di})-\frac{\beta_d}{(\beta_s+\beta_d)^2}Var(\epsilon_{si})}{\frac{1}{(\beta_s+\beta_d)^2}Var(\epsilon_{di})-\frac{1}{(\beta_s+\beta_d)^2}Var(\epsilon_{si})}\]
\[=\beta_s\frac{Var(\epsilon_{di})}{Var(\epsilon_{di})-Var(\epsilon_{si})}-\beta_d\frac{Var(\epsilon_{si})}{Var(\epsilon_{di})-Var(\epsilon_{si})}\]
That is, some linear combination of the slopes of the demand and supply curves.

\subsection{Omitted Variables}
Another example of endogeneity would be a structural equation connecting two variables that are \underline{\smash{both chosen}} by economics agents, say, wage and educaiton
\[wage_i=\beta_1+\beta_2educ_i+\epsilon_i\]
Both \(wage_i\) and \(educ_i\) may be affected by person i's ability or some other factor belonging to \(\epsilon_i\). Here the structural equation is thought of reflecting a causal relationship, which would be observed if we could randomly assign education levels to people independent of ability or anything else.
In reality as this does not occur we cannot rule out that this choice may have been affected by other factors influencing wage.

\section{Instrumental Variables}
We formalise the device used in Solution 2 of the measurement error.
Consider a linear regression model:
\[y_i=x_i'\beta+\epsilon_i\]
where \(E(\epsilon_i|x_i)=0\) and so \(x_i\) is endogenous. Suppose we have a variable \(w_i\) such that:
\[E[w_i\epsilon_i]=0 \text{ (exogeneity)}\]
\[E[w_iw_i']>0 \text{ (no redundant instruments, =non singular?)}\]
\[E[w_ix_i'] \text{ has full column rank (relevance)}\]
Then\[0=E[w_i\epsilon_i]=E[w_i(y_i-x_i'\beta)]=E[w_iy_i]-E[w_ix_i']\beta\]
\[\implies \beta=E[w_ix_i']^{-1}E[w_iy_i]\]

This motivates \(\hat\beta_{IV}\)
\[\hat\beta_{IV}=(W'X)^{-1}W'Y=\left(\frac{1}{n}\sum_{i=1}^nw_ix_i'\right)^{-1}\frac{1}{n}\sum_{i=1}^nw_iy_i\]
\[\xrightarrow{p}E[w_ix_i']^{-1}E[w_iy_i]=E[w_ix_i']^{-1}E[w_ix_i']\beta+E[w_i\epsilon_i]=\beta\]
\begin{note} \textbf{Understanding the Relevance Condition} \\
    Suppose we start with the regression model:
    \[y_i=x_{1i}'\beta_1+x_{2i}'\beta_2 + \epsilon_i\]
    where \(x_{2i}\) is a scalar endogenous variable, and \(x_{1i}\) is a \((k-1)\times1\) vector of exogenous variables.
    \\ \\ We define \(w_i\) as a vector of instruments, where exogenous variables instrument themselves and \(z_i\) is a scalar IV for \(x_{2i}\)
    \\ \\ Given \(E[w_iw_i']\) nonsingular
    \[E[w_ix_i'] \text{ full column rank} \iff E(w_iw_i')^{-1}E[w_ix_i']\text{ full column rank}\]
    But this represents the set of population coefficients in a regression model of \(x_i\) on \(w_i\).
    \[\because E(w_iw_i')^{-1}E[w_ix_i']=E(w_iw_i')^{-1}E[w_i'(x_{(1,1i)}\, ...\,x_{(k-1,1i)}\: x_{2i})]\]
    \[=(\vec\beta_{1,1}\, ...\, \vec\beta_{k-1,1}\, ...\, \vec\beta_2)\]
Consider \(\vec\beta_{1,1}\):
\\ This represents the coefficient of \(x_{1,1i}\) in the regression of \(x_{1,1i}\) on \(w_i\).
But clearly as \(x_{1,1i}\) is included as a regressor in \(w_i\), \(\vec\beta_{1,1}=\begin{psmallmatrix} 1 \\ 0 \\ \vdots \\ 0 \end{psmallmatrix}\).
\\ For \(\vec\beta_2\):
\\ This represents \(\beta\) in the regression model:
\[x_{2i}=w_i'\vec\beta+\nu_i= \pi_1x_{1,1i} + \,...\, + \pi_{k-1}x_{k-1,1i} \,+\, \pi_kz_i + \nu_i\]
\\Therefore: \[E(w_iw_i')^{-1}E[w_ix_i']= \begin{pmatrix} I_{k-1} & \vec\pi \\ \vec0' & \pi_k \end{pmatrix}\]
where \(\vec\pi\) is \((\pi_1\, ...\, \pi_{k-1})'\)
\\ \\
Clearly then for this to be full rank (implying the relevance condition), \underline{\textbf{we need \(\pi_k\neq0\)}}.
\\ This means that the instrument \(z_i\) is correlated with \(x_{2i}\), even after the effects of all the other exogenous variables have been controlled for.

\end{note}
\vspace{5mm}
\begin{example} \textbf{Demand and Supply shifters} \\
    Suppose the market is a local fish market as in Graddy(1985).
    We may think supply would be affected by weather offshore \(w_i\), so that:
    \[q_i=-\beta_sp_i+\gamma w_i+\epsilon_{si}\]
    whereas demand will not be directly affected by \(w_i\) so that:
    \[q_i=\beta_dp_i+\epsilon_{di}\]
    Then we can use \(w_i\) as an instrument for \(p_i\) in the estimation of the demand (but not supply) equation.
    As these two equations need to equal, we can guarantee the relevance condition for \(\gamma\neq0\), as we can express \(p_i\) as a function of \(w_i\).
\end{example}
\vspace{1mm}
\begin{example} \textbf{Education and Wage} \\
Angrist and Krueger (1991) propose the
quarter of birth indicator as the instrument for education. Due to compulsory education laws in the United States,
you cannot drop out from school until you are 16, so people who are born in the first quarter of the year, being oldest
in their class, may drop out more often than those born in the other quarters. This insures that \(E[w_ix_i]\neq0\), whereas,
arguably, the quarter of birth should not be related to any other determinants of your wage, so that \(E[w_i\epsilon_i] = 0\).
\end{example}

\section{Appendix}

Proof of Lemma 13.0.1:
\begin{proof}
    \underline{\smash{Approach 1 (CMT only)}}:
        \[X_n\overset{p}{\to}X\text{ and }Y_n\overset{p}{\to}Y \iff (X_n,Y_n) \overset{p}{\to}(X,Y)\]
    Define a continuous function:
    \[f(x,y)=xy\]
    Then by the CMT (applied to vector):
    \[f(X_n,Y_n)\overset{p}{\to}f(X,Y)\]
    \[\implies X_nY_n\overset{p}{\to}XY\]
    \\
    \underline{\smash{Approach 2 (Slutsky + CMT)}}:
    \[X_n\overset{p}{\to}X \implies X_n\overset{d}{\to}X\]
    \[Y_n\overset{p}{\to}X\]
    By Slutsky's theorem:
    \[X_nY_n\overset{d}{\to}XY\]
    \[\implies X_nY_n\overset{p}{\to}XY\]
    (this implication only holds for RHS constant)
    \end{proof}

\end{document}