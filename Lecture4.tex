\documentclass[DIV=14,titlepage=false]{scrreprt}
\usepackage{parskip}
%%%%%%%%%%%%%%%%%%%%%%%%%%%%%%%%%%%%%%%%%%%%%%%%%%%%%%%%%%%%%%%%%%%%%%%%%%%%%%%
%                                Basic Packages                               %
%%%%%%%%%%%%%%%%%%%%%%%%%%%%%%%%%%%%%%%%%%%%%%%%%%%%%%%%%%%%%%%%%%%%%%%%%%%%%%%
% Gives us multiple colors.
\usepackage[usenames,dvipsnames,pdftex]{xcolor}
% Lets us style link colors.
\usepackage{hyperref}
% Lets us import images and graphics.
\usepackage{graphicx}
% Lets us use figures in floating environments.
\usepackage{float}
% Lets us create multiple columns.
\usepackage{multicol}
% Gives us better math syntax.
\usepackage{amsmath,amsfonts,mathtools,amsthm,amssymb}
% Lets us strikethrough text.
\usepackage{cancel}
% Lets us edit the caption of a figure.
\usepackage{caption}
% Lets us import pdf directly in our tex code.
\usepackage{pdfpages}
% Lets us do algorithm stuff.
\usepackage[ruled,vlined,linesnumbered]{algorithm2e}
% Gets rid of some errors.
\usepackage{scrhack}
\def\class{article}
\usepackage{geometry}
\geometry{margin=0.9in}
%%%%%%%%%%%%%%%%%%%%%%%%%%%%%%%%%%%%%%%%%%%%%%%%%%%%%%%%%%%%%%%%%%%%%%%%%%%%%%%
%                                Basic Settings                               %
%%%%%%%%%%%%%%%%%%%%%%%%%%%%%%%%%%%%%%%%%%%%%%%%%%%%%%%%%%%%%%%%%%%%%%%%%%%%%%%

%%%%%%%%%%%%%
%  Symbols  %
%%%%%%%%%%%%%

\let\implies\Rightarrow
\let\impliedby\Leftarrow
\let\iff\Leftrightarrow
\let\epsilon\varepsilon

%%%%%%%%%%%%
%  Tables  %
%%%%%%%%%%%%

\setlength{\tabcolsep}{5pt}
\renewcommand\arraystretch{1.5}

%%%%%%%%%%%%%%%%%%%%%%%
%  Center Title Page  %
%%%%%%%%%%%%%%%%%%%%%%%

\usepackage{titling}
\renewcommand\maketitlehooka{\null\mbox{}\vfill}
\renewcommand\maketitlehookd{\vfill\null}

%%%%%%%%%%%%%%%%%%%%%%%%%%%%%%%%%%%%%%%%%%%%%%%%%%%%%%%
%  Create a grey background in the middle of the PDF  %
%%%%%%%%%%%%%%%%%%%%%%%%%%%%%%%%%%%%%%%%%%%%%%%%%%%%%%%

\usepackage{eso-pic}
\newcommand\definegraybackground{
  \definecolor{reallylightgray}{HTML}{FAFAFA}
  \AddToShipoutPicture{
    \ifthenelse{\isodd{\thepage}}{
      \AtPageLowerLeft{
        \put(\LenToUnit{\dimexpr\paperwidth-222pt},0){
          \color{reallylightgray}\rule{222pt}{297mm}
        }
      }
    }
    {
      \AtPageLowerLeft{
        \color{reallylightgray}\rule{222pt}{297mm}
      }
    }
  }
}

%%%%%%%%%%%%%%%%%%%%%%%%
%  Modify Links Color  %
%%%%%%%%%%%%%%%%%%%%%%%%

\hypersetup{
  % Enable highlighting links.
  colorlinks,
  % Change the color of links to blue.
  linkcolor={black},
  % Change the color of citations to black.
  citecolor={black},
  % Change the color of url's to blue with some black.
  urlcolor=blue
}

%%%%%%%%%%%%%%%%%%
% Fix WrapFigure %
%%%%%%%%%%%%%%%%%%

\newcommand{\wrapfill}{\par\ifnum\value{WF@wrappedlines}>0
    \parskip=0pt
    \addtocounter{WF@wrappedlines}{-1}%
    \null\vspace{\arabic{WF@wrappedlines}\baselineskip}%
    \WFclear
\fi}

%%%%%%%%%%%%%%%%%
% Multi Columns %
%%%%%%%%%%%%%%%%%

\let\multicolmulticols\multicols
\let\endmulticolmulticols\endmulticols

\RenewDocumentEnvironment{multicols}{mO{}}
{%
  \ifnum#1=1
    #2%
  \else % More than 1 column
    \multicolmulticols{#1}[#2]
  \fi
}
{%
  \ifnum#1=1
\else % More than 1 column
  \endmulticolmulticols
\fi
}

\newlength{\thickarrayrulewidth}
\setlength{\thickarrayrulewidth}{5\arrayrulewidth}

%%%%%%%%%%%%%%%%%%%%
%  Import Figures  %
%%%%%%%%%%%%%%%%%%%%

\usepackage{import}
\pdfminorversion=7

% EXAMPLE:
% 1. \incfig{limit-graph}
% 2. \incfig[0.4]{limit-graph}
% Parameters:
% 1. The figure name. It should be located in figures/NAME.tex_pdf.
% 2. (Optional) The width of the figure. Example: 0.5, 0.35.
\newcommand\incfig[2][1]{%
  \def\svgwidth{#1\columnwidth}
  \import{./figures/}{#2.pdf_tex}
}

\begingroup\expandafter\expandafter\expandafter\endgroup
\expandafter\ifx\csname pdfsuppresswarningpagegroup\endcsname\relax
\else
  \pdfsuppresswarningpagegroup=1\relax
\fi

%%%%%%%%%%%%%
%  Correct  %
%%%%%%%%%%%%%

% EXAMPLE:
% 1. \correct{INCORRECT}{CORRECT}
% Parameters:
% 1. The incorrect statement.
% 2. The correct statement.
\definecolor{correct}{HTML}{009900}
\newcommand\correct[2]{{\color{red}{#1 }}\ensuremath{\to}{\color{correct}{ #2}}}



\newcommand{\R}{\mathbb{R}}
\newcommand{\Z}{\mathbb{Z}}
\newcommand{\E}{\mathbb{E}}
\newcommand{\B}{\ensuremath{\mathcal{B}}}
\newcommand{\X}{\ensuremath{\mathcal{X}}}
\newcommand{\Y}{\ensuremath{\mathcal{Y}}}
\newcommand{\mA}{\ensuremath{\mathbf{A}}}
\newcommand{\mB}{\ensuremath{\mathbf{B}}}
\newcommand{\mC}{\ensuremath{\mathbf{C}}}
\newcommand{\mD}{\ensuremath{\mathbf{D}}}
\newcommand{\mX}{\ensuremath{\mathbf{X}}}
\newcommand{\mY}{\ensuremath{\mathbf{Y}}}
\newcommand{\mx}{\ensuremath{\mathbf{x}}}
\newcommand{\my}{\ensuremath{\mathbf{y}}}
\newcommand{\mI}{\ensuremath{\mathbf{I}}}
\newcommand{\mi}{\ensuremath{\mathbf{\iota}}}
\newcommand{\mmu}{\ensuremath{\mathbf{\mu}}}
\newcommand{\mc}{\ensuremath{\mathbf{c}}}
\newcommand{\mSigma}{\ensuremath{\mathbf{\Sigma}}}
\newcommand{\mzero}{\ensuremath{\mathbf{0}}}
\newcommand{\independent}{\perp\!\!\!\!\perp} 
\setlength{\parindent}{0pt}
%%%%%%%%%%%%%%%%%%%%%%%%%%%%%%%%%%%%%%%%%%%%%%%%%%%%%%%%%%%%%%%%%%%%%%%%%%%%%%%
%                                 Environments                                %
%%%%%%%%%%%%%%%%%%%%%%%%%%%%%%%%%%%%%%%%%%%%%%%%%%%%%%%%%%%%%%%%%%%%%%%%%%%%%%%

\usepackage{varwidth}
\usepackage{thmtools}
\usepackage[most,many,breakable]{tcolorbox}

\tcbuselibrary{theorems,skins,hooks}
\usetikzlibrary{arrows,calc,shadows.blur}

%%%%%%%%%%%%%%%%%%%
%  Define Colors  %
%%%%%%%%%%%%%%%%%%%

\definecolor{myblue}{RGB}{45, 111, 177}
\definecolor{mygreen}{RGB}{56, 140, 70}
\definecolor{myred}{RGB}{199, 68, 64}
\definecolor{mypurple}{RGB}{197, 92, 212}

\definecolor{definition}{HTML}{228b22}
\definecolor{theorem}{HTML}{00007B}
\definecolor{example}{HTML}{2A7F7F}
\definecolor{definition}{HTML}{228b22}
\definecolor{prop}{HTML}{191971}
\definecolor{lemma}{HTML}{983b0f}
\definecolor{exercise}{HTML}{88D6D1}

\colorlet{definition}{mygreen!85!black}
\colorlet{claim}{mygreen!85!black}
\colorlet{corollary}{mypurple!85!black}
\colorlet{proof}{theorem}

%%%%%%%%%%%%%%%%%%%%%%
%  Helpful Commands  %
%%%%%%%%%%%%%%%%%%%%%%

% EXAMPLE:
% 1. \createnewtheoremstyle{thmdefinitionbox}{}{}
% 2. \createnewtheoremstyle{thmtheorembox}{}{}
% 3. \createnewtheoremstyle{thmproofbox}{qed=\qedsymbol}{
%       rightline=false, topline=false, bottomline=false
%    }
% Parameters:
% 1. Theorem name.
% 2. Any extra parameters to pass directly to declaretheoremstyle.
% 3. Any extra parameters to pass directly to mdframed.
\newcommand\createnewtheoremstyle[3]{
  \declaretheoremstyle[
  headfont=\bfseries\sffamily, bodyfont=\normalfont, #2,
  mdframed={
    #3,
  },
  ]{#1}
}

% EXAMPLE:
% 1. \createnewcoloredtheoremstyle{thmdefinitionbox}{definition}{}{}
% 2. \createnewcoloredtheoremstyle{thmexamplebox}{example}{}{
%       rightline=true, leftline=true, topline=true, bottomline=true
%     }
% 3. \createnewcoloredtheoremstyle{thmproofbox}{proof}{qed=\qedsymbol}{backgroundcolor=white}
% Parameters:
% 1. Theorem name.
% 2. Color of theorem.
% 3. Any extra parameters to pass directly to declaretheoremstyle.
% 4. Any extra parameters to pass directly to mdframed.
\newcommand\createnewcoloredtheoremstyle[4]{
  \declaretheoremstyle[
  headfont=\bfseries\sffamily\color{#2}, bodyfont=\normalfont, #3,
  mdframed={
    linewidth=2pt,
    rightline=false, leftline=true, topline=false, bottomline=false,
    linecolor=#2, backgroundcolor=#2!5, #4,
  },
  ]{#1}
}

%%%%%%%%%%%%%%%%%%%%%%%%%%%%%%%%%%%
%  Create the Environment Styles  %
%%%%%%%%%%%%%%%%%%%%%%%%%%%%%%%%%%%

\makeatletter
\@ifclasswith\class{nocolor}{
  % Environments without color.

  \createnewtheoremstyle{thmdefinitionbox}{}{}
  \createnewtheoremstyle{thmtheorembox}{}{}
  \createnewtheoremstyle{thmexamplebox}{}{}
  \createnewtheoremstyle{thmclaimbox}{}{}
  \createnewtheoremstyle{thmcorollarybox}{}{}
  \createnewtheoremstyle{thmpropbox}{}{}
  \createnewtheoremstyle{thmlemmabox}{}{}
  \createnewtheoremstyle{thmexercisebox}{}{}
  \createnewtheoremstyle{thmdefinitionbox}{}{}
  \createnewtheoremstyle{thmquestionbox}{}{}
  \createnewtheoremstyle{thmsolutionbox}{}{}

  \createnewtheoremstyle{thmproofbox}{qed=\qedsymbol}{}
  \createnewtheoremstyle{thmexplanationbox}{}{}
}{
  % Environments with color.

  \createnewcoloredtheoremstyle{thmdefinitionbox}{definition}{}{}
  \createnewcoloredtheoremstyle{thmtheorembox}{theorem}{}{}
  \createnewcoloredtheoremstyle{thmexamplebox}{example}{}{
    rightline=true, leftline=true, topline=true, bottomline=true
  }
  \createnewcoloredtheoremstyle{thmclaimbox}{claim}{}{}
  \createnewcoloredtheoremstyle{thmcorollarybox}{corollary}{}{}
  \createnewcoloredtheoremstyle{thmpropbox}{prop}{}{}
  \createnewcoloredtheoremstyle{thmlemmabox}{lemma}{}{}
  \createnewcoloredtheoremstyle{thmexercisebox}{exercise}{}{}

  \createnewcoloredtheoremstyle{thmproofbox}{proof}{qed=\qedsymbol}{backgroundcolor=white}
  \createnewcoloredtheoremstyle{thmexplanationbox}{example}{qed=\qedsymbol}{backgroundcolor=white}
}
\makeatother

%%%%%%%%%%%%%%%%%%%%%%%%%%%%%
%  Create the Environments  %
%%%%%%%%%%%%%%%%%%%%%%%%%%%%%

\declaretheorem[numberwithin=section, style=thmtheorembox,     name=Theorem]{theorem}
\declaretheorem[numbered=no,          style=thmexamplebox,     name=Example]{example}
\declaretheorem[numberwithin=section, style=thmclaimbox,       name=Claim]{claim}
\declaretheorem[numberwithin=section, style=thmcorollarybox,   name=Corollary]{corollary}
\declaretheorem[numberwithin=section, style=thmpropbox,        name=Proposition]{prop}
\declaretheorem[numberwithin=section, style=thmlemmabox,       name=Lemma]{lemma}
\declaretheorem[numberwithin=section, style=thmexercisebox,    name=Exercise]{exercise}
\declaretheorem[numbered=no,          style=thmproofbox,       name=Proof]{replacementproof}
\declaretheorem[numbered=no,          style=thmexplanationbox, name=Proof]{expl}

\makeatletter
\@ifclasswith\class{nocolor}{
  % Environments without color.

  \newtheorem*{note}{Note}

  \declaretheorem[numberwithin=section, style=thmdefinitionbox, name=Definition]{definition}
  \declaretheorem[numberwithin=section, style=thmquestionbox,   name=Question]{question}
  \declaretheorem[numberwithin=section, style=thmsolutionbox,   name=Solution]{solution}
}{
  % Environments with color.

  \newtcbtheorem[number within=section]{Definition}{Definition}{
    enhanced,
    before skip=2mm,
    after skip=2mm,
    colback=red!5,
    colframe=red!80!black,
    colbacktitle=red!75!black,
    boxrule=0.5mm,
    attach boxed title to top left={
      xshift=1cm,
      yshift*=1mm-\tcboxedtitleheight
    },
    varwidth boxed title*=-3cm,
    boxed title style={
      interior engine=empty,
      frame code={
        \path[fill=tcbcolback]
        ([yshift=-1mm,xshift=-1mm]frame.north west)
        arc[start angle=0,end angle=180,radius=1mm]
        ([yshift=-1mm,xshift=1mm]frame.north east)
        arc[start angle=180,end angle=0,radius=1mm];
        \path[left color=tcbcolback!60!black,right color=tcbcolback!60!black,
        middle color=tcbcolback!80!black]
        ([xshift=-2mm]frame.north west) -- ([xshift=2mm]frame.north east)
        [rounded corners=1mm]-- ([xshift=1mm,yshift=-1mm]frame.north east)
        -- (frame.south east) -- (frame.south west)
        -- ([xshift=-1mm,yshift=-1mm]frame.north west)
        [sharp corners]-- cycle;
      },
    },
    fonttitle=\bfseries,
    title={#2},
    #1
  }{def}

  \NewDocumentEnvironment{definition}{O{}O{}}
    {\begin{Definition}{#1}{#2}}{\end{Definition}}

  \newtcolorbox{note}[1][]{%
    enhanced jigsaw,
    colback=gray!20!white,%
    colframe=gray!80!black,
    size=small,
    boxrule=1pt,
    title=\textbf{Note:-},
    halign title=flush center,
    coltitle=black,
    breakable,
    drop shadow=black!50!white,
    attach boxed title to top left={xshift=1cm,yshift=-\tcboxedtitleheight/2,yshifttext=-\tcboxedtitleheight/2},
    minipage boxed title=1.5cm,
    boxed title style={%
      colback=white,
      size=fbox,
      boxrule=1pt,
      boxsep=2pt,
      underlay={%
        \coordinate (dotA) at ($(interior.west) + (-0.5pt,0)$);
        \coordinate (dotB) at ($(interior.east) + (0.5pt,0)$);
        \begin{scope}
          \clip (interior.north west) rectangle ([xshift=3ex]interior.east);
          \filldraw [white, blur shadow={shadow opacity=60, shadow yshift=-.75ex}, rounded corners=2pt] (interior.north west) rectangle (interior.south east);
        \end{scope}
        \begin{scope}[gray!80!black]
          \fill (dotA) circle (2pt);
          \fill (dotB) circle (2pt);
        \end{scope}
      },
    },
    #1,
  }

  \newtcbtheorem{Question}{Question}{enhanced,
    breakable,
    colback=white,
    colframe=myblue!80!black,
    attach boxed title to top left={yshift*=-\tcboxedtitleheight},
    fonttitle=\bfseries,
    title=\textbf{Question:-},
    boxed title size=title,
    boxed title style={%
      sharp corners,
      rounded corners=northwest,
      colback=tcbcolframe,
      boxrule=0pt,
    },
    underlay boxed title={%
      \path[fill=tcbcolframe] (title.south west)--(title.south east)
      to[out=0, in=180] ([xshift=5mm]title.east)--
      (title.center-|frame.east)
      [rounded corners=\kvtcb@arc] |-
      (frame.north) -| cycle;
    },
    #1
  }{def}

  \NewDocumentEnvironment{question}{O{}O{}}
  {\begin{Question}{#1}{#2}}{\end{Question}}

  \newtcolorbox{Solution}{enhanced,
    breakable,
    colback=white,
    colframe=mygreen!80!black,
    attach boxed title to top left={yshift*=-\tcboxedtitleheight},
    title=\textbf{Solution:-},
    boxed title size=title,
    boxed title style={%
      sharp corners,
      rounded corners=northwest,
      colback=tcbcolframe,
      boxrule=0pt,
    },
    underlay boxed title={%
      \path[fill=tcbcolframe] (title.south west)--(title.south east)
      to[out=0, in=180] ([xshift=5mm]title.east)--
      (title.center-|frame.east)
      [rounded corners=\kvtcb@arc] |-
      (frame.north) -| cycle;
    },
  }

  \NewDocumentEnvironment{solution}{O{}O{}}
  {\vspace{-10pt}\begin{Solution}{#1}{#2}}{\end{Solution}}
}
\makeatother

%%%%%%%%%%%%%%%%%%%%%%%%%%%%
%  Edit Proof Environment  %
%%%%%%%%%%%%%%%%%%%%%%%%%%%%

\renewenvironment{proof}[1][\proofname]{\vspace{-10pt}\begin{replacementproof}}{\end{replacementproof}}
\newenvironment{explanation}[1][\proofname]{\vspace{-10pt}\begin{expl}}{\end{expl}}

\theoremstyle{definition}

\newtheorem*{notation}{Notation}
\newtheorem*{previouslyseen}{As previously seen}
\newtheorem*{problem}{Problem}
\newtheorem*{observe}{Observe}
\newtheorem*{property}{Property}
\newtheorem*{intuition}{Intuition}

\title{%
R300 Econometrics}
\author{Metrics Enjoyers}
\date{Michaelmas Term, 2023-2024}

\setuptoc{toc}{leveldown}
\setcounter{chapter}{3}

\begin{document}
\chapter{Gauss-Markov Theorem. Estimation of \(\sigma^2\). Distribution of OLS in normal regression}

\section{Gauss-Markov Theorem}
\begin{theorem}Consider an \(n\times1\) random vector \(Y\) and an \(n\times k\) random matrix \(X\).
   \\  \\ Assume (no need for iid, large n or normality): 
\begin{itemize}
    \item \(\bold{GM1}\) No perfect multicollinearity: \(\text{rank}(X) = k\)
    \item \(\bold{GM2}\) Strict Exogeneity \(E(Y|X)=X\beta\), equivalently \(E(\epsilon|X)=0\)
    \item \(\bold{GM3}\) Homoskedasticity and no serial correlation \(Var(Y|X)=\sigma^2 I\), 
    \\ equivalently \(Var(\epsilon|X)=\sigma^2 I\)
\end{itemize}
    Then, the OLS estimator \(\hat{\beta}_{OLS}\) has the \underline{minimum conditional variance} in the class of estimators that, conditional on every X, are linear in Y and unbiased.
    Thus \(\hat{\beta}_{OLS}\) is the Best Linear conditionally Unbiased Estimator (BLUE).
\end{theorem}

A linear estimator of \(\beta\) is any estimator of the form \(\tilde{\beta}=A(X)Y\) where \(A(X)\) is a \(k\times n\) matrix.
For OLS \(\tilde{\beta}_{OLS}=A(X)Y=(X'X)^{-1}X'Y\)

\begin{definition}
    Minimum conditional variance implies:
    \[Var(\tilde{\beta}|X) - Var(\hat{\beta}_{OLS}|X) \quad \text{is positive semi-definite} \quad \forall \tilde{\beta}\]
    This \(k\times k\) matrix \(A\) is positive semi-definite iff \(z'Az\geq0\) for all \(k\times1\) vectors \(z\).
    Thus for any \(z\): \[z'Var(\tilde{\beta}|X)z\geq z'Var(\hat{\beta}_{OLS}|X)z\]
    Note this is equivalent to: \[Var(z'\tilde{\beta}|X) \geq Var(z'\hat{\beta}_{OLS}|X)\]
    Thus \underline{\smash{any linear combination}} of the elements of \(\tilde{\beta}\) has a conditional variance that is at least as large as the conditional variance of the corresponding linear combination of the elements of \(\hat{\beta}_{OLS}\). 
    
    In particular, any component of \(\tilde{\beta}\) has a conditional variance that is at least as large as the conditional variance of the corresponding component of \(\hat{\beta}_{OLS}\).
\end{definition}

\newpage

\section{GM PROOF}

\vspace{5mm}

\begin{proof}
\vspace{5mm}
\begin{lemma}
We know \[\hat{\beta}_{OLS}=(X'X)^{-1}X'Y\] is conditioanlly (and unconditionally) unbiased under GM2, and is a linear function of Y.
We also know under GM3 that \[Var(\hat{\beta}_{OLS}|X)=\sigma^2(X'X)^{-1}\]
\end{lemma}
Now consider any other linear conditionally unbiased estimator \(\tilde{\beta}=A(X)Y\).
\[E(\tilde{\beta}|X)=E(AY|X)=AE(Y|X)=AX\beta \quad \text{under GM3}\]
As we assume conditioanlly unbiased, for any \(\beta\)
\[AX\beta=\beta \implies AX=I\]
\begin{lemma}
    \[Var(\tilde{\beta}|X)=Var(AY|X)=AVar(Y|X)A'=A\sigma^2I_nA'=\sigma^2AA'\]
\end{lemma}

Decomposing A:
\[A=A-(X'X)^{-1}X'+(X'X)^{-1}X'=W+(X'X)^{-1}X'\]
Thus: \[Var(\tilde{\beta}|X)=\sigma^2(W+(X'X)^{-1}X')(W+(X'X)^{-1}X')'\]
\[=\sigma^2(W+(X'X)^{-1}X')(W'+X(X'X)^{-1})\]

But \[WX=AX-(X'X)^{-1}X'X=I-I=0\]

Therefore,
\[Var(\tilde{\beta}|X)=\sigma^2WW'+\sigma^2(X'X)^{-1}\]
\[= Var(\hat{\beta}_{OLS}|X)+\sigma^2WW'\]

\begin{lemma}
    \(\quad \sigma^2WW' \quad \text{is positive semi-definite}\)
    \\
    \\
    For any k-dimensional vector \(z\), denote the k-dimensional vector \(W'z\) as \(\alpha=(\alpha_1,...,\alpha_k)'\)
    \[z'\sigma^2WW'z=\sigma^2(z'W)(W'z)=\sigma^2\alpha'\alpha=\sigma^2\sum_{i=1}^k\alpha_i^2\geq0\]

\end{lemma}

Thus \(Var(\tilde{\beta}|X) - Var(\hat{\beta}_{OLS}|X)\) is psd for any linear conditionally unbiased estimator \(\tilde{\beta}=AY\).

\end{proof}

\newpage

\section{Estimation of \(\sigma^2\)}

Given the following but how to estimate?:
\[Var(\hat{\beta}_{OLS}|X)=\sigma^2(X'X)^{-1}\]
We are given X, thus only need to estimate \(\sigma^2\).

Note: \(\sigma^2=E(\epsilon_i^2|X)\) and since trivially \(\sigma^2=E(\sigma^2)\) we have:
\[\sigma^2=E(E(\epsilon_i^2|X))=E(\epsilon_i^2)\]

This suggests the MOM estimator:
\[\hat{\sigma}^2_{MOM}=\frac{1}{n}\sum_{i=1}^n{\epsilon}_i^2\]except we don't know \(\epsilon_i\) as \(\beta\) is unknown.

But with \(\hat{\beta}=\hat{\beta}_{OLS}\) we can use the sample analague \(\hat{\epsilon}_i=y_i-x_i'\hat{\beta}_{OLS}\) and thus:
\vspace{5mm}
\begin{theorem} The \underline{biased} (ML) estimator of \(\sigma^2\) is:
    \[\hat{\sigma}^2=\frac{1}{n}\sum_{i=1}^n{\hat{\epsilon}}_i^2\left(=\frac{1}{n}\sum_{i=1}^n\hat{\epsilon}'_i\hat{\epsilon}_i\right)\]
\end{theorem}

\subsection{Unbiased Estimator of \(\sigma^2\)}

We first compute the bias of \(E(\hat{\sigma}^2|X)\) to then correct for it:

\vspace{5mm}

\begin{lemma}
    \[\hat{\epsilon}=M_XY=M_X(X\beta+\epsilon)=M_X\epsilon\]
\end{lemma}

Then  \[E(\hat{\sigma}^2|X) = E(\hat{\epsilon}'\hat{\epsilon}|X)\]
\[=E(\epsilon'M_X'M_X\epsilon|X) = E(\epsilon'M_X\epsilon|X) = E(tr(\epsilon'M_X\epsilon)|X) , \quad \text{since argument is a scalar}\] \[\because\text{trace multiplications are commutative if conformations exists}\implies\]
\[=E(tr(M_X\epsilon\epsilon')|X)=tr(E(\epsilon'M_X\epsilon|X)) = tr(M_XE(\epsilon\epsilon'|X)) = tr(M_X\sigma^2I_n) = \sigma^2tr(M_X)\]

\begin{lemma}
    \[tr(M_X)=n-k\]
\end{lemma}
\vspace{5mm}    
\begin{proof}
    \[M_X=(I-X(X'X)^{-1}X')\]
    \[tr(M_X)=tr(I_n-X(X'X)^{-1}X')=tr(I_n)-((X'X)^{-1}X'X)\]
    \[tr(I_n)-tr(I_k)\]
    \[=n-k\]
\end{proof}

Thus \[E(\hat{\sigma}^2|X)=\frac{n-k}{n}\sigma^2\]

\begin{theorem}
    \[\hat{\sigma}^2_{u}=\frac{n}{n-k}\hat{\sigma}^2_{ML}= \frac{\hat{{\epsilon}}^{\prime}\hat{{\epsilon}}}{n-k}\] is an unbiased estimator of \(\sigma^2\)
    is an unbiased estimator of \(\sigma^2\)
\end{theorem}

\section{Estimation of standard errors}

By default STATA computes s.e. of component \(\hat{\beta}_j\) of \(\hat{\beta}_{OLS}\) as 
the square roots from the i-th diagonal element of \(\hat{\sigma}^2(X'X)^{-1}\) or more explicitly with the partitioned regression formulae:
\[\hat{se}(\hat{\beta}_i)=\frac{\hat{\sigma_u}}{\sqrt{X_i'M_{-i}X_i}}\]
where \(X_i\) is the i-th column of X (i-th regressor) and \(M_{-i}\) is the residual maker matrix in the regression on all the other explanatory variables but \(X_i\).

\section{Distribution of OLS}

Knowing the mean and variance of OLS is \underline{not sufficient} to test hypotheses about \(\beta\). We need to also know the distribution.
In small samples this is easy to derive with the following assumption:
\[\bold{Normal \ Regression:} \epsilon|X ~ N(0,\sigma^2I_n)\]
This subsumes GM2 and GM3 and adds normality. 

\vspace{5mm}

\begin{claim} There are several properties of the multivariate Gaussian which become useful in derivations.
    \begin{itemize}
        
        \item If \(\bold{Z\sim N(0,\sigma^2I_n)}\) and \(\bold{A}\) be any deterministic \(r\times n\) matrix, then \(\bold{AZ \sim N(0,\sigma^2AA')}\). In particular \underline{\smash{any linear combinations of normals is normal}}.
        \item The Normal distribution \(N(0,\sigma^2I_n)\) is \underline{\smash{invariant to rotations/orthogonal transformations}}. If \(\bold{Z \sim N(0,\sigma^2I_n)}\) and \(\bold{Q}\) is any \(n\times n\) orthogonal matrix, then \(\bold{QZ \sim N(0,\sigma^2I_n)}\), i.e. \(QZ\) has the same distribution as \(Z\).
    \end{itemize}
\end{claim}
\vspace{5mm}
\begin{theorem}
    Using the first property and the normal regression assumption, we obtain:
    \[\hat\beta_{OLS}|X = \beta + (X'X)^{-1}X'\epsilon|X \sim N(\beta,\sigma^2(X'X)^{-1})\]
\end{theorem}

\end{document}

% Nicely formatted boxes
%\begin{theorem}
%  This is a theorem.
%  \end{theorem}
%  \begin{proof}[Theorem 1]
%  This is a proof.
%  \end{proof}
%  \begin{example}
%  This is an example.
%  \end{example}
%  \begin{explanation}
%  This is an explanation.
%  \end{explanation}
%  \begin{claim}
%  This is a claim.
%  \end{claim}
%  \begin{corollary}
%  This is a corollary.
%  \end{corollary}
%  \begin{prop}
%  This is a proposition.
%  \end{prop}
%  \begin{lemma}
%  This is a lemma.
%  \end{lemma}
%  \begin{question}
%  This is a question.
%  \end{question}
%  \begin{solution}
%  This is a solution.
%  \end{solution}
%   \begin{question}
%  This is another question.
%  \end{question}
%  \begin{solution}
%  This is another solution.
%  \end{solution}
%  \begin{exercise}
%  This is an exercise.
%  \end{exercise}
%  \begin{definition}[Test]
%  This is a definition.
%  \end{definition}
%  \begin{note}
%  This is a note.
%  \end{note}