\documentclass[DIV=14,titlepage=false]{scrreprt}

%%%%%%%%%%%%%%%%%%%%%%%%%%%%%%%%%%%%%%%%%%%%%%%%%%%%%%%%%%%%%%%%%%%%%%%%%%%%%%%
%                                Basic Packages                               %
%%%%%%%%%%%%%%%%%%%%%%%%%%%%%%%%%%%%%%%%%%%%%%%%%%%%%%%%%%%%%%%%%%%%%%%%%%%%%%%
% Gives us multiple colors.
\usepackage[usenames,dvipsnames,pdftex]{xcolor}
% Lets us style link colors.
\usepackage{hyperref}
% Lets us import images and graphics.
\usepackage{graphicx}
% Lets us use figures in floating environments.
\usepackage{float}
% Lets us create multiple columns.
\usepackage{multicol}
% Gives us better math syntax.
\usepackage{amsmath,amsfonts,mathtools,amsthm,amssymb}
% Lets us strikethrough text.
\usepackage{cancel}
% Lets us edit the caption of a figure.
\usepackage{caption}
% Lets us import pdf directly in our tex code.
\usepackage{pdfpages}
% Lets us do algorithm stuff.
\usepackage[ruled,vlined,linesnumbered]{algorithm2e}
% Gets rid of some errors.
\usepackage{scrhack}
\def\class{article}
\usepackage{geometry}
\geometry{margin=0.9in}
%%%%%%%%%%%%%%%%%%%%%%%%%%%%%%%%%%%%%%%%%%%%%%%%%%%%%%%%%%%%%%%%%%%%%%%%%%%%%%%
%                                Basic Settings                               %
%%%%%%%%%%%%%%%%%%%%%%%%%%%%%%%%%%%%%%%%%%%%%%%%%%%%%%%%%%%%%%%%%%%%%%%%%%%%%%%

%%%%%%%%%%%%%
%  Symbols  %
%%%%%%%%%%%%%

\let\implies\Rightarrow
\let\impliedby\Leftarrow
\let\iff\Leftrightarrow
\let\epsilon\varepsilon

%%%%%%%%%%%%
%  Tables  %
%%%%%%%%%%%%

\setlength{\tabcolsep}{5pt}
\renewcommand\arraystretch{1.5}

%%%%%%%%%%%%%%%%%%%%%%%
%  Center Title Page  %
%%%%%%%%%%%%%%%%%%%%%%%

\usepackage{titling}
\renewcommand\maketitlehooka{\null\mbox{}\vfill}
\renewcommand\maketitlehookd{\vfill\null}

%%%%%%%%%%%%%%%%%%%%%%%%%%%%%%%%%%%%%%%%%%%%%%%%%%%%%%%
%  Create a grey background in the middle of the PDF  %
%%%%%%%%%%%%%%%%%%%%%%%%%%%%%%%%%%%%%%%%%%%%%%%%%%%%%%%

\usepackage{eso-pic}
\newcommand\definegraybackground{
  \definecolor{reallylightgray}{HTML}{FAFAFA}
  \AddToShipoutPicture{
    \ifthenelse{\isodd{\thepage}}{
      \AtPageLowerLeft{
        \put(\LenToUnit{\dimexpr\paperwidth-222pt},0){
          \color{reallylightgray}\rule{222pt}{297mm}
        }
      }
    }
    {
      \AtPageLowerLeft{
        \color{reallylightgray}\rule{222pt}{297mm}
      }
    }
  }
}

%%%%%%%%%%%%%%%%%%%%%%%%
%  Modify Links Color  %
%%%%%%%%%%%%%%%%%%%%%%%%

\hypersetup{
  % Enable highlighting links.
  colorlinks,
  % Change the color of links to blue.
  linkcolor={black},
  % Change the color of citations to black.
  citecolor={black},
  % Change the color of url's to blue with some black.
  urlcolor=blue
}

%%%%%%%%%%%%%%%%%%
% Fix WrapFigure %
%%%%%%%%%%%%%%%%%%

\newcommand{\wrapfill}{\par\ifnum\value{WF@wrappedlines}>0
    \parskip=0pt
    \addtocounter{WF@wrappedlines}{-1}%
    \null\vspace{\arabic{WF@wrappedlines}\baselineskip}%
    \WFclear
\fi}

%%%%%%%%%%%%%%%%%
% Multi Columns %
%%%%%%%%%%%%%%%%%

\let\multicolmulticols\multicols
\let\endmulticolmulticols\endmulticols

\RenewDocumentEnvironment{multicols}{mO{}}
{%
  \ifnum#1=1
    #2%
  \else % More than 1 column
    \multicolmulticols{#1}[#2]
  \fi
}
{%
  \ifnum#1=1
\else % More than 1 column
  \endmulticolmulticols
\fi
}

\newlength{\thickarrayrulewidth}
\setlength{\thickarrayrulewidth}{5\arrayrulewidth}

%%%%%%%%%%%%%%%%%%%%
%  Import Figures  %
%%%%%%%%%%%%%%%%%%%%

\usepackage{import}
\pdfminorversion=7

% EXAMPLE:
% 1. \incfig{limit-graph}
% 2. \incfig[0.4]{limit-graph}
% Parameters:
% 1. The figure name. It should be located in figures/NAME.tex_pdf.
% 2. (Optional) The width of the figure. Example: 0.5, 0.35.
\newcommand\incfig[2][1]{%
  \def\svgwidth{#1\columnwidth}
  \import{./figures/}{#2.pdf_tex}
}

\begingroup\expandafter\expandafter\expandafter\endgroup
\expandafter\ifx\csname pdfsuppresswarningpagegroup\endcsname\relax
\else
  \pdfsuppresswarningpagegroup=1\relax
\fi

%%%%%%%%%%%%%
%  Correct  %
%%%%%%%%%%%%%

% EXAMPLE:
% 1. \correct{INCORRECT}{CORRECT}
% Parameters:
% 1. The incorrect statement.
% 2. The correct statement.
\definecolor{correct}{HTML}{009900}
\newcommand\correct[2]{{\color{red}{#1 }}\ensuremath{\to}{\color{correct}{ #2}}}



\newcommand{\R}{\mathbb{R}}
\newcommand{\Z}{\mathbb{Z}}
\newcommand{\E}{\mathbb{E}}
\newcommand{\B}{\ensuremath{\mathcal{B}}}
\newcommand{\X}{\ensuremath{\mathcal{X}}}
\newcommand{\Y}{\ensuremath{\mathcal{Y}}}
\newcommand{\mA}{\ensuremath{\mathbf{A}}}
\newcommand{\mB}{\ensuremath{\mathbf{B}}}
\newcommand{\mC}{\ensuremath{\mathbf{C}}}
\newcommand{\mD}{\ensuremath{\mathbf{D}}}
\newcommand{\mX}{\ensuremath{\mathbf{X}}}
\newcommand{\mY}{\ensuremath{\mathbf{Y}}}
\newcommand{\mx}{\ensuremath{\mathbf{x}}}
\newcommand{\my}{\ensuremath{\mathbf{y}}}
\newcommand{\mI}{\ensuremath{\mathbf{I}}}
\newcommand{\mi}{\ensuremath{\mathbf{\iota}}}
\newcommand{\mmu}{\ensuremath{\mathbf{\mu}}}
\newcommand{\mc}{\ensuremath{\mathbf{c}}}
\newcommand{\mSigma}{\ensuremath{\mathbf{\Sigma}}}
\newcommand{\mzero}{\ensuremath{\mathbf{0}}}
\newcommand{\independent}{\perp\!\!\!\!\perp} 
\setlength{\parindent}{0pt}
%%%%%%%%%%%%%%%%%%%%%%%%%%%%%%%%%%%%%%%%%%%%%%%%%%%%%%%%%%%%%%%%%%%%%%%%%%%%%%%
%                                 Environments                                %
%%%%%%%%%%%%%%%%%%%%%%%%%%%%%%%%%%%%%%%%%%%%%%%%%%%%%%%%%%%%%%%%%%%%%%%%%%%%%%%

\usepackage{varwidth}
\usepackage{thmtools}
\usepackage[most,many,breakable]{tcolorbox}

\tcbuselibrary{theorems,skins,hooks}
\usetikzlibrary{arrows,calc,shadows.blur}

%%%%%%%%%%%%%%%%%%%
%  Define Colors  %
%%%%%%%%%%%%%%%%%%%

\definecolor{myblue}{RGB}{45, 111, 177}
\definecolor{mygreen}{RGB}{56, 140, 70}
\definecolor{myred}{RGB}{199, 68, 64}
\definecolor{mypurple}{RGB}{197, 92, 212}

\definecolor{definition}{HTML}{228b22}
\definecolor{theorem}{HTML}{00007B}
\definecolor{example}{HTML}{2A7F7F}
\definecolor{definition}{HTML}{228b22}
\definecolor{prop}{HTML}{191971}
\definecolor{lemma}{HTML}{983b0f}
\definecolor{exercise}{HTML}{88D6D1}

\colorlet{definition}{mygreen!85!black}
\colorlet{claim}{mygreen!85!black}
\colorlet{corollary}{mypurple!85!black}
\colorlet{proof}{theorem}

%%%%%%%%%%%%%%%%%%%%%%
%  Helpful Commands  %
%%%%%%%%%%%%%%%%%%%%%%

% EXAMPLE:
% 1. \createnewtheoremstyle{thmdefinitionbox}{}{}
% 2. \createnewtheoremstyle{thmtheorembox}{}{}
% 3. \createnewtheoremstyle{thmproofbox}{qed=\qedsymbol}{
%       rightline=false, topline=false, bottomline=false
%    }
% Parameters:
% 1. Theorem name.
% 2. Any extra parameters to pass directly to declaretheoremstyle.
% 3. Any extra parameters to pass directly to mdframed.
\newcommand\createnewtheoremstyle[3]{
  \declaretheoremstyle[
  headfont=\bfseries\sffamily, bodyfont=\normalfont, #2,
  mdframed={
    #3,
  },
  ]{#1}
}

% EXAMPLE:
% 1. \createnewcoloredtheoremstyle{thmdefinitionbox}{definition}{}{}
% 2. \createnewcoloredtheoremstyle{thmexamplebox}{example}{}{
%       rightline=true, leftline=true, topline=true, bottomline=true
%     }
% 3. \createnewcoloredtheoremstyle{thmproofbox}{proof}{qed=\qedsymbol}{backgroundcolor=white}
% Parameters:
% 1. Theorem name.
% 2. Color of theorem.
% 3. Any extra parameters to pass directly to declaretheoremstyle.
% 4. Any extra parameters to pass directly to mdframed.
\newcommand\createnewcoloredtheoremstyle[4]{
  \declaretheoremstyle[
  headfont=\bfseries\sffamily\color{#2}, bodyfont=\normalfont, #3,
  mdframed={
    linewidth=2pt,
    rightline=false, leftline=true, topline=false, bottomline=false,
    linecolor=#2, backgroundcolor=#2!5, #4,
  },
  ]{#1}
}

%%%%%%%%%%%%%%%%%%%%%%%%%%%%%%%%%%%
%  Create the Environment Styles  %
%%%%%%%%%%%%%%%%%%%%%%%%%%%%%%%%%%%

\makeatletter
\@ifclasswith\class{nocolor}{
  % Environments without color.

  \createnewtheoremstyle{thmdefinitionbox}{}{}
  \createnewtheoremstyle{thmtheorembox}{}{}
  \createnewtheoremstyle{thmexamplebox}{}{}
  \createnewtheoremstyle{thmclaimbox}{}{}
  \createnewtheoremstyle{thmcorollarybox}{}{}
  \createnewtheoremstyle{thmpropbox}{}{}
  \createnewtheoremstyle{thmlemmabox}{}{}
  \createnewtheoremstyle{thmexercisebox}{}{}
  \createnewtheoremstyle{thmdefinitionbox}{}{}
  \createnewtheoremstyle{thmquestionbox}{}{}
  \createnewtheoremstyle{thmsolutionbox}{}{}

  \createnewtheoremstyle{thmproofbox}{qed=\qedsymbol}{}
  \createnewtheoremstyle{thmexplanationbox}{}{}
}{
  % Environments with color.

  \createnewcoloredtheoremstyle{thmdefinitionbox}{definition}{}{}
  \createnewcoloredtheoremstyle{thmtheorembox}{theorem}{}{}
  \createnewcoloredtheoremstyle{thmexamplebox}{example}{}{
    rightline=true, leftline=true, topline=true, bottomline=true
  }
  \createnewcoloredtheoremstyle{thmclaimbox}{claim}{}{}
  \createnewcoloredtheoremstyle{thmcorollarybox}{corollary}{}{}
  \createnewcoloredtheoremstyle{thmpropbox}{prop}{}{}
  \createnewcoloredtheoremstyle{thmlemmabox}{lemma}{}{}
  \createnewcoloredtheoremstyle{thmexercisebox}{exercise}{}{}

  \createnewcoloredtheoremstyle{thmproofbox}{proof}{qed=\qedsymbol}{backgroundcolor=white}
  \createnewcoloredtheoremstyle{thmexplanationbox}{example}{qed=\qedsymbol}{backgroundcolor=white}
}
\makeatother

%%%%%%%%%%%%%%%%%%%%%%%%%%%%%
%  Create the Environments  %
%%%%%%%%%%%%%%%%%%%%%%%%%%%%%

\declaretheorem[numberwithin=section, style=thmtheorembox,     name=Theorem]{theorem}
\declaretheorem[numbered=no,          style=thmexamplebox,     name=Example]{example}
\declaretheorem[numberwithin=section, style=thmclaimbox,       name=Claim]{claim}
\declaretheorem[numberwithin=section, style=thmcorollarybox,   name=Corollary]{corollary}
\declaretheorem[numberwithin=section, style=thmpropbox,        name=Proposition]{prop}
\declaretheorem[numberwithin=section, style=thmlemmabox,       name=Lemma]{lemma}
\declaretheorem[numberwithin=section, style=thmexercisebox,    name=Exercise]{exercise}
\declaretheorem[numbered=no,          style=thmproofbox,       name=Proof]{replacementproof}
\declaretheorem[numbered=no,          style=thmexplanationbox, name=Proof]{expl}

\makeatletter
\@ifclasswith\class{nocolor}{
  % Environments without color.

  \newtheorem*{note}{Note}

  \declaretheorem[numberwithin=section, style=thmdefinitionbox, name=Definition]{definition}
  \declaretheorem[numberwithin=section, style=thmquestionbox,   name=Question]{question}
  \declaretheorem[numberwithin=section, style=thmsolutionbox,   name=Solution]{solution}
}{
  % Environments with color.

  \newtcbtheorem[number within=section]{Definition}{Definition}{
    enhanced,
    before skip=2mm,
    after skip=2mm,
    colback=red!5,
    colframe=red!80!black,
    colbacktitle=red!75!black,
    boxrule=0.5mm,
    attach boxed title to top left={
      xshift=1cm,
      yshift*=1mm-\tcboxedtitleheight
    },
    varwidth boxed title*=-3cm,
    boxed title style={
      interior engine=empty,
      frame code={
        \path[fill=tcbcolback]
        ([yshift=-1mm,xshift=-1mm]frame.north west)
        arc[start angle=0,end angle=180,radius=1mm]
        ([yshift=-1mm,xshift=1mm]frame.north east)
        arc[start angle=180,end angle=0,radius=1mm];
        \path[left color=tcbcolback!60!black,right color=tcbcolback!60!black,
        middle color=tcbcolback!80!black]
        ([xshift=-2mm]frame.north west) -- ([xshift=2mm]frame.north east)
        [rounded corners=1mm]-- ([xshift=1mm,yshift=-1mm]frame.north east)
        -- (frame.south east) -- (frame.south west)
        -- ([xshift=-1mm,yshift=-1mm]frame.north west)
        [sharp corners]-- cycle;
      },
    },
    fonttitle=\bfseries,
    title={#2},
    #1
  }{def}

  \NewDocumentEnvironment{definition}{O{}O{}}
    {\begin{Definition}{#1}{#2}}{\end{Definition}}

  \newtcolorbox{note}[1][]{%
    enhanced jigsaw,
    colback=gray!20!white,%
    colframe=gray!80!black,
    size=small,
    boxrule=1pt,
    title=\textbf{Note:-},
    halign title=flush center,
    coltitle=black,
    breakable,
    drop shadow=black!50!white,
    attach boxed title to top left={xshift=1cm,yshift=-\tcboxedtitleheight/2,yshifttext=-\tcboxedtitleheight/2},
    minipage boxed title=1.5cm,
    boxed title style={%
      colback=white,
      size=fbox,
      boxrule=1pt,
      boxsep=2pt,
      underlay={%
        \coordinate (dotA) at ($(interior.west) + (-0.5pt,0)$);
        \coordinate (dotB) at ($(interior.east) + (0.5pt,0)$);
        \begin{scope}
          \clip (interior.north west) rectangle ([xshift=3ex]interior.east);
          \filldraw [white, blur shadow={shadow opacity=60, shadow yshift=-.75ex}, rounded corners=2pt] (interior.north west) rectangle (interior.south east);
        \end{scope}
        \begin{scope}[gray!80!black]
          \fill (dotA) circle (2pt);
          \fill (dotB) circle (2pt);
        \end{scope}
      },
    },
    #1,
  }

  \newtcbtheorem{Question}{Question}{enhanced,
    breakable,
    colback=white,
    colframe=myblue!80!black,
    attach boxed title to top left={yshift*=-\tcboxedtitleheight},
    fonttitle=\bfseries,
    title=\textbf{Question:-},
    boxed title size=title,
    boxed title style={%
      sharp corners,
      rounded corners=northwest,
      colback=tcbcolframe,
      boxrule=0pt,
    },
    underlay boxed title={%
      \path[fill=tcbcolframe] (title.south west)--(title.south east)
      to[out=0, in=180] ([xshift=5mm]title.east)--
      (title.center-|frame.east)
      [rounded corners=\kvtcb@arc] |-
      (frame.north) -| cycle;
    },
    #1
  }{def}

  \NewDocumentEnvironment{question}{O{}O{}}
  {\begin{Question}{#1}{#2}}{\end{Question}}

  \newtcolorbox{Solution}{enhanced,
    breakable,
    colback=white,
    colframe=mygreen!80!black,
    attach boxed title to top left={yshift*=-\tcboxedtitleheight},
    title=\textbf{Solution:-},
    boxed title size=title,
    boxed title style={%
      sharp corners,
      rounded corners=northwest,
      colback=tcbcolframe,
      boxrule=0pt,
    },
    underlay boxed title={%
      \path[fill=tcbcolframe] (title.south west)--(title.south east)
      to[out=0, in=180] ([xshift=5mm]title.east)--
      (title.center-|frame.east)
      [rounded corners=\kvtcb@arc] |-
      (frame.north) -| cycle;
    },
  }

  \NewDocumentEnvironment{solution}{O{}O{}}
  {\vspace{-10pt}\begin{Solution}{#1}{#2}}{\end{Solution}}
}
\makeatother

%%%%%%%%%%%%%%%%%%%%%%%%%%%%
%  Edit Proof Environment  %
%%%%%%%%%%%%%%%%%%%%%%%%%%%%

\renewenvironment{proof}[1][\proofname]{\vspace{-10pt}\begin{replacementproof}}{\end{replacementproof}}
\newenvironment{explanation}[1][\proofname]{\vspace{-10pt}\begin{expl}}{\end{expl}}

\theoremstyle{definition}

\newtheorem*{notation}{Notation}
\newtheorem*{previouslyseen}{As previously seen}
\newtheorem*{problem}{Problem}
\newtheorem*{observe}{Observe}
\newtheorem*{property}{Property}
\newtheorem*{intuition}{Intuition}

\setuptoc{toc}{leveldown}

\begin{document}
\vspace{-10pt}
\setcounter{chapter}{5}

\chapter{Convergence concepts. Asymptotics of OLS.}
\vspace{-10pt}
\section{Convergence concepts}
\begin{definition}[Convergence in probability]
    A sequence of random scalars $\{ z_i \}_{i=1}^{\infty}$ converges in probability to $z$ iff $\forall \epsilon>$0, $\underset{ n \to \infty}{\lim} P(|z_n-z|\geq\epsilon)=0$, or equivalently $\underset{ n \to \infty}{\lim} P(|z_n-z|<\epsilon)=1.$ Written as $z_n \overset{p}{\to}z$ or $z_n-z = o_p(1)$ or plim$_{ n \to \infty}z_n=z$.
\end{definition}

This definition is extended to a sequence of random vectors or random matrices by requiring element-by-element convergence in probability. That is, a sequence of K-dimensional vectors {$\mathbf{z_n}$} converges in probability to a K-dimensional vector $\mathbf{z}$ if, for any $\epsilon>0$ \[\underset{ n \to \infty}{\lim} P(|z_{nk}-z_k|>\epsilon)=0 \quad \text{for all} k=1,2,...,K \] where $z_{nk}$ is the $k$-th element of $\mathbf{z_n}$ and $z_k$ the $k$-th element of $\mathbf{z}$.

\begin{exercise}
Let {$X_n$} be an IID sequence of continuous random variables having a uniform distribution over support \[ R_{X_n} = \left[ -\frac{1}{n}, \frac{1}{n} \right] \] with pdf \[ f_{X_n}(x) = 
\begin{cases} 
\frac{n}{2} & \text{if } x \in \left[-\frac{1}{n} ; \frac{1}{n}\right] \\
0 & \text{if } x \notin \left[-\frac{1}{n} ; \frac{1}{n}\right]
\end{cases}
\]
Find the probability limit (if it exists) of the sequence {$X_n$}. 
\end{exercise}
\begin{solution}
    Intuitively as $n\to \infty$ the probability density becomes concentrated around $x=0$; it seems reasonable to conjecture $X_n \overset{p}{\to} X=0$. To show this formally, for any $\epsilon>0$:
    \begin{align*}
        \underset{ n \to \infty}{\lim} P(|X_n-X|>\epsilon) &= \underset{ n \to \infty}{\lim} P(|X_n-0|>\epsilon)\\
        &= \underset{ n \to \infty}{\lim} [1-P(-\epsilon \leq X_n \leq \epsilon)]\\
        &= 1- \underset{ n \to \infty}{\lim} \int_{-\epsilon}^{\epsilon} f_{X_n}(x)dx\\
        &= 1- \underset{ n \to \infty}{\lim} \int_{\max(-\epsilon,-1/n)}^{\min(\epsilon, 1/n)} \frac{n}{2}dx \quad \text{($f(x)$ has no density outside $ [-\tfrac{1}{n}, \tfrac{1}{n}]$)}\\
        &= 1- \underset{ n \to \infty}{\lim} \int_{-1/n}^{1/n} \frac{n}{2}dx \quad \text{(when $n$ becomes large, $\frac{1}{n}<\epsilon$)}\\
        &= 1- \underset{ n \to \infty}{\lim}1\\
        &=0
    \end{align*}

\end{solution}

\begin{definition}[Convergence in distribution]
    A sequence of random scalars $\{ z_i \}_{i=1}^{\infty}$ converges in distribution to $z$ iff, $\underset{ n \to \infty}{\lim} F_{z_n}(z) = F_z (z)$ at all points where $F_z$ is continuous. Written as $z_n \overset{d}{\to} z$ or $z_n - z = O_p(1)$ or as "$z$ is the limiting distribution of $z_n$".
\end{definition}

Convergence in distribution is also known as weak convergence or the convergence in law. 
\begin{theorem}
    $\mathbf{z_n} \overset{d}{\to} \mathbf{z}$ iff $\E f(\mathbf{z_n}) \to \E f(\mathbf{z})$ for all bounded, continuous functions $ f$.
\end{theorem}
\begin{claim}
Convergence in probability implies convergence in distribution but not vice versa. The reverse only holds when the limit in distribution is a constant.
\end{claim}

\begin{example}[$z_n \overset{d}{\to} z \not \implies z_n \overset{p}{\to} z$]
    Let \( z \sim N(0,1) \). Let \( z_n = -z \) for \( n = 1,2,3,\ldots \); hence \( z_n \sim N(0,1) \).
    \( z_n \) has the same distribution function as \( z \) for all \( n \) so, trivially, \( \lim_{n \to \infty} F_n(x) = F(x) \) for all \( x \).
    Therefore, \( z_n \overset{d}{\to} z \). But \( P(|z_n - z| > \varepsilon) = P(|2z| > \varepsilon) = P(|z| > \varepsilon/2) \neq 0 \).
    So \( z_n \) does not tend to \( z \) in probability.
\end{example}

The extension to a sequence of random vectors is immediate: $\mathbf{z_n} \overset{d}{\to} \mathbf{z}$ if the joint c.d.f. $F_n$ of the random vector {$\mathbf{z_n}$} converges to the joint c.d.f. $F$ of {$\mathbf{z}$} at every continuity point of F. However, element-by-element convergence does not necessarily imply convergence for the vector sequence (unlike with convergence in probability). Intuitively this is because different c.d.f.'s can have the same marginals.\\
A common way to establish the connection between scalar convergence in distribution and vector convergence in distribution is for every linear combination of $z_{nk}$ to converge to the linear combination of $z_n$. Formally:

\begin{definition}[Cramer-Wold device]
    $\mathbf{z_n} \overset{d}{\to} \mathbf{z}$ if and only if $\mathbf{\lambda ' z_n} \overset{d}{\to} \mathbf{\lambda ' z}$ for every $\mathbf{\lambda} \in \R^k$ with $\mathbf{\lambda'\lambda}=1$. 
\end{definition}

\begin{note}
    \textbf{Big} $O$ \textbf{Little} $o$ \textbf{notation}
    \begin{itemize}
        \item Roughly speaking, a function is $ o(z) $ iff it's of lower asymptotic order than $ z $.
        \item $ f(n) = o(g(n)) $ iff $ \lim_{n \to \infty} f(n) / g(n) = 0 $. 
        \item If $ \{ f(n) \} $ is a sequence of random variables, then $ f(n) = o_p (g(n)) $ iff $ plim_{n \to \infty} f(n) / g(n) = 0 $.
        \item  We write $ X_n - X = o_p (n^{- \gamma}) $ iff $ n^{\gamma}(X_n - X) \xrightarrow{\text{p}} 0 $.
        \vspace{20pt}
        \item Roughly speaking, a function is $ O(z) $ iff it's of the same asymptotic order as $ z $.
        \item $ f(n) = O(g(n)) $ iff $ | f(n) / g(n) | < K $ for all $ n > N $ and some positive integer $ N $ and some constant $ K >0 $. 
        \item If $ \{ f(n) \} $ is a sequence of random variables, then $ f(n) = o_p (g(n)) $ iff $ plim_{n \to \infty} f(n) / g(n) = 0 $.
    \end{itemize}
\end{note}

\begin{definition}[Continuous mapping theorem (CMT)]
    Let $f$ be continuous at every point $a \in C$ where $P (z \in C) = 1$. Then
    \begin{enumerate}
        \item If $\mathbf{z_n} \overset{p}{\to} \mathbf{z}$, then $f (\mathbf{z_n}) \overset{p}{\to} f(\mathbf{z})$ 
        \item If $\mathbf{z_n} \overset{d}{\to} \mathbf{z}$, then $f (\mathbf{z_n}) \overset{d}{\to} f(\mathbf{z})$ 
    \end{enumerate}

\end{definition}

\begin{example}
    The CMT allows $f$ to be discontinuous only if the probability of being at a discontinuity point is zero.\\
     Consider $f(u)=u^{-1}$ is discontinuous at $u=0$, but if $z_n \overset{d}{\to} z \sim N(0,1)$ then $P (z =0) = 0$ so $z_n^{-1} \overset{d}{\to} z^{-1}$
\end{example}

\begin{corollary}[Slutsky's theorem]
    If \( z_n \overset{d}{\to} z \) and \( c_n \overset{p}{\to} c \) as \( n \to \infty \), then
\begin{enumerate}
    \item \( z_n + c_n \overset{d}{\to} z + c \)
    \item \( z_n c_n \overset{d}{\to} zc \)
    \item \( \frac{z_n}{c_n} \overset{d}{\to} \frac{z}{c} \) if \( c \neq 0 \).
\end{enumerate}
\end{corollary}
The requirement that \( c_n \) converges to a constant is important. If it were to converge to a non-degenerate random variable, the theorem would be no longer valid. For example, let \( z_n \sim \text{Uniform}(0,1) \) and \( c_n = -z_n \). The sum \( z_n + c_n = 0 \) for all values of \( n \). Moreover, \( c_n \overset{d}{\to} c \) where \( z \sim \text{Uniform}(0,1) \), \( c \sim \text{Uniform}(-1,0) \), and \( z \) and \( c \) are independent.

\begin{note}
    The theorem remains valid if we replace all convergences in distribution with convergences in probability.
\end{note}

\begin{proof}
This theorem follows from the fact that if \( z_n \) converges in distribution to \( z \) and \( c_n \) converges in probability to a constant \( c \), then the joint vector \( (z_n, c_n) \) converges in distribution to \( (z, c) \).

Next we apply the continuous mapping theorem, recognising the functions \( g(z,c) \) such as \( g(z,c) = z + c \), \( g(z,c) = zc \), and \( g(z,c) = z c^{-1} \) are continuous (for the last function to be continuous, \( c \) has to be invertible).
\end{proof}

\begin{definition}[Khinchine's law of large numbers]
    If $Y_i$ are i.i.d. with finite mean $\E Y_i = m < \infty$ then $\tfrac{1}{n}\sum_{i=1}^{n} Y_i \overset{p}{\to}m$
\end{definition}


\begin{lemma}[Markov's inequality]
    Let \( \xi \) be a non-negative random variable and let \( \varepsilon > 0 \) be a positive number. Then for any real number \( p > 0 \), the following inequality holds:
\[
P(|\xi| \geq \varepsilon) \leq \frac{E[|\xi|^p]}{\varepsilon^p}.
\]
\end{lemma}

\begin{proof}
    Let \( \xi \) be a non-negative random variable and \( \varepsilon > 0 \). For any positive integer \( p \):

    \begin{align*}
        E[|\xi|^p] &= \int_{0}^{\infty} x^p f_{\xi}(x) \, dx & \text{(expectation definition)} \\
        &= \int_{0}^{\varepsilon} x^p f_{\xi}(x) \, dx + \int_{\varepsilon}^{\infty} x^p f_{\xi}(x) \, dx & \text{(splitting the integral)} \\
        &\geq \int_{\varepsilon}^{\infty} \varepsilon^p f_{\xi}(x) \, dx & \text{(since \( x^p \geq \varepsilon^p \) for \( x \geq \varepsilon \))} \\
        &= \varepsilon^p P(|\xi| \geq \varepsilon) & \text{(definition of probability)} \\
        P(|\xi| \geq \varepsilon) &\leq \frac{E[|\xi|^p]}{\varepsilon^p} & \text{(Markov's inequality)}
        \end{align*}
\end{proof}


\begin{lemma}[Chebyshev's inequality]
    Let \( \eta \) be a random variable with \( \E[\eta] = m \) and \( \text{Var}(\eta) < \infty \). Then for any \( \varepsilon > 0 \),
    \[
    P(|\eta - \E[\eta]| \geq \varepsilon) \leq \frac{\text{Var}(\eta)}{\varepsilon^2}.
    \]
 \end{lemma}
\begin{proof}
    Using Markov's inequality, for any random variable \( \eta \) with finite expectation \( E[\eta] \) and finite non-zero variance \( \text{Var}(\eta) \), and for any \( \varepsilon > 0 \), we have:

    \begin{align*}
        P(|\eta - E[\eta]| \geq \varepsilon) &= P((\eta - E[\eta])^2 \geq \varepsilon^2) & \text{(squaring both sides)} \\
        &\leq \frac{E[(\eta - E[\eta])^2]}{\varepsilon^2} & \text{(applying Markov's inequality)} \\
        &= \frac{\text{Var}(\eta)}{\varepsilon^2}. & \text{(variance definition)}
        \end{align*}

\end{proof}

\begin{definition}[Chebyshev's law of large numbers]
    If $Y_i$ are uncorrelated, and $\E Y_i = m < \infty$, $Var(Y_i)=\sigma_i^2<\infty$ and $\tfrac{1}{n^2}\sum_{i=1}^{n} \sigma^2_i \to 0$, then $\tfrac{1}{n}\sum_{i=1}^{n} (Y_i-m) \overset{p}{\to}0$
\end{definition}

\begin{proof}[Chebyshev’s law of large numbers]

    Let \( Y_1, Y_2, \ldots, Y_n \) be uncorrelated random variables with \( E[Y_i] = m \) and \( \text{Var}(Y_i) = \sigma^2_i < \infty \). Assume that \( \frac{1}{n^2} \sum_{i=1}^n \sigma^2_i \to 0 \) as \( n \to \infty \). Define \( S_n = \frac{1}{n} \sum_{i=1}^n (Y_i - m) \). We want to show that \( S_n \to 0 \) in probability.
    By Chebyshev's inequality, for any \( \varepsilon > 0 \),
    \[
    P(|S_n - E[S_n]| \geq \varepsilon) \leq \frac{\text{Var}(S_n)}{\varepsilon^2}.
    \]
    
    Since \( E[S_n] = 0 \) and the \( Y_i \)'s are uncorrelated, we have
    \[
    \text{Var}(S_n) = \text{Var}\left(\frac{1}{n} \sum_{i=1}^n (Y_i - m)\right) = \frac{1}{n^2} \sum_{i=1}^n \text{Var}(Y_i - m) = \frac{1}{n^2} \sum_{i=1}^n \sigma^2_i.
    \]
    \[
        \implies
    P(|S_n| \geq \varepsilon) \leq \frac{1}{n^2} \frac{\sum_{i=1}^n \sigma^2_i}{\varepsilon^2}.
    \]
    
    Since \( \frac{1}{n^2} \sum_{i=1}^n \sigma^2_i \to 0 \), it follows that for any \( \varepsilon > 0 \),
    \[
    P(|S_n| \geq \varepsilon) \to 0 \text{ as } n \to \infty.
    \]
    
    Hence, \( S_n \to 0 \) in probability.
\end{proof}

\begin{definition}[Univariate Lindeberg-Lévy Central Limit Theorem]
    If \( Y_i \) are i.i.d. random variables with finite mean \( m \) and variance \( \sigma^2 \), then
    \[
    \sqrt{n}\left(\frac{1}{n} \sum_{i=1}^{n} Y_i - m\right) \overset{d}{\to} N(0, \sigma^2)
    \]

\end{definition}

\begin{definition}[Multivariate Lindeberg-Lévy Central Limit Theorem]
    If \( Y_i \) are i.i.d. with mean \( m \) and variance-covariance \( \Sigma \), then
    \[
    \sqrt{n}\left(\frac{1}{n} \sum_{i=1}^{n} Y_i - m\right) \overset{d}{\to} N(0, \Sigma).
    \]
\end{definition}

\begin{proof}
    Set \( \mathbf{c}  \in  \mathbb{R}^k \) with \( \mathbf{c}'\mathbf{c} = 1 \) and define \( u_i = \mathbf{c}'(\mathbf{y}_i - \mathbf{m}) \). The \( u_i \) are i.i.d. with \( E(u_i^2) = \mathbf{c}'\mathbf{\Sigma} \mathbf{c} < \infty \). By the univariate CLT,
    \[
        \mathbf{c}' \sqrt{n}(\mathbf{\bar{y}} - \mathbf{m}) = \frac{1}{\sqrt{n}} \sum_{i=1}^n u_i \overset{d}{\to} N(0, \mathbf{c}'\mathbf{\Sigma} \mathbf{c})
    \]
    Notice that if \( \mathbf{z} \sim N(0, \mathbf{\Sigma}) \) then \( \mathbf{c}'\mathbf{z} \sim N(0, \mathbf{c}'\mathbf{\Sigma} \mathbf{c}) \). Thus
    \[
    \mathbf{c}'\sqrt{n}(\mathbf{\bar{y}} - \mathbf{m}) \overset{d}{\to} \mathbf{c}'\mathbf{z}.
    \]
    Since this holds for all \( \mathbf{c} \), we can use the Cramer-Wold device:
    \[
    \sqrt{n}(\mathbf{\bar{y}} - \mathbf{m}) \overset{d}{\to} \mathbf{z} \sim N(0, \mathbf{\Sigma})
    \]
    \end{proof}


\section{OLS in large samples}

\begin{minipage}{.5\textwidth}
    \begin{align*}
        &(\text{OLS0}) & (y_i, x_i) &\text{ is an i.i.d. sequence} \\
        &(\text{OLS1}) & E(x_i x_i') &\text{ is finite non-singular} \\
        &(\text{OLS2}) & E(y_i|x_i) &= x_i'\beta \\
        &(\text{OLS3}) & \text{Var}(y_i|x_i) &= \sigma^2 \\
        &(\text{OLS4}) & E\epsilon_i^4 &< \infty, \quad E\|x_i\|^4 < \infty
    \end{align*}
\end{minipage}%
\begin{minipage}{.5\textwidth}
    \begin{align*}
        &(\text{GM1}) & \text{rank } \mathbf{X} &= k \\
        &(\text{GM2}) & E(\mathbf{Y}|\mathbf{X}) &= \mathbf{X}'\beta \\
        &(\text{GM3}) & \text{Var} (\mathbf{Y}|\mathbf{X}) &= \sigma^2 \mathbf{I}
    \end{align*}
\end{minipage}\\

\textbf{Remarks}\\
(OLS0): Equivalent to random sampling, tells us that the pairs $(x_i,y_i)$ are independent across $i$.\\
(OLS1): Ensures $\mathbf{X'X}$ is invertible, or comparatively in sample $\frac{1}{n}\sum_{i=1}^{n}x_i x_i'$ exists.\\
(OLS2): Since all other $x$'s are independent, this is equivalent to conditioning on all $x$'s\\
(OLS3): Homoskedasticity and no serial correlation\\
(OLS4): Implies the existence of $\E(\epsilon_i ^2 x_i x_i ')$ via Cauchy-Schwartz. This is required to use the CLT.\\
\begin{lemma}[Expectation inequality]
    For any random vector $Y\in \R^m$ with $\E \| Y \| < \infty$ then \[ \| \E[Y] \| \leq E \| Y \| \]
\end{lemma}
\begin{lemma}[Holder's inequality]
    If $p>1$ and $q>1$ and $\frac{1}{p} + \frac{1}{q} = 1$, then for any random $m\times n$ matrices $X$ and $Y$, \[ \left(\E \|X'Y\|\right) \leq \left(\E \|X\|^p\right)^{1/p}\left(\E \|Y\|^q\right)^{1/q}\]
\end{lemma}
\begin{corollary}[Cauchy-Schwartz inequality]
    For any random $m\times n$ matrices $X$ and $Y$, \[ \left(\E \|X'Y\|\right) \leq \left(\E \|X\|^2\right)^{1/2}\left(\E \|Y\|^2\right)^{1/2}\]
\end{corollary}

To see that the elements of $\E(\epsilon_i ^2 x_i x_i ')$ are finite:
\begin{align*}
    \| \E(\epsilon_i ^2 x_i x_i ') \| &\leq \E \|\epsilon_i ^2 x_i x_i '\| &\text{(using Lemma 6.2.1)}\\
    &= \E(\epsilon_i ^2 \|x_i\|^2)&\\
    &\leq \E\left(\epsilon_i ^4\right)^{1/2} \E\left( \|x_i\|^4\right)^{1/2} &\text{(using Corollary 6.2.1)}\\
    &< \infty &\text{(using OLS4)}
\end{align*}


\begin{theorem}
    Under OLS0-4:
    \begin{enumerate}
        \item $\hat\beta_{OLS} \overset{p}{\to} \beta$
        \item $\sqrt{n}(\hat\beta_{OLS}-\beta) \overset{d}{\to}$ N$\left(0,\sigma^2[\E(x_ix_i')]^{-1}\right)$
    \end{enumerate}
\end{theorem}
\begin{proof}
1. We only require OLS0-2 for consistency\footnote[1]{Strictly we only need OLS0,1,2': $\E(x_i \epsilon_i) =0$ }
\begin{align*}
    \hat\beta_{OLS} &= (X'X)^{-1}X'Y\\
    &= \beta + (X'X)^{-1}X'\epsilon\\
    &= \beta + \left(\frac{1}{n}\sum_{i=1}^{n} x_i x_i'\right)^{-1} \frac{1}{n}\sum_{i=1}^{n} x_i \epsilon_i
\end{align*}
Since $x_i \epsilon_i$ is i.i.d. by OLS0\footnote[2]{$x_i\epsilon_i = x_i(y_i-x_i'\beta)$ and we know $(y_i, x_i)$ i.i.d.} we can use Khinchine's LLN
\begin{align*}
     \frac{1}{n}\sum_{i=1}^{n} x_i x_i' \overset{p}{\to} \E(x_ix_i') \quad \text{ and } \quad \frac{1}{n}\sum_{i=1}^{n} x_i \epsilon_i &\overset{p}{\to} \E(x_i \epsilon_i) \\
     &=\E(\E(x_i \epsilon_i|x_i))\\
     &= 0 \quad \text{(using OLS2)}
\end{align*}
By the Continuous Mapping Theorem,
\[ \left(\frac{1}{n}\sum_{i=1}^{n} x_i x_i'\right)^{-1} \overset{p}{\to} [\E(x_ix_i')]^{-1} \quad \text{(exists due to OLS1)} \]
\[ \implies \left(\frac{1}{n}\sum_{i=1}^{n} x_i x_i'\right)^{-1} \frac{1}{n}\sum_{i=1}^{n} x_i \epsilon_i \overset{p}{\to} 0\]
2.
\[ \hat\beta_{OLS} - \beta = (X'X)^{-1}X'\epsilon = \left(\frac{1}{n}\sum_{i=1}^{n} x_i x_i'\right)^{-1} \frac{1}{n}\sum_{i=1}^{n} x_i \epsilon_i \]
\[ \implies \sqrt{n}\left(\hat\beta_{OLS} - \beta\right) = \left(\frac{1}{n}\sum_{i=1}^{n} x_i x_i'\right)^{-1} \frac{1}{\sqrt{n}}\sum_{i=1}^{n} x_i \epsilon_i \]

Using the CLT: 
\[ \frac{1}{\sqrt{n}}\sum_{i=1}^{n} x_i \epsilon_i \overset{d}{\to} N(0,Var(x_i\epsilon_i)) = N\left(0,\sigma^2\E(x_ix_i')\right)\]
Where the second equality follows from: 
\begin{align*}
    \text{Var}(x_i\epsilon_i) &= E[x_i\epsilon_i\epsilon_i'x_i'] - E[x_i\epsilon_i]E[x_i\epsilon_i]' \\
    &= E[\epsilon_i^2 x_i x_i'] - E[x_i\epsilon_i]E[x_i\epsilon_i]' \quad \text{(since $\epsilon_i$ scalar) }\\
    &= E[E(\epsilon_i^2 x_i x_i'|x_i)] - E[E(x_i\epsilon_i|x_i)]E[x_i\epsilon_i]' \quad \text{(first expectation exists by OLS4)} \\
    &= E[E(\epsilon_i^2 |x_i) x_i x_i'] - E[x_iE(\epsilon_i|x_i)]E[x_i\epsilon_i]' \\
    &= \sigma^2 E[x_i x_i']. \quad \text{(using OLS2) }
\end{align*}

Using the CMT: 

\begin{align*}
    \left(\frac{1}{n}\sum_{i=1}^{n} x_i x_i'\right)^{-1} \frac{1}{\sqrt{n}}\sum_{i=1}^{n} x_i \epsilon_i & \overset{d}{\to} [\E(x_ix_i')]^{-1} N\left(0,\sigma^2\E(x_ix_i')\right)\\
    &\sim N\left(0, [\E(x_ix_i')]^{-1}\sigma^2\E(x_ix_i')[\E(x_ix_i')]^{-1}\right)\\
    \sqrt{n}\left(\hat\beta_{OLS} - \beta\right) & \overset{d}{\to} N\left(0, \sigma^2[\E(x_ix_i')]^{-1}\right)
\end{align*}
\end{proof}
\begin{theorem}
    Under OLS0-4:
    \begin{enumerate}
        \item $\hat\sigma^2 \overset{p}{\to} \sigma^2$
        \item $W \overset{d}{\to} \chi^2(p)$
        \item $t \overset{d}{\to} N(0,1)$
    \end{enumerate}
\end{theorem}
\begin{proof}
    1.\footnote{See Lecture 5 for derivation of the first step}
    \begin{align*}
        \hat\sigma^2 &= \frac{1}{n-k} \epsilon' M_X \epsilon \\
        &= \frac{1}{n-k} \epsilon' (I-X(X'X)^{-1}X')\epsilon\\
        &= \frac{1}{n-k} \epsilon'\epsilon - \frac{1}{n-k} \epsilon' X(X'X)^{-1}X'\epsilon\\
        &= \frac{n}{n-k} \frac{1}{n} \sum_{i=1}^{n}\epsilon_i^2 - \frac{n}{n-k} \frac{1}{n} \sum_{i=1}^{n} x_i \epsilon_i \left(\frac{1}{n}\sum_{i=1}^{n} x_i x_i'\right)^{-1} \frac{1}{n} \sum_{i=1}^{n} x_i' \epsilon_i
    \end{align*}
    
    Using Khinchine's LLN:
    \[
        \frac{1}{n} \sum_{i=1}^{n}\epsilon_i^2 \overset{p}{\to} \E[\epsilon_i^2], \quad
        \frac{1}{n} \sum_{i=1}^{n}x_i\epsilon_i \overset{p}{\to} \E[x_i \epsilon_i]=0, \quad
        \frac{1}{n} \sum_{i=1}^{n} x_i x_i' \overset{p}{\to} \E(x_ix_i'), \quad
        \frac{1}{n} \sum_{i=1}^{n}x_i' \epsilon_i \overset{p}{\to} \E[x_i' \epsilon_i]=0
    \]
    
    Using CMT and Slutsky:
    \begin{align*}
        \hat\sigma^2 &\overset{p}{\to} \frac{n}{n-k} \E[\epsilon_i^2] + \frac{n}{n-k}\times 0 \\
        &= \E[\epsilon_i^2] \quad \text{(as } n \to \infty)\\
        &= \sigma^2
    \end{align*}
2.
\[ W= \frac{\sqrt{n}\left(R\hat\beta-q\right)'\left(\sigma^2R\left(\frac{1}{n}X'X\right)^{-1}R'\right)^{-1}\sqrt{n}\left(R\hat\beta-q\right)}{\hat\sigma^2/\sigma^2}\]
We have seen that $\hat\sigma^2 \overset{p}{\to} \sigma^2$, and
\begin{align*}
    \sqrt{n}\left(R\hat\beta-q\right) &= \sqrt{n}(\hat\beta - \beta) \quad \text{(since $H_0:R\beta =q$)}\\
    &\overset{d}{\to} R N\left(0, \sigma^2[\E(x_ix_i')]^{-1}\right)\\
    &= N\left(0, \sigma^2R[\E(x_ix_i')]^{-1}R'\right)\\
    &=\left(\sigma^2R[\E(x_ix_i')]^{-1}R'\right)^{1/2}N(0,I_p)
\end{align*}
Since $\frac{1}{n}X'X \overset{p}{\to}\E[x_ix_i']$, by the CMT,
\[\left(\sigma^2R\left(\frac{1}{n}X'X\right)^{-1}R'\right)^{-1} \overset{p}{\to} \left(\sigma^2R[\E(x_ix_i')]^{-1}R'\right)^{-1}\]
\begin{align*}
    \implies W &\overset{d}{\to} \frac{\left(\left(\sigma^2R[\E(x_ix_i')]^{-1}R'\right)^{1/2}N(0,I_p)\right)'\left(\sigma^2R[\E(x_ix_i')]^{-1}R'\right)^{-1}\left(\sigma^2R[\E(x_ix_i')]^{-1}R'\right)^{-1/2}N(0,I_p)}{1}\\
    &=(N(0,I_p))'\left(\sigma^2R[\E(x_ix_i')]^{-1}R'\right)^{1/2}\left(\sigma^2R[\E(x_ix_i')]^{-1}R'\right)^{-1}\left(\sigma^2R[\E(x_ix_i')]^{-1}R'\right)^{1/2}N(0,I_p)\\
    &=(N(0,I_p))'I_p N(0,I_p)\\
    &= \chi^2(p)
\end{align*}
3.
\begin{align*}
     t&= \frac{\hat\beta_j-\beta}{\sqrt{\hat\sigma^2 (X' X)^{-1}_{jj}}} \\
     &= \frac{(\hat\beta_j-\beta) / \sqrt{\sigma^2 (X' X)^{-1}_{jj}}}{\sqrt{\hat\sigma^2/{\sigma^2}}}\\
     &= \frac{\overset{d}{\to}N(0,1)}{\sqrt{\overset{p}{\to} 1}}\quad \text{($\hat\sigma^2 \overset{p}{\to} \sigma^2$ and Theorem 6.2.1-2)}\\
     & \overset{d}{\to} N(0,1) \quad \text{(by Slutsky)}
\end{align*}
\end{proof}
The distribution of the Wald statistic is as expected, recall $W/p |x \sim F (p,n-k)$ under normal regression, and thus we see $W|x \sim pF (p,n-k) \overset{d}{\to} \chi^2(p)$. Why?
\begin{align*}
    p\times F &= p\frac{\chi^2(p)/p}{\chi^2(n-k)/(n-k)}\\
    &= \frac{\chi^2(p)}{\chi^2(n-k)/(n-k)}
\end{align*}
\[
    \frac{\chi^2(n-k)}{n-k} = \frac{1}{n-k}\sum_{i=1}^{n-k}Z_i^2 \overset{p}{\to} \E[Z_i^2]= 1
\]
\[
\implies pF \overset{d}{\to} \chi^2(p)    
\]
\textbf{Asymptotic confidence intervals and sets}\\
Since $t \overset{d}{\to} N(0,1)$ we can build asymptotic confidence intervals for $\beta_j$. From the critical values of $N(0,1)$:\[
Pr \left(\left|\frac{\sqrt{n}(\hat\beta_j-\beta)}{\sqrt{\hat\sigma^2 (\frac{1}{n} X' X)^{-1}_{jj}}}\right| \leq 1.96 \right) \approx 0.95\]
\[ \implies Pr \left(\left|\hat\beta_j-\beta\right|\leq 1.96{\sqrt{\hat\sigma^2 (X' X)^{-1}_{jj}}} \right) \approx 0.95 \quad \text{(cancel n's and rearrange)}
\]
\[ \implies \left[\hat\beta_j - 1.96 \sqrt{\hat\sigma^2 (X' X)^{-1}_{jj}}, \hat\beta_j + 1.96 \sqrt{\hat\sigma^2 (X' X)^{-1}_{jj}} \right] \quad \text{Asymptotic confidence interval}
\]
This gives us the set all all values of $\beta_j$ that are not rejected by he t-test with asymptotic size 5\%. We say that the confidence interval is obtained by inversion of the test. We can similarly invert the Wald test, consider a test of the entire vector $\beta = b$ (i.e. $R=I_p$):
\begin{align*}
    W &= (\hat \beta - b)'(\hat\sigma^2 (X'X^{-1}))^{-1}(\hat \beta - b)\\
    &= \frac{(\hat \beta - b)'X'X(\hat \beta - b)}{\hat\sigma^2}
\end{align*}
The asymptotic 95\% confidence set for $\beta$ is the ellipsoid with centre $\hat \beta$:
    \[ \implies \left\{ \beta : \frac{(\hat \beta - \beta)'X'X(\hat \beta - \beta)}{\hat\sigma^2} \leq \chi^2_{0.95}(k) \right\} \]

\section{Delta method}
Sometimes we need to know confidence intervals or sets for some (possibly nonlinear) function of regression parameters. We can do this with the delta method.

\begin{definition}[Delta method]
    Suppose $\hat \theta$ is a k-dimensional vector where $\sqrt{n}(\hat\theta - \theta) \overset{d}{\to} \xi$, and suppose $g: \R^k \to \R$ has continuous first dervatives. Denote by $G(\theta)$ the $r \times k$ matrix of first derivatives evaluated at $\theta$: $G(\theta)\equiv\frac{\partial g(\theta)}{\partial \theta'}$ then as $n\to \infty$ \[ \sqrt{n} (g(\hat \theta ) - g(\theta)) \overset{d}{\to} G(\theta) \xi \] . In particular, if $\xi \sim N(0,V)$ then as $n\to \infty$ \[ \sqrt{n} (g(\hat \theta ) - g(\theta)) \overset{d}{\to} N(0,GVG') \]
\end{definition}

\begin{proof}
    By the mean value theorem, there exists a k-dimensional vector $\bar \theta$ between $\hat \theta$ and $\theta$ such that\[
    g(\hat{\theta}) - g(\theta) = \underset{r \times k}{G(\bar{\theta})} \underset{k \times 1}{(\hat{\theta} - \theta)}\]
    \[ \implies \sqrt{n}(g(\hat{\theta}) - g(\theta)) = G(\bar{\theta}) \sqrt{n} (\hat{\theta} - \theta)\]
    Since $\bar \theta$ is between $\hat \theta$ and $\theta$ and since $\hat \theta \overset{p}{\to} \theta$ we know $\bar \theta \overset{p}{\to}\theta$. $G()$ is assumed continuous, so by CMT: \[G(\bar \theta) \overset{p}{\to} G(\theta)\]
    \[ \implies \sqrt{n}(g(\hat{\theta}) - g(\theta)) = G(\bar{\theta}) \sqrt{n} (\hat{\theta} - \theta) \overset{p}{\to} G(\theta) \xi\]
\end{proof}

\begin{exercise}
    Let $\{\hat\theta_n\}$ be a sequence of 2x1 random vectors satisfying $\sqrt{n}(\hat\theta_0-\theta_0)\overset{d}{\to}N(0,V)$ where the asymptotic mean is $\theta_0 = [0,1]'$ and the asymptotic covariance matrix is $I_2$.\\
     Denote the two entries of $\hat\theta_n$ by $\hat\theta_{n,1}$ and $\hat\theta_{n,2}$. Derive the asymptotic distribution of the sequence of products $\{\hat\theta_{n,1}\hat\theta_{n,2}\}$
\end{exercise}

\begin{solution}
    We can apply the delta method because the function \[g(\theta) = g(\theta_1, \theta_2)=\theta_1 \theta_2\] is continuously differentiable. The asymptotic mean of the transformed sequence is \[g(\theta_0)=\theta_{0,1} \theta_{0,2} = 0\times1=0\] The Jacobian of the function is \[G(\theta)=\begin{bmatrix}
        \frac{\partial g(\theta_1, \theta_2)}{\partial \theta_1} & \frac{\partial g(\theta_1, \theta_2)}{\partial \theta_2}
    \end{bmatrix} 
    = [\theta_2, \theta_1] \] By evaluating at $\theta_0$ we obtain $G(\theta_0) = [1,0]$.\\
    Therefore the asymptotic covariance matrix is
    \[
        G(\theta_0)VG(\theta_0)'= 
            \begin{bmatrix}
                1 & 0
            \end{bmatrix}
            \begin{bmatrix}
                1 & 0 \\
                0 & 1
            \end{bmatrix}
            \begin{bmatrix}
                1 \\
                0
            \end{bmatrix}=
            \begin{bmatrix}
                1 & 0
            \end{bmatrix}
            \begin{bmatrix}
                1\\
                0 
            \end{bmatrix}=1
    \]
    And we can write $\sqrt{n}\hat\theta_{n,1}\hat\theta_{n,2} \overset{d}{\to} N(0,1)$
\end{solution}




\begin{example}[Nerlove's returns to scale]
    \[ \log TC_i = \beta_1 + \beta_2 \log Q_i + \beta_3 \log p_{C_i} + \beta_4 \log p_{L_i}+ \beta_5 \log p_{F_i} + \epsilon_i \]Suppose we want to study the asymptotic confidence region of the normalised regression with coefficients $\alpha = (\beta_3/\beta_2,\beta_4/\beta_2,\beta_5/\beta_2)'$ (i.e. the powers of the Cobb-Douglas production function). Define \[ g(\beta) = \begin{bmatrix}
        \beta_3/\beta_2\\
        \beta_4/\beta_2\\
        \beta_5/\beta_2
    \end{bmatrix}
\]
\[
    G(\beta) = \frac{\partial g(\beta)}{\partial \theta'} = \begin{bmatrix}
        0 & -\beta_3/\beta_2^2 & 1/\beta_2 & 0 & 0\\
        0 & -\beta_4/\beta_2^2 & 0 & 1/\beta_2 & 0\\
        0 & -\beta_5/\beta_2^2 & 0 & 0 & 1/\beta_2\\
    \end{bmatrix}
\]
Thus considering the Wald statistic with $H_0: \hat \alpha = \alpha$, i.e.: $R=I_3, q=\alpha$:
\begin{align*}
    W &= \frac{\left(R\hat\beta-q\right)'\left(\sigma^2R\left(X'X\right)^{-1}R'\right)^{-1}\left(R\hat\beta-q\right)}{\hat\sigma^2/\sigma^2}\\
    &= \frac{\sqrt{n}\left(\hat \alpha - \alpha \right)'\left(\sigma^2R\left(\frac{1}{n}X'X\right)^{-1}R'\right)^{-1}\sqrt{n}\left(\hat \alpha - \alpha\right)}{\hat\sigma^2/\sigma^2}\\
    &\overset{d}{\to} [N(0,I_3)]'I_3 N(0,I_3)\quad \text{(using theorem 6.2.2)}\\
    &= \chi^2(3)
\end{align*}
Hence the asymptotic 95\% confidence set for $\alpha$ is the ellipsoid
\[ \left\{ \alpha : (\hat \alpha - \alpha)'\left( G(\hat\beta)\hat\sigma^2(X'X)^{-1}G(\hat\beta)'\right)(\hat \alpha - \alpha) \leq \chi^2_{0.95}(3) \right\} \]


\end{example}

\end{document}